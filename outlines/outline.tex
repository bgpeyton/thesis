\documentclass[journal=jpcafh]{achemso}
\setkeys{acs}{articletitle = true}
\SectionNumbersOn
\usepackage{amsmath}

% Lin Resp equations
\def\bra#1{\langle#1 |}
\def\ket#1{| #1 \rangle}
\def\linresp#1#2{\langle\langle#1;#2\rangle\rangle}

% Lists
\usepackage{enumitem}

% Graphics
\usepackage{caption}
\usepackage{subcaption}
\usepackage{graphicx}
\usepackage{wrapfig}

\title{Alternative Methods for the Calculation of Molecular Properties using Electronic Structure Theory}
\author{Benjamin G. Peyton}
\affiliation{Department of Chemistry, Virginia Tech, Blacksburg, VA 24061, USA}

\begin{document}
\section*{Outline} \label{outline}
\begin{enumerate}
    \item Abstract
    \item Introduction
        \begin{enumerate}
            \item Conventional EST for molecular properties
            \item Fragmentation
            \item Machine learning
            \item RT-EST
        \end{enumerate}
    \item Theory
        \begin{enumerate}
            \item Ground-state EST 
                \begin{enumerate}
                    \item Orbital-invariant MP2
                    \item CC
                    \item One/two particle density matrices
                \end{enumerate}
            \item Local correlation
            \item Response theory
                \begin{enumerate}
                    \item Dipole expansions to property tensors 
                    \item Quasi-energy to response functions
                \end{enumerate}
            \item Real-time CC
            \item Fragmentation schemes
            \item Machine learning
        \end{enumerate}
    \item Paper 1 
    \item Paper 2 
    \item Paper 3 
    \item Conclusions
\end{enumerate}

\end{document}
