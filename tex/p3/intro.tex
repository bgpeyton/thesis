% Introduction

\section{Introduction} \label{se:intro} Dynamic molecular properties induced
by the absorption, scattering, or refraction of an electromagnetic field
(EMF) give rise to a number of experimental techniques for the detailed
investigation and characterization of molecular light-matter interactions
and structure.\cite{Barron2004} Among these properties are absorbance,
circular dichroism (CD), birefringence, Raman scattering, and many more.
These techniques are essential for modern synthetic chemistry in both
research and industrial settings.

Theoretical chemistry has become a ``full partner with experiment''
\cite{Goddard1985} in this regard, providing high-quality benchmark
calculations for affirming or even predicting many molecular properties,
saving time, and increasing certainty in spectral assignments and
molecular structure determination. Computing dynamic properties with
current \textit{ab initio} methods generally involves frequency-domain
perturbation theory, referred to as response theory, to directly calculate
the quantity of interest. \cite{Crawford2006,Norman2011,Helgaker2012}
Coupled cluster (CC) response theory\cite{Koch1990,Pedersen1997} has
emerged as a robust solution to frequency-domain property calculations,
when its cost is not prohibitive.\cite{Crawford2018,Crawford2019} Many
techniques for circumventing the high-degree polynomial scaling of
coupled cluster methods exist, with extensions of these to property
calculations and response theory providing promising results.
\cite{McAlexander2016,Kumar2017,Howard2018,DCunha2021}

However, there are several drawbacks to the response
formalism.\cite{Langhoff,Goings2018,Li2020} First and foremost, the
perturbations must be ``small'' relative to the intramolecular forces
present in the system. This immediately precludes the possibility of
simulating high-energy experiments such as X-ray spectroscopy, which
have numerous applications in materials science and beyond. Second,
only broadband excitations can be modeled straightforwardly: response
theory typically assumes a single, uniform ``kick'' perturbation
across all frequencies. Experimental apparatus, on the other hand,
can make use of complex, multi-phase procedures involving tuned laser
pulses, pump-probe analysis, \textit{etc}.\cite{Maiuri2020} Finally,
temporally controlled multi-photon events such as high harmonic generation
(HHG)\cite{Lewenstein1994,Gorlach} lie outside the realm of the response
formalism. Together, these drawbacks mean a wide variety of experiments
cannot be predicted or supplemented with response theory calculations. To
overcome this, we must move to non-perturbative, time-domain electronic
structure theory, \cite{Goings2018,Crawford2019,Li2020} where there are
fewer limitations on the form of the perturbing EMF.

The alternative of real-time CC (RTCC) methods has been discussed
nearly as far back as the origins of CC itself in the realm of nuclear
physics.\cite{Hoodbhoy1978,Hoodbhoy1979,Gunnarsson78} More recently, a
renewed interest in real-time coupled cluster has developed for the reasons
discussed above. In the past 10 years, several implementations have been
reported, \cite{Huber2011,Kvaal2012,Nascimento2019,Pedersen2019,Park2019}
with new insights into the aspects of numerical
integration\cite{Pedersen2019,Kristiansen2020} and
interpretation\cite{Pedersen2019,Pedersen2021} as
well as applications for a number of spectral properties.
\cite{Nascimento2016,Nascimento2017,Nascimento2019,Park2019,Park2021b}
Orbital adaptive\cite{Kvaal2012} and orbital optimized\cite{Sato2018}
variants have also explored the limitations of unrelaxed canonical
Hartree Fock orbitals, and the effects of alternative reference
orbitals on the propagation of unphysical imaginary components to
energetics and electric dipole moments. 
Notably absent are studies
on the ability to \textit{reduce} the cost of real-time coupled
cluster methods. 
% TDC comment: "Wouldn't Padé approximants fall into this category?"  
% answer: yes, but no one has used Padé for CC yet (to my knowledge) 

Real-time time-dependent density functional theory
(RT-TDDFT) calculations, a cheaper alternative introduced in the 1990s
(then called the \textit{time-dependent local-density approximation}),
\cite{Yabana1996,Yabana1997,Yabana1999,Bertsch2000} have become routine.
\cite{Lopata2011,Castro2015,Tussupbayev2015,Goings2016a,Bruner2016,Goings2018,Sun2019a,Li2020}
Efforts to reduce the cost of RT-TDDFT have largely focused on reducing
simulation time, utilizing techniques such as Pad\'e approximants to
accelerate the convergence of the Fourier transform,\cite{Bruner2016}
and fitting schemes to avoid the Fourier transform all together,
eliminating the problem of short trajectories resulting in low-resolution
spectra.\cite{Ding2013} Repisky \textit{et al.} introduced the concept of 
dipole pair contributions,\cite{Repisky2015,Kadek2015}
which are typically less complicated than the total electric dipole,
and so these may be individually approximated efficiently using the 
techniques mentioned above.   
\cite{Bruner2016} However, the problems of frequency-domain DFT carry
over directly to the time domain, such as the underestimation of excited
state energies\cite{Peach2008} and difficulties arising from the adiabatic
approximation.\cite{Fuks2013,Fuks2015,Bruner2016} We refer the reader to
a recent, comprehensive review article\cite{Li2020} and citations therein
for a more complete discussion of these challenges. Regardless, the success
of RT-TDDFT under most conditions combined with its drastically reduced
computational cost make it the only viable method for large molecules
at present.

Borrowing from the vast literature of reduced-scaling ground-state or
frequency-domain CC, there are numerous potential candidates for reducing the
cost of RTCC, besides adapting the successful approaches implemented for RT-TDDFT.
First, the standard non-perturbative truncated approaches used
for properties such as CC2\cite{Christiansen1995} and CC3\cite{Koch1997}
are immediately possible, as are property-optimized basis sets.
\cite{Wolinski1990,Sadlej1977,Roos1985,Sadlej1991a,Benkova2005,Baranowska2010,Baranowska2013,Aharon2020a,Howard2018}
Further, details of implementation such as choice of intermediates, the
effects of single- or mixed-precision, or the use of graphical processing
units have only just begun to be explored. \cite{Wang2022} 
An alternative formulation developed by DePrince and Bartlett, 
dubbed the time-dependent equation-of-motion CC (TD-EOM-CC)
\cite{Nascimento2016,Nascimento2017,Nascimento2019,Park2019,Park2021b}
method, reduces the cost by targeting the difficulty of
numerical integration of multiple ``stiff'' coupled differential
equations. By selecting a given moment function to propagate in time,
the coupled sets of $t$- and $\lambda$-amplitude expressions do
not have to be propagated, reducing both the number and difficulty
of numerical integrations required. 

Absent from this list is the family of local correlation methods,\cite{Werner2006}
which have been wildly successful for reduced-scaling approaches to
ground state energies for correlated methods and selected properties.
\cite{Crawford2019,Aharon2020a,DCunha2021,Kodrycka2022} These methods seek to
build a reduced virtual orbital space based on lower-cost criteria, such
as (pair) energies from low-order perturbation theory or atomic orbital
charge analysis. While still only routine for ground-state calculations,
these methods and variants thereof have shown promise in the calculation
of selected response properties.

In this work, we report the first application of local correlation to 
RTCC. This is achieved through a simulation approach,\cite{Hampel1996} 
which forgoes computational savings in favor of algorithmic simplicity, 
for the purposes of rapid exploration and development. 
The effects of occupied and virtual space localization are considered for the simulations of small
hydrogen clusters in the presence of electric field perturbations. Absorption cross sections as
well as electronic CD (ECD) spectra are computed using successively smaller fractions of the canonical
orbital space using the popular 
projected atomic orbital (PAO)
\cite{Pulay1983,Saebo1985,Saebo1986,Saebo1993}
and pair natural orbital (PNO)
\cite{Neese2009,Neese2009a}
schemes. The results are analyzed with respect to full-space simulations. 
Finally, wave function amplitude dynamics are investigated
in order to determine the extent to which these schemes suppress or cause large amplitude 
deviations, which cause instabilities in numerical integration and spurious oscillations in the 
dipole trajectory.
