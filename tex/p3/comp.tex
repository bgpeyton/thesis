% Computational details
\section{Computational Details} \label{se:comp}
The CC wave function for a helical H$_2$ tetramer was propagated
for 1000 a.u., with a time step of 0.02 a.u., in the presence of an explicit electric field. A short pulse
approximating a Dirac delta pulse was applied to generate all possible excited states. Atomic coordinates
are found in the SI.

To approximate a Dirac delta pulse, we apply a narrow time-dependent Gaussian field 
of the form
\begin{equation}
    E(t) = \textrm{F}e^{-\frac{(t-\nu)^{2}}{2\sigma^2}}
\end{equation}
with field strength $\textrm{F}$, center $\nu$, and standard deviation $\sigma$.
The field is propagated in the y-direction, which is along the helical axis of the system.
All calculations in this work use a field defined by $\textrm{F} = 1\times 10^{-3}$,
$\nu = 0.05$, and $\sigma = 0.01$, all in atomic units. Electric and magnetic 
dipole moments were damped using a damping function of the form $e^{-t\tau}$, 
with $\tau = 150$.

The reference simulation was performed in the MO space following a localization of the occupied
orbitals using the Pipek-Mezey procedure.\cite{Pipek1989} All PNO and PAO spaces were also 
built following the same occupied orbital localization. These simulations
were then repeated in both the PNO and PAO virtual spaces, with cutoffs
corresponding to average virtual orbital domains containing roughly 20\%, 40\%, 60\%,
80\%, and 90\% of the untruncated MO virtual space, as well as untruncated PNO- and PAO-basis
simulations to ascertain the effects of the virtual space localization on
the amplitude dynamics of the wave function. 

The effect of local correlation was computed using a simulation approach\cite{Hampel1996},
in which all tensor contractions are done in the MO basis. In every 
CC iteration, before the 
energy denominator is applied when computing an update to the amplitude tensors, the residuals
are transformed into the local basis using $\textbf{Q}$ and into the semi-canonical
basis using $\textbf{L}$. These matrices are computed from either Eqs.~(\ref{eq:Q_pno}) 
and (\ref{eq:L_pno}), respectively, for PNOs, or Eqs.~(\ref{eq:Q_pao}) and (\ref{eq:L_pao}),
respectively, for PAOs. The residuals in the localized basis $\tilde{r}_\mu$ are 
computed by
\begin{subequations} \label{eq:rotate}
\begin{equation} \label{eq:rotate_r1}
    \tilde{\textbf{r}}_i = \textbf{L}_{ii}^T\textbf{Q}_{ii}^T\textbf{r}_i
\end{equation}
\begin{equation} \label{eq:rotate_r2}
    \tilde{\textbf{r}}_{ij} = \textbf{L}_{ij}^T\textbf{Q}_{in}^T\textbf{r}_{ij}\textbf{Q}_{in}\textbf{L}_{ij}
\end{equation}
\end{subequations}
where $r_\mu$ are the residuals from Eqs.~(\ref{eq:t_res}) or (\ref{eq:l_res}). 
Once the energy denominator is applied, the resulting amplitude step is back-transformed
into the MO basis. This process is also applied to the residual, without the
application of the energy denominator, in every step of the time propagation.
To include 
adequately diffuse basis functions, the cc-pVDZ basis
set augmented with diffuse functions\cite{Dunning1989,Woon1994} was used throughout.  

For absorption spectra, 
the imaginary component of the Fourier transform of the \textit{induced} electric dipole 
$\tilde{\mu} = (\langle\mu\rangle - \mu_0)$ following an electric-field kick 
may be directly divided by the Fourier transform of the field strength to yield the spectrum. 
In the case of circular dichroism spectra, however, it is advantageous to first 
analyze the Fourier transform of the derivative of the field. 
For a Dirac delta pulse $E_\delta(t) = \kappa\delta(t)$, 
the Fourier transform of the derivative yields
\begin{equation}
    \textrm{FFT}[E_\delta] = i\omega\kappa.
\end{equation}
Therefore, for such a field, the CD is proportional to the negative of the 
\textit{real} part of the Fourier 
transform of the induced magnetic dipole. 
In practice, the assumption of a Dirac delta pulse is sufficient, provided
a thin Gaussian or Lorentzian pulse is used. 

Discrete Fourier transformation was done using a wrapper to the \texttt{fft} submodule
of the SciPy python library.\cite{scipy} 
All methods were implemented in the Python-based coupled
cluster package, PyCC\cite{pycc}, a NumPy-based\cite{numpy} open-source code developed 
in the Crawford group for the testing and implementation of novel coupled cluster methods. 
The code utilizes the Psi4 electronic
structure package\cite{Smith2020} for integral generation and computing reference 
wave functions. 
The RTCC code makes use of the \texttt{opt\_einsum} package\cite{opteinsum} for tensor contractions,
and time propagation is performed using an in-house suite of integrators. The integrator
used throughout this work was the fourth-order Runge-Kutta method.\cite{rk}

