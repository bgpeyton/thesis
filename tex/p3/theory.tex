% Theory

\section{Theoretical Background} \label{se:theory}
\subsection{Real-Time Coupled Cluster Theory} \label{ss:rtcc}
Conventional RTCC implementations begin by computing the ground-state right- and left-hand
CC wave function amplitudes $t_{\mu}$ and $\lambda_{\mu}$ via the residual expressions
\begin{subequations}
    \begin{equation} \label{eq:t_res}
        \bra{\mu}\bar{H}\ket{\Phi} = 0
    \end{equation}
    \begin{equation} \label{eq:l_res}
        \bra{\Phi}(1 + \hat\Lambda)[\bar{H},\tau_\mu]\ket{\Phi} = 0
    \end{equation}
\end{subequations}
where $\ket{\Phi}$ is the Hartree-Fock ground state determinant and $\ket{\mu}$ 
are substituted determinants (singles, doubles, \textit{etc}.) obtained by the 
action of the second-quantized excitation and de-excitation operators $\tau_\mu$,
and $\bar{H}$ is the similarity transformed electronic Hamiltonian
\begin{equation}
    \bar{H} = e^{-\hat{T}}\hat{H}e^{\hat{T}}
\end{equation}
with right-hand cluster operators 
\begin{subequations}
    \begin{equation}
        \hat{T} = \sum_\mu^N \hat{T}_\mu
    \end{equation}
    \begin{equation} \label{eq:t_mu}
        \hat{T}_\mu = \sum_{\mu}\tau_\mu t_{\mu}
    \end{equation}
\end{subequations}
and left-hand cluster operators
\begin{subequations}
    \begin{equation}
        \hat{\Lambda} = \sum_\mu^N \hat{\Lambda}_\mu
    \end{equation}
    \begin{equation} \label{eq:l_mu}
        \hat{\Lambda}_\mu = \sum_{\mu}\tau_\mu^\dagger\lambda_\mu.
    \end{equation}
\end{subequations}

The time evolution of the amplitudes is governed by the nonlinear differential equations
obtained through the time-dependent Schr\"odinger equation (in atomic units)
\begin{subequations}
    \begin{equation} \label{eq:diff_t}
        i\frac{\partial t_\mu}{\partial{t}} = \bra{\mu}\bar{H}(t)\ket{\Phi}
    \end{equation}
    \begin{equation} \label{eq:diff_l}
        -i\frac{\partial \lambda_\mu(t)}{\partial{t}} = \bra{\Phi}(1 + \hat\Lambda(t))[\bar{H}(t),\tau_\mu]\ket{\Phi}.
    \end{equation}
\end{subequations}
The right-hand sides of Eqs.~(\ref{eq:diff_t}) and (\ref{eq:diff_l}) are simply the residual
expressions, where we have replaced the Hamiltonian with a time-dependent Hamiltonian. By 
including a field perturbation as a time-dependent addition to the Fock operator, the right-hand
sides may be computed by updating the Hamiltonian from time $t$ to time $t^\prime = t + h$ and 
recomputing the residual expressions. 
This is achieved using a numerical integrator, which produces solutions to equations 
of the form 
\begin{equation}
    \frac{dy(t)}{dt} = f(y,t).
\end{equation}
Here, $y$ is the amplitude tensor, and the function $f$ is the residual expression. 
Multiple integration schemes are possible; for simplicity, we adopt the popular
fourth-order Runge-Kutta integrator,\cite{rk} defined by 
\begin{equation}
\begin{aligned}
    k_1 &= f\left(y,t\right) \\
    k_2 &= f\left(y+\frac{1}{2}hk_1,t+\frac{1}{2}h\right) \\
    k_3 &= f\left(y+\frac{1}{2}hk_2,t+\frac{1}{2}h\right) \\
    k_4 &= f\left(y+hk_3,t+h\right)
\end{aligned}
\end{equation}
with time step $h$, and the propagated tensor is computed as
\begin{equation}
    y(t+h) = y(t) + \frac{1}{6}h(k_1 + 2k_2 + 2k_3 + k4).
\end{equation}
%Numerical integrators are sensitive to the time step $h$. If $h$ is too large, significant
%oscillations in the amplitudes may occur, causing unstable propagation of the amplitudes to 
%positive or negative infinity. 
%This may be tested by observing either the amplitude dynamics or the 
%unphysical imaginary component of the Lagrangian energy expectation value. 
%In our experience, time steps of $0.01$ to $0.1$a.u. are sufficient for most systems subjected
%to a Dirac delta pulse and fields having periods significantly larger than $h$.

\subsection{Properties} \label{ss:prop}
To within the dipole approximation, 
the complex time-domain response tensors 
$\boldsymbol{\alpha}$ and 
$\textbf{G}^\prime$ can be defined by low-order expansions 
of the time-dependent electric and magnetic dipole moment expectation values 
in an electric field $\textbf{E}$ with frequency $\omega$, \textit{viz.}
\begin{subequations} \label{eq:exps}
    \begin{equation} \label{eq:mu_exp}
        \langle\mu\rangle_i = \mu_0 + \alpha_{ij}E_j
    \end{equation}
    \begin{equation} \label{eq:m_exp}
        \langle m\rangle_i = -\frac{1}{\omega}\frac{\partial E_j}{\partial t}G^\prime_{ij}
    \end{equation}
\end{subequations}
where $i$ and $j$ are Cartesian coordinates, and we have suppressed 
the time dependence for clarity.  
%The absorption cross-section $\boldsymbol{\sigma}$ 
%in the time domain 
The dipole strength function $S$
is related to the 
imaginary component of $\boldsymbol{\alpha}$ by
\begin{equation} \label{eq:abs}
    S(t) \propto \textrm{Tr}\left[ \textrm{Im}\left( \boldsymbol{\alpha}(t) \right) \right]. 
\end{equation}
Fourier transformation of $S$ from the time to frequency domain
yields the broadband absorption spectrum. 
The differential molar extinction coefficient is proportional to the imaginary part
of the Rosenfeld $\textbf{G}^\prime$ tensor\cite{Rosenfeld1929} by
\begin{equation} \label{eq:ecd}
    \eta_i(t) \propto -\textrm{Tr}[\textrm{Im}(\textbf{G}^\prime(t))].
\end{equation}
The Fourier transform of Eq.~(\ref{eq:ecd}) yields the ECD spectrum.

We note here two important points. First, we could just as easily
define both $\boldsymbol{\alpha}$ and $\textbf{G}^\prime$ with respect
to the electric dipole expectation value; however, by expanding both
moments in an electric field, we may recover both properties by computing
expectation values of both the electric and magnetic dipole operators
along the same electric field-perturbed trajectory. In principle, we may
compute \textit{any} electric-field-perturbed expectation value from a
single propagation -- this is in contrast to the RT-EOM-CC method, which
propagates a single moment function. Additional expectation values
would require additional moment function propagations.

%Second, the resolution of the resulting Fourier transform is directly
%related to the excited-state lifetimes. As these simulations are
%adiabatic, artificial lifetimes are introduced using an exponential
%damping function (see Section \ref{se:comp}). Furthermore, stick spectra as predicted by EOM-CC
%either contain all three components of the dipole or must be separated by
%symmetry to include only the transitions desired (if possible). We focus on
%the dipole component oriented along the helical axis of spatially arranged
%hydrogen dimers, fixed to be the y-axis. This component is inseparable from
%the x-component by symmetry; thus, we provide comparisons to EOM for only
%a select set of parameters. [NOTE: this may may be relegated to the Comp
%Details] [NOTE: I could probably include all of the comparisons in the SI]

Second, we note that the low-order expansions in Eq.~(\ref{eq:exps}) are
an approximation. The total dipole moments will contain many higher-order
terms; however, at the field strengths used in this work, these effects
are expected to be negligible. These terms can be separated and have been
examined in the context of real-time simulations of X-ray absorption 
spectroscopy (XAS).\cite{Park2021b}
While very important to the advantages of the RTCC method, these effects
are beyond the scope of the current work.

\subsection{Local Correlation} \label{ss:local}
\subsubsection{Projected Atomic Orbitals} \label{sss:pao}
In the PAO method, the virtual space is localized using a linear combination of $N$ atom-centered 
atomic orbital (AO) basis functions. For every occupied orbital $i$, the contribution to the 
Mulliken charge $q$ from every atom $j$ is computed
\begin{equation}
    q_j = \sum_{\mu\in j}\sum_\lambda S_{\mu\lambda}C_{\lambda i}C_{\mu i}
\end{equation}
where $\mu$ is the set of atomic orbitals centered on atom $j$, and $\lambda$ runs over all AOs.
$\textbf{S}$ and $\textbf{C}$ are the AO overlap matrix and 
the Hartree-Fock molecular orbital (MO) coefficients, respectively.
Atoms (and their AO basis functions) are 
added to the domain of orbital $i$ one at a time based on decreasing charge contribution. 
Each time the domain is altered, new coefficients $\textbf{C}^{\prime}$ are computed by solving
\begin{equation}
    \sum_{\nu\in i}S_{\mu\nu}C^{\prime}_{\nu i} = \sum_{\lambda}S_{\mu\lambda}C_{\lambda i}
\end{equation}
where $\mu$ and $\nu$ are atomic orbitals belonging to domain ${i}$, and $\lambda$
runs over all AOs.
The Boughton-Pulay completeness value $b_i$ 
\begin{equation}
    b_i = 1 - \sum_{\mu}\sum_{\lambda}C^{\prime}_{\mu i}S_{\mu\lambda}C_{\lambda i}
\end{equation}
is then compared to a chosen cutoff, $\delta_{PAO}$. If $b_i$ is above the cutoff, the 
next atom is added to the domain of $i$, and the process is repeated until the cutoff value
is met. Orbitals with negligible norms less than another chosen parameter $\delta_{norm}$
(usually corresponding to contributions from core orbitals prone to linear dependencies)
are also removed from each domain.

Once the domains are assigned, occupied \textit{pairs} $ij$ are then assigned \textit{pair} 
domains, based on the union of the domains of the two occupied orbitals. 
An occupied-space projector $\tilde{\textbf{C}}^\prime$ is computed  
\begin{equation}
    \tilde{\textbf{C}}^\prime = \textbf{1} - \textbf{DS}
\end{equation}
where $\textbf{D}$ is (half of) the closed-shell Hartree-Fock one-particle density matrix. 
$\tilde{\textbf{C}}^\prime$ serves to remove contributions from the occupied orbital space, creating
the redundant PAO space. Linear dependencies are removed by diagonalizing the projected 
overlap matrix
\begin{equation}
    \tilde{\textbf{S}}_{ij} = \tilde{\textbf{C}}^{\prime T}_{ij}\textbf{S}\tilde{\textbf{C}}^{\prime}_{ij}
\end{equation}
where $\tilde{\textbf{C}}^\prime_{ij}$ contains only the columns for atomic orbitals belonging to
the domain of pair ${ij}$. PAOs which correspond to eigenvalues of $\tilde{\textbf{S}}_{ij}$ 
below one final parameter, 
$\delta_{lin}$, are then removed.
The orbitals are normalized to yield the non-redundant PAO basis for a given pair, 
$\tilde{\textbf{C}}_{ij}$.

Two transformation matrices are required to transform MO-basis quantities 
into the PAO basis. The first, which rotates from the MO to the redundant PAO
basis for a given pair, is computed as
\begin{equation} \label{eq:Q_pao}
    \textbf{Q}^{PAO}_{ij} = \textbf{C}^{\dagger}_{ij}\textbf{S}\tilde{\textbf{C}}^\prime_{ij}.
\end{equation}
To facilitate the use of the usual orbital energy denominator terms during the update of the 
amplitude equations at every iteration, a semi-canonical virtual basis for pair $ij$,
$L^{PAO}_{ij} = \chi_{ij}\tilde{C}$,
is found by diagonalizing the Fock matrix in the space of non-redundant PAOs $\tilde{F}$:
\begin{equation} \label{eq:L_pao}
    \tilde{F}\chi_{ij} = \epsilon_{ij}\chi_{ij}
\end{equation}
where $\epsilon_{ij}$ are the semi-canonical orbital energies for the virtual space of 
occupied pair $ij$. 

\subsubsection{Pair Natural Orbitals}
PNOs are obtained by diagonalizing a pair density built in a space of localized occupied 
orbital pairs. The MP2-level pair density is defined as
\begin{equation} \label{eq:pair_D}
    \textbf{D}_{ij} = 2\frac{\textbf{T}_{ij}\tilde{\textbf{T}}_{ij}^\dagger + \textbf{T}_{ij}^\dagger\tilde{\textbf{T}}_{ij}}{1+\delta_{ij}}
\end{equation}
with $\tilde{\textbf{T}}_{ij} = 2\textbf{T}_{ij} - \textbf{T}_{ij}^\dagger$.
Diagonalizing $\textbf{D}_{ij}$ yields the transformation matrix from the MO to the PNO basis
$\textbf{Q}^{PNO}$:
\begin{equation} \label{eq:Q_pno}
    \textbf{D}_{ij}\textbf{Q}^{PNO}_{ij} = \textbf{D}_{ij}n_{ij}
\end{equation}
with occupation numbers $n_{ij}$. Truncation of the space is done by removing PNOs which 
correspond to occupation numbers below a cutoff, $\delta_{PNO}$. As before, the transformation
matrix from the PNO to a semi-canonical PNO basis is found by diagonalizing the Fock matrix
in the space of PNOs:
\begin{equation} \label{eq:L_pno}
    \tilde{\textbf{F}}\textbf{L}^{PNO}_{ij} = \epsilon_{ij}\textbf{L}^{PNO}_{ij}.
\end{equation}
