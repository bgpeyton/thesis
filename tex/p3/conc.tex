%conc.tex

\section{Conclusions} \label{conc} Here we present the first application
of local correlation to RTCC simulations. The popular PNO and PAO virtual
space localization schemes are applied to the calculation of dynamic electric
and magnetic dipole moments in the presence of an explicit electric field,
providing absorption and ECD spectra, respectively. 
For a helical H$_2$ tetramer, truncation of the
localized virtual space to successively larger fractions of the canonical
virtual space resulted in convergence to the canonical result; however, this
convergence is slow, and errors in excitation energies and intensity are
present even in some of the largest spaces tested, especially for ECD. This
corroborates the results of recent studies applying locally correlated
methods to the prediction of dynamic properties in the frequency domain
using response theory.

%As in the frequency domain, there are a number of possible approaches
%to building a more appropriate virtual space which preserves accuracy in
%response properties upon aggressive truncation. 
Examining the amplitude
dynamics during the propagation, it is shown that the $t_1$ and $\lambda_1$
amplitudes respond most strongly to the field -- a large increase in the
norm of these matrices is observed upon application of the field, followed
by a steady oscillation. The $t_2$ and $\lambda_2$ tensors, by comparison,
remain relatively static throughout. 
These oscillations are largely, but not completely, localized to a selection 
of only a few orbitals, as evidenced by consideration of 
time-dependent deviations in the $t_1$ amplitudes from the ground-state.
In the localized virtual spaces tested,
these oscillations are delocalized throughout the $t_1$ and $\lambda_1$
matrices. 

Orbital extent alone cannot explain the shortcomings of 
the PNO space; however, its effect is significant. 
These results provide an insight into the importance of 
singly-substituted determinants in the time-dependent wave function in the 
presence of an electric field, as well as a potential metric to gauge the 
performance of new localization schemes for frequency- or time-domain 
calculations of dynamic response properties.
In order to attain a balanced description of wave function components 
important for both energy and property calculations, the combination 
of appropriately determined spaces such as the combined PNO++ approach
has been fruitful.\cite{DCunha2021} 
Still neglected in this approach
are the singles amplitudes, which are absent in the MP2 wave functions
used to approximate the occupied pair domains. Schemes to include 
these effects, such as approximate CC2-level $t_1$ guess amplitudes,
may further improve the space and allow greater flexibility for 
truncation. The prospect of utilizing these 
methodologies within the current framework is promising, and work is 
underway to explore their efficiency.
