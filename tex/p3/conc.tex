%conc.tex

\section{Conclusions} \label{conc}
Here we present the first application of local correlation to RTCC
simulations. The popular PNO and PAO virtual space localization schemes are
applied to the calculation of dynamic electric and magnetic dipole moments
in the presence of an explicit electric field, providing absorption and
ECD spectra, respectively.  Truncation of the localized virtual space to
successively larger fractions of the canonical virtual space result in
convergence to the canonical result; however, this convergence is slow,
and errors in excitation energies and intensity are present even in the
largest spaces tested, especially for ECD. This corroborates the results
of recent studies applying locally correlated methods to the prediction
of dynamic properties in the frequency domain using response theory.

As in the frequency domain, there are a number of possible approaches
to building a more appropriate virtual space which preserves accuracy in
response properties upon aggressive truncation. Examining the amplitude
dynamics during the propagation, it is shown that the $t_1$ and $\lambda_1$
amplitudes respond most strongly to the field -- a large increase in the
norm of these matrices is observed upon application of the field, followed
by a steady oscillation. The $t_2$ and $\lambda_2$ tensors, by comparison,
remain relatively static throughout. 
These oscillations are largely, but not completely, localized to a selection 
of only a few orbitals.
%of orbitals with large spatial extent in the LMO basis. 
In the localized virtual spaces tested,
these oscillations are delocalized throughout the $t_1$ and $\lambda_1$
matrices. Orbital extent alone cannot explain the shortcomings of 
the PNO space; however, its effect is significant. 
These results provide an insight into the importance of 
singly-substituted determinants in the time-dependent wave function in the 
presence of an electric field, as well as a potential metric to gauge the 
performance of new localization schemes for frequency- or time-domain 
calculations of dynamic response properties.
