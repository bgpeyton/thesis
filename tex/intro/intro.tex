\chapter{Introduction} \label{ch:int}

Modern synthetic organic chemistry employs a vast array of sophisticated instrumentation.
Principle among these are probes of light-matter interaction, which reveal rich structural and 
electro-magnetic characteristics. These instruments work by measuring the absorption, 
reflection, or refraction of an electromagnetic field (EMF) interacting with a target system. 
Experimental techniques which measure these interactions with respect to the energy or frequency
of the incident EMF are known as spectroscopy. Many staple experimental apparatus probe these
relationships, including (but certainly not limited to)
ultraviolet-visible absorption (UV-Vis), 
optical rotatory dispersion (ORD), 
flame atomic absorption (flame AA),
and electronic circular dichroism (ECD).
These data can be used for characterization or establishing structure-property relationships 
which aid in the development of novel systems for applications in materials, bio-organics, 
and more. 
Interpreting or predicting the results of such experiments requires a knowledge of fundamental 
light-matter interactions on a quantum level.  

A quantum description of any molecular interaction may be viewed as the effects of a perturbation
on a quantum-mechanical system. In the case of EMF, this perturbation may be ``static'' (fixed)
or ``dynamic'' (varying in frequency or time). A fully quantum-mechanical description of the 
light-matter system would require quantum electrodynamics (QED); however, it is often sufficient
to treat the EMF from a classical perspective, treating only the molecular response using
quantum mechanics. This allows us to utilize many well-established methods in the field of 
theoretical chemistry. These generally provide a representation of the system (a wave function or 
density) with which to take the expectation value of a given operator or, in some cases, predict 
the expectation value directly. Methods which utilize only mathematical techniques (that is, no
experimental or phenomenological parameters) are said to be \textit{ab initio} methods. These 
methods often (but not always) simplify or ignore the quantum effects of the nuclei and their 
motion, an approximation known as the Born-Oppenheimer approximation. 
Another subclass of \textit{ab initio} methods, which 
center around solving the time-dependent or time-independent Schr\"odinger wave equation to 
obtain an explicit form of the wave function, are 
known as wave function-based methods. Finally, methods that go beyond the mean-field or Hartree-Fock 
approximation, in which electrons only interact through an average self-consistent field, are 
classified as ``correlated'' methods. 
It is these correlated wave function-based methods under the Born-Oppenheimer
approximation upon which the bulk of the current work is built. As such, following this in Chapter 
\ref{ch:theory} is a primer in the theoretical underpinnings necessary to understand this work which are not described 
in detail in the publications that follow in sections [PAPER 1-3]. Chapter \ref{ch:theory} is roughly 
divided into details of electronic structure theory (Section \ref{se:est}) and molecular response 
properties (Section \ref{se:res}). 

First and foremost is many-body perturbation theory (MBPT). This framework allows us to 
separate the quantum mechanical properties of the isolated system versus some perturbing 
force, which is expanded in orders and truncated under the assumption that higher-order terms 
become negligible.
This force is often taken to be electron correlation when describing the electronic ground 
state wave function; however, it may also be a static or dynamic EMF. In section (\ref{ss:mp2})
we present some basic characteristics of general MBPT, in preparation for its application in the 
chapters that follow.

Secondly, in section (\ref{ss:cc}) we present the wave function-based method upon which the bulk 
of this work is based or seeks to approximate: coupled cluster (CC) theory. 
This method, like perturbation theory, is most commonly used for computing electron correlation 
(though its roots are in nuclear physics). 
Unlike perturbation theory, we do not perform an order-by-order expansion of the perturbation; 
instead, a ``cluster'' operator folds in contributions based on a physical intuition, that is the 
instantaneous occupation of many quantum states which give rise to electron correlation. Truncation
is then based on the number of quantum states (substituted or `excited` determinants). This method,
while accurate and systematically improvable, is notoriously expensive, suffering from high-order
polynomial scaling. Extensions to excited states and molecular response properties only compound this 
issue; as such, [PAPER 1] and [PAPER 2] are focused on ways to circumvent using CC on any but a small 
subset of systems. (It should be noted that, while [Paper 1] utilizes a different correlated method 
based on a density-functional theory approach, the primary goal was to ascertain the effects of 
fragmentation of a system through the many-body expansion on the computation of molecular properties. 
This is a benchmark study, whose goal is understood to be conclusions that would be applied to more 
expensive methods such as CC, where such benchmark studies could not be performed.)
Additional considerations of reduced density matrices, which are used throughout the work to refer
to correlated wave functions, follows in Section \ref{ss:rdm}.

Finally, we describe the method by which we couple the classical perturbing field to the 
quantum-mechanical correlated wave function. 
By applying perturbation theory we derive general expressions for tensors which describe 
the molecular response to an EMF. In the exact theory, these tensors turn out to be functions of the 
excited electronic states of the system; however, an approach to generalize 
these expressions to approximate theories such as truncated coupled cluster is discussed, which also 
avoids the costly evaluation of excited-state wave functions. For dynamic EMF, these properties may be 
described as a function of frequency, so the most common approach is to derive the working equations
using a Fourier transform of expressions obtained using the time-dependent Schr\"odinger equation 
(more generally known as response theory); however, these expressions may also be evaluated explicitly
in the time-domain. In Section \ref{se:res} we focus on the frequency-domain formulation, while
[PAPER 3] explores the recently-revived prospect of explicit time-propagation, as well as a 
technique to make this process cheaper through a well-established concept known as 
``local correlation''. 
