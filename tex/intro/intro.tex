\chapter{Introduction} \label{ch:int}

Modern synthetic organic chemistry employs a vast array of sophisticated
instrumentation.  Principal among these are probes of light-matter
interactions, which reveal rich structural and electro-magnetic
characteristics. These instruments work by measuring the absorption,
scattering, or refraction of an electromagnetic field (EMF) interacting
with a target system.\cite{Barron2004} Experimental techniques which
measure these interactions with respect to the energy or frequency of
the incident EMF are known as spectroscopy. Many staple experimental
apparatus probe these relationships, including (but certainly not limited
to) 
ultraviolet-visible absorption (UV-Vis), 
nuclear magnetic resonance (NMR),
electronic and vibrational circular dichroism (ECD and VCD), and 
optical rotatory dispersion (ORD).
These data can be used for characterizing synthetic products or
establishing structure-property relationships which aid in the development
of novel systems for applications in materials, bio-organics, and more.
Interpreting or predicting the results of such experiments requires a
knowledge of fundamental light-matter interactions on a quantum level.

A quantum description of any molecular interaction may be viewed as the
effects of a perturbation on a quantum-mechanical system. In the case of EMF,
this perturbation may be static (fixed) or dynamic (varying in frequency
or time). A fully quantum-mechanical description of the light-matter
system would require quantum electrodynamics (QED); however, it is often
sufficient to treat the EMF from a classical perspective, treating only the
molecular response using quantum mechanics.\cite{Barron2004,Helgaker2012}
This allows us to utilize many well-established methods in the field of
theoretical chemistry. These generally provide a representation of the
system (a wave function or density) with which to take the expectation
value of a given operator or, in some cases, predict the expectation value
directly.\cite{Szabo1996} Methods which utilize only mathematical techniques
(that is, no experimental or phenomenological parameters) are said to be
\textit{ab initio} methods. These methods often (but not always) simplify or
ignore the quantum effects of the nuclei and their motion, an approximation
known as the Born-Oppenheimer approximation.\cite{born1927zur} Another
subclass of \textit{ab initio} methods, which centers around solving the
time-dependent or time-independent Schr\"odinger wave equation to obtain
an explicit form of the wave function, are known as wave function-based
methods. Finally, methods that go beyond the mean-field or Hartree-Fock
approximation, allowing electrons to interact through more than an average
self-consistent field, are classified as correlated methods.  It is
these correlated wave function-based methods under the Born-Oppenheimer
approximation upon which the bulk of the current work is built. As such,
following this in Chapter \ref{ch:theory} is a primer in the theoretical
underpinnings necessary to understand this work.  The focus will be
on fundamentals which are not explicitly presented in the publications
that follow in Chapters~\ref{ch:p1}-\ref{ch:p3}. Chapter \ref{ch:theory}
is roughly divided into details of electronic structure theory (Section
\ref{se:est}) and molecular response properties (Section \ref{se:res}).

The first topic discussed is many-body perturbation theory
(MBPT).\cite{Bartlett2009} This framework allows us to separate the quantum
mechanical properties of the system from the effects of some perturbing
force, which is expanded in orders and truncated under the assumption
that higher-order terms become negligible.  This force is often taken to
be electron correlation when describing the electronic ground state wave
function;\cite{Moller1934} however, it may also be a static or dynamic
EMF. In Section (\ref{ss:mp2}) we present some basic characteristics of
general MBPT, in preparation for its application in the chapters that follow.

Second, in Section (\ref{ss:cc}) we present the wave
function-based method which the bulk of this work is either
based upon or seeks to approximate: coupled cluster (CC) theory.
\cite{Sinanoglu1964,Cizek1966,Cizek1969,Crawford2000} This method,
like perturbation theory, is most commonly used for computing electron
correlation (though its roots are in nuclear physics).  Unlike perturbation
theory, we do not perform an order-by-order expansion of the perturbation;
instead, a cluster operator folds in contributions based on a physical
intuition, 
that is the substitution of electrons in occupied orbitals into
unoccupied or virtual orbitals,
%that is the instantaneous occupation of many quantum states
which gives rise to electron correlation. 
Truncation is then based on the
%number of quantum states (substituted or often-called ``excited'' determinants). 
number of simultaneous electron substitutions allowed -- singles, doubles, \textit{etc}.
This
method, while accurate and systematically improvable, is notoriously
expensive, suffering from high-order polynomial scaling. Extensions to
excited states and molecular response properties only compound this issue;
\cite{Hoodbhoy1979,Crawford2006,Helgaker2012,Crawford2019} as such, Chapters~\ref{ch:p1}
and \ref{ch:p2} are focused on ways to circumvent using CC on any but a small
subset of systems. (It should be noted that, while Chapter~\ref{ch:p1} utilizes a
different correlated method based on a density-functional theory approach,
the primary goal was to ascertain the effects of fragmentation of a system
through the many-body expansion on the computation of molecular properties.
This is a benchmark study, whose conclusions are understood to be applicable
to more expensive methods such as CC, where such benchmark studies could
not be performed.)  Additional considerations of reduced density matrices,
\cite{RDM1976,Harris1992,pinkbook} which are used throughout the work to
refer to correlated wave functions, follows in Section \ref{ss:rdm}.

Finally, we describe the method by which we couple the classical perturbing field to the 
quantum-mechanical correlated wave function. 
By applying time-dependent perturbation theory\cite{Langhoff} 
we derive general expressions for tensors which describe 
the molecular response to an EMF. In the exact theory, these tensors are functions of the 
excited electronic states of the system; however, an approach to generalize 
these expressions to approximate theories such as truncated coupled cluster is discussed,
\cite{Koch1990,Pedersen1997,Christiansen1998,Norman2011}
which also avoids the costly evaluation of excited-state wave function parameters. 
For dynamic EMF, these properties may be 
described as a function of frequency, so the most common approach is to derive the working equations
using a Fourier transform of expressions obtained using the time-dependent Schr\"odinger equation 
(more generally known as response theory); however, these expressions may also be evaluated explicitly
in the time-domain.
\cite{Goings2018,Li2020}
In Section \ref{se:res} we focus on the frequency-domain formulation, while
Chapter~\ref{ch:p3} explores the recently-revived prospect of explicit time-propagation, as well as a 
technique to make this process cheaper through a well-established concept known as 
local correlation.\cite{Werner2006} 
