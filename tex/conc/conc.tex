\chapter{Conclusions} \label{ch:conc} The time- or frequency-dependent
response of a system to electromagnetic fields gives rise to many different
dynamic properties which can be experimentally probed by various forms
of spectroscopy. Theoretical models of electronic structure can provide
assistance in the design and characterization of novel materials through
\textit{ab initio} computation of the light-matter interactions which
govern this response. The accurate computation of these properties on
modern computing hardware with correlated methods, such as coupled cluster
theory, is often prohibitively expensive for all but the smallest systems.
These computations typically employ a semi-classical approach, which
models the response of a system, typically treated as a quantum-mechanical
wave function or density,
to an external electromagnetic field, represented as a classical potential.
The current work seeks to improve the efficiency of these methods through
considerations of both frequency- and time-domain formulations.

After discussing the basics of correlated electronic structure treatments
and the foundations of semi-classical approaches to property prediction,
three studies were presented. First, we examine the effects of basis set
superposition error on the many-body expansion of molecular properties,
computed using the frequency-domain response theory approach. Previous
efforts to apply this divide-and-conquer approach to dynamic molecular
properties proved difficult -- specifically, slow, oscillatory convergence
to the analytic result was observed, especially in the case of optical
rotation calculations. Literature precedent suggests that the correction of
basis set superposition errors, along with careful considerations of other
methodological parameters such as numerical precision, alleviate oscillatory
convergence issues commonly encountered when calculating binding energies of
large water clusters through a many-body approach. While these corrections
proved to dampen the oscillations, the troublesome case of optical rotation
for systems in modest solvent shells could still not be reproduced with
computationally tractable expansions. Thus, fragmentation-type approaches
such as the many-body expansion are of limited utility in such cases.

Moving beyond simple fragmentation of the system, a condensed wave function
representation of the entire system could allow for the rapid evaluation of
target properties. The second study utilized a machine-learning approach. The
input is a representation of molecular systems, developed using wave
function-based descriptors. This representation, dubbed the density tensor
representation, was derivative of the recently published $t$-amplitude tensor
representation, but unique in its construction by exploiting the naturally
dense representation of the wave function by the reduced density matrix. A
simple machine-learning method, kernel ridge regression, was applied to
compute the ground-state energies and electric dipole moments of several
small organic molecules across molecular dynamics trajectories. MP2-level
densities were used to build representations, and the training set
consisted of CCSD-level ground-state correlated energies and electric
dipoles. Utilizing new implementations of both $t$-amplitude and density
tensor representations in an open-source, python-based code proved the
density tensor representation to be the more efficient representation
of the two, requiring less than a dozen training calculations to provide
results within chemical accuracy across short trajectories for the systems
considered. Proposed extensions of the procedure to dynamic properties
were explored, and the implementation of these methods is ongoing.

Using frequency-domain response theory to compute training set values
results in certain limitations. This perturbative approach excludes the
possibility of strong electromagnetic fields, such as those used in ultrafast
spectroscopy. These properties may be computed using a non-perturbative,
explicitly time-dependent approach, referred to as real-time electronic
structure theory. These methods require the expensive (and possibly unstable)
propagation of the wave function in time. For real-time coupled cluster,
this means repeatedly evaluating the high-degree polynomial-scaling residual
expressions for long periods of time. To alleviate the cost, the third and
final study focused on the application of the local correlation family of
reduced-scaling methods to real-time coupled cluster.

An open-source coupled cluster development code based on the Psi4 electronic
structure package was developed which can simulate the effects of various
local correlation schemes, namely PAO and PNO, with minimal modifications
of the underlying expressions. The code is capable of modeling explicit
electric fields, and simulations of both absorption and electronic
circular dichroism spectra are used to test the efficiency of the PAO and
PNO approaches. These tests revealed deficiencies in the virtual orbital
spaces produced by truncating the wave function in these locally correlated
bases, which are corroborated by recent results of their application in
frequency-domain response theory. The explicitly time-dependent nature of
the wave function allowed for the detailed investigation of separate wave
function elements as they evolved in time, revealing a strong dependence
on single substitutions, particularly in diffuse orbital spaces. These
substitutions are not considered in the typical construction of PAOs
or PNOs. These bases are also intentionally spatially localized, since
they are designed to capture the correlation energy, which is inherently
local for insulators and other weakly-correlated systems. Alternative 
approaches, such as the recently-proposed PNO++ method,
may address these problems, allowing for more aggressive truncation of the
virtual orbital space, leading to far cheaper evaluation of the expensive
residual expressions required for time propagation.

The work presented here comprises several different methods for improving
the efficiency of accurate, systematically improvable computations of 
field-dependent molecular properties. All solutions have been implemented
as open-source software, which can be freely accessed, used, and extended.
Continued development on the methods presented in Chapters~\ref{ch:p2} 
and \ref{ch:p3}, in particular, are encouraging. 
%Active development continues on extensions to the methods described
%in Chapters~\ref{ch:p2} and \ref{ch:p3}. For the former, these
%include extensions to dynamic properties, and an integration with the
%QCArchive computing ecosystem. For the latter, the implementation of
%perturbation-aware truncation criterion and production-level algorithms is
%currently underway. The capability to use cheaper coupled cluster methods,
%such as CC2, is nearing completion. Finally, preliminary results of a pilot
%implementation of an approach specifically targeting the singles amplitudes,
%avoiding the expensive doubles residual expressions all together, have
%been encouraging. 
In the near term, these and other improvements will continue to close the
gap between experimental and theoretical understandings of light-matter
interactions. In the far term, as this gap is closed and sufficient computational
resources become more widely available, the theoretical prediction of 
field-induced properties will become an indispensable tool for future 
synthetic chemists, whose interpretations of spectroscopic results will be 
improved by a fundamental understanding of these interactions at a quantum level.
