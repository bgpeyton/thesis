%conc.tex%
\section{Conclusions} \label{conc}
Here we introduce the density tensor representation (DTR) for machine-learning quantum mechanics applications. 
The representation is based on the previous t-amplitude tensor representation (TATR), with improvements made through strictly theoretical considerations of three categories: systematic improvement, storage, and simplified representation-target mapping. Investigating the limits of these categories on small test sets show a number of favorable properties. 
The DTR can be easily defined for any electronic structure method in which a density can be defined. When compared to the TATR for MP2, it achieves superior accuracy across most test cases when the MP2 wave function is expected to produce reasonable results. This accuracy is in the sub-mE$_h$ range for correlation energies. 
Furthermore, applications to molecular properties are both theoretically and operationally justified for representations utilizing electronic densities as raw wave function features. 
Roughly milliDebye error was achieved for correlated electronic dipole moments of several small molecules near equilibrium.
Extensions to include additional properties and molecular transferability are also considered, with the data-efficient DTR model providing a vital stepping stone to these generalizing improvements. 
