\section{Real-Time Coupled Cluster Theory} \label{se:rtcc}
As mentioned in Section \ref{se:res}, a time-domain approach to the calculation of molecular properties
is possible. Previous work has been done in the realms of configuration interaction (CI) and density
functional theory (DFT), and more recently flavors of real-time coupled cluster theory (RT-CC) have been
applied to small systems in a variety of field perturbations. Here we summarize the advantages,
disadvantages, and theoretical underpinnings of real-time electronic structure theory, with a 
specific focus on RT-CC techniques.

CC theory, particularly the CC singles and doubles variant with a perturbative triples correction (CCSD(T)),
has been called the ''gold standard" of electronic structure theory. Following its success in ground-state 
theory, the application of CC to molecular response properties has been successful in the case of small 
molecules in simple isotropic fields.
However, two areas of extraordinary experimental interest have yet to be thoroughly explored with CC: 
``strong" field perturbations (on the order of the strength of molecular interactions) such as X-Ray
spectroscopy, and shaped-pulse perturbations, implicated in \textit{e.g.} so-called pump-probe experiments.
These gaps in the literature are due to the failings of response theory under these conditions. 
Hinging on the assumption of relatively weak perturbations, response theory breaks down in the case 
of extremely strong fields [PROBABLY NEED A NUMBER HERE]. Also implicit in the perturbative treatment
in both exact and approximate response theory is the broadband excitation of the system into any and 
all possible excited states. This precludes the possibility of modeling the response to tuned laser pulses
which target specific excited states or ranges of excited states. Both of these cases are possible in an
explicitly time-dependent formalism.

Extending ground-state CC to a time-dependent formalism is achieved through the introduction of 
time-dependent amplitudes. This process is exactly analogous to the derivation of the wave function
parameters in dynamic response theory, ~Eqs.(\ref{eq:tdse}) - (\ref{eq:sot}). The residual expression
~Eq.(\ref{eq:cc_res}) for time-dependent cluster amplitudes $\tau^\mu$ becomes
\begin{equation} \label{eq:rt_amps}
    i\hbar\partt{\tau^\mu} = \bra{\mu}\bar H(t)\ket{0}
\end{equation}
where the right-hand-side of ~Eq.(\ref{eq:rt_amps}) is simply the usual CC residual expression in the
presence of a time-dependent Hamiltonian. An expression for the time-evolution of the left-hand wave 
function parameters $\lambda^\mu$ is obtained in the same manner.
We adopt the field-perturbed Hamiltonian used in 
Section \ref{ss:exact}; however, we are now free to define our field form at the time of computation,
since we require no further simplification of these expressions to continue.

Unfortunately, nonlinear differential equations like ~Eq.(\ref{eq:rt_amps}) do not have a closed-form 
solution. Instead, we may utilize \textit{numerical integration} techniques to directly propagate 
the amplitudes in time. One simple (and popular) class of approximate integrators is the Runge-Kutta 
family of methods. 
These methods take a ''step" forward in time by computing approximate intermediate steps
and combining the solutions. The number of steps determines the ''order" of the propagation. For example,

