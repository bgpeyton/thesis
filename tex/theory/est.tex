\section{Ground state electronic structure theory} \label{se:est}
Electron correlation is defined in the present work to mean any additional effect beyond what is
included in the ground state Hartree-Fock wave function. This anti-symmetrized, 
single-determinant wave function, generally composed of linear combinations of atomic Gaussian basis functions, will be denoted by $\ket{0}$, and defines the canonical molecular orbital (MO) basis. 
This section will briefly describe a selection of methods for recovering the correlation energy which are
pertinent to this work.
Correlation is usually included by considering the effects of substituted determinants in which electrons
in their ground state occupied orbitals are ``excited'' into higher-energy virtual orbitals. These 
occupied and virtual orbitals are often chosen to be the canonical MOs of the reference 
wave function; however, as discussed later, some schemes apply a different approach. Therefore, this 
chapter will focus on \textit{general} expressions, without assuming canonical or orthogonal orbital
spaces except where mentioned.   

\subsection{Orbital invariant MP2} \label{ss:mp2}
A second-order correction to the Hartree-Fock energy can be obtained through perturbation theory.
By partitioning the Hamiltonian and expanding the energy and wave function in orders of a perturbation, contributions to the correlation energy are obtained order-by-order. Specifying the Hylleraas partitioning of the normal-ordered electronic Hamiltonian,
\begin{equation} \label{eq:H}
    % I should invert an \underbrace to show which terms are included in which part of H
    \hat{H}_N = E^{(0)} + \hat{F}_N + \hat{W}_N = \hat{H}^{(0)} + \hat{H}^{(1)} 
\end{equation}
where $E^{(0)}$ is the Hartree-Fock energy and $\hat{F}_N$ and $\hat{W}_N$ are the normal-ordered Fock 
operator and fluctuation potential, respectively, the first non-zero contribution is the Hylleraas 
functional form of the second-order energy,
\begin{equation} \label{eq:hyll}
    E^{(2)} = \bra{0}\hat{W}_N\ket{\Psi^{(1)}}
\end{equation}
where $\ket{\Psi^{(1)}}$ is the first-order correction to the reference wave function $\ket{0}$.
The MP2 method is equivalent to an iterative minimization of ~Eq(\ref{eq:hyll}) through a residual expression defined by the first-order wave function equation,
\begin{equation} \label{eq:mp2_res}
    \hat{W}_N\ket{0} + \hat{F}_N\ket{\Psi^1} = 0.
\end{equation}
Since the Hamiltonian does not connect $\ket{0}$ to singly-substituted determinants due to the Brilloun condition, it follows that only double substitutions are present in $\ket{\Psi^{(1)}}$. 
Projection of ~Eq(\ref{eq:mp2_res}) onto a basis of doubly-substituted determinants yields programmable expressions for the double excitation amplitudes, $t^{ab}_{ij}$. In the limit of canonical HF orbitals, ~Eq(\ref{eq:hyll}) reduces to the more commonly known MP2 expressions,
\begin{subequations}
    \begin{equation}
        E^{(2)} = \frac{1}{4}t^{ab}_{ij}g^{ab}_{ij}
    \end{equation}
    \begin{equation}
        t^{ab}_{ij} = - \frac{g^{ab}_{ij}}{\epsilon_a + \epsilon_b - \epsilon_i - \epsilon_j} 
    \end{equation}
\end{subequations}
where $g^{ab}_{ij}$ is the antisymmetryized two-electron integral $\bra{ij}\ket{ab}$ and $\epsilon_p$ is
the $p$th diagonal element of the Fock matrix. 
This requires only one $O(N^5)$ step in the rotation of the two-electron integrals into the MO basis followed by a ``one-shot" energy calculation. 
However, commonplace localizations of the occupied space requires the iterative evaluation of 
~Eq(\ref{eq:mp2_res}). Furthermore, oftentimes reduced nonorthogonal virtual spaces are employed in the 
projection basis (as will be explored in LATER SECTION), which necessitates the inclusion of overlap 
matrices.  

\subsection{Coupled Cluster} \label{ss:cc}
The coupled-cluster family of electronic structure methods employ an exponential cluster operator $\hat{T}$ to include the effects of substituted determinants,
\begin{equation} \label{eq:cc_TDSE}
    \hat{H}e^{\hat{T}}\ket{\Psi} = Ee^{\hat{T}}\ket{\Psi}
\end{equation}
with
\begin{equation} \label{eq:T}
    \hat{T}_n = \left(\frac{1}{n!}\right)^2 t_{ij\ldots}^{ab\ldots}a_a^{\dagger}a_b^{\dagger}\ldots a_j a_i.
\end{equation}
Rather than perturbative expansion, the cluster operator is expanded in orders of substitutions (singles, doubles, \textit{et cetera}), then truncated to a given substitution level to produce tractable equations. Additional simplification is achieved by using the similarity-transformed Hamiltonian
\begin{equation} \label{eq:Hbar}
    \bar{H} = e^{-\hat{T}}\hat{H}e^{\hat{T}}
\end{equation}
which has the same eigenspectrum as $\bar{H}$, but naturally truncates at four nested commutators in a Cambell-Baker-Hausdorff expansion
\begin{equation} \label{eq:cbh}
    \bar{H} = \hat{H} + [\hat{H}, \hat{T}] + \frac{1}{2!}[[\hat{H}, \hat{T}], \hat{T}] + \frac{1}{3!}[[[\hat{H}, \hat{T}], \hat{T}], \hat{T}] + \frac{1}{4!}[[[[\hat{H}, \hat{T}], \hat{T}], \hat{T}], \hat{T}] + . . .
\end{equation}
As before, programmable expressions are obtained for the amplitudes $t_{ij\ldots}^{ab\ldots}$ by projection onto substituted determinants. Here also, the choice of projection basis may require the inclusion of overlap matrices in the iterative procedure. 

\subsection{Reduced Density Matrices} \label{ss:rdm}
A general expression for the ground-state energy of an arbitrary wave function $\ket{\Psi}$ can be written in terms of one- and two-particle reduced density matrices (1-RDM $D$ and 2-RDM $\Gamma$)
\begin{equation} \label{eq:pdm_e}
    \begin{aligned}
    E & = \bra{\Psi}\hat{H}\ket{\Psi} \\
      & = D_{pq}h_{p}^{q} + \Gamma_{pqrs}g_{pq}^{rs}
    \end{aligned}
\end{equation}
where $h_{p}^{q}$ are the one-electron integrals of the electronic Hamiltonian, and $D$ and $\Gamma$ are defined according to the form of $\ket{\Psi}$ and the excitation operators of the second quantized Hamiltonian:
\begin{subequations}
\begin{equation} \label{eq:opdm}
    D_{pq} = \bra{\Psi}a_p^{\dagger}a_q\ket{\Psi}
\end{equation}
\begin{equation} \label{eq:tpdm}
    \Gamma_{pqrs} = \bra{\Psi}a_p^{\dagger}a_q^{\dagger}a_sa_r\ket{\Psi}.
\end{equation}
\end{subequations}
Clearly, this requires the left-hand wave function expression $\bra{\Psi}$. For the electronic Hamiltonian, which is Hermitian, the left- and right-hand wave functions are identical, as is the case for MP2. However, due to the similarity transformation of the Hamiltonian in ~Eq(\ref{eq:Hbar}), this Hermiticity is destroyed. It is therefore necessary to solve for the CC left-hand wave function,
\begin{equation} \label{eq:cc_lwfn}
    \bra{\mathcal{L}} = \bra{\Psi}\hat{\mathcal{L}}
\end{equation}
where $\bra{\mathcal{L}}$ is a left-hand cluster operator analogous to ~Eq(\ref{eq:T}), with analogous amplitudes $t_{ij\ldots}^{ab\ldots}$. This additional set of coupled equations can be solved in the same manner as the right-hand amplitude expressions. We may then rewrite the coupled cluster energy expression using ~Eq(\ref{eq:cc_TDSE}): 
\begin{equation} \label{eq:cc_energy_total}
    \bra{\Psi}\hat{\mathcal{L}}\bar{H}\ket{\Psi} = E.
\end{equation}

As will be explored in [RESPONSE CHAPTER], properties may be expressed as derivatives of the electronic energy. Given the left- and right-hand wave functions and assuming the Hellmann-Feynman theorem holds (and thus $\ket{\Psi}$ carries no dependence on $\Omega$), the derivative of the energy with respect to an arbitrary operator $\Omega$ can be conveniently expressed
\begin{equation} \label{eq:dE}
    \begin{aligned}
    \frac{\partial E}{\partial \Omega} &= \bra{\Psi}\frac{\partial \hat{H}}{\partial \Omega}\ket{\Psi}\\
                                  &= D_{pq}\frac{\partial h_{pq}}{\partial \Omega} + \Gamma_{pqrs}\frac{\partial g_{pqrs}}{\partial \Omega}
    \end{aligned}
\end{equation}
%% NOTE: a lot of notation to clean up here
without further specification of the wave function or Hamiltonian. Thus, time-independent first-order properties may be expressed without differentiation of the RDMs. The cases of higher-order and time-dependent properties will be handled in [RESPONSE CHAPTER]. 
