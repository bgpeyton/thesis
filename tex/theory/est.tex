\section{Ground state electronic structure theory} \label{se:est}
Electron correlation is defined in the present work to mean any additional effect beyond what is
included in the ground state Hartree-Fock (HF) wave function. This anti-symmetrized, 
single-determinant wave function, generally composed of linear combinations of atomic Gaussian basis functions, will be denoted by $\ket{0}$, and defines the canonical molecular orbital (MO) basis. 
This section will briefly describe a selection of methods for recovering the correlation energy which are
pertinent to this work.
Correlation is usually included by considering the effects of substituted determinants in which electrons
in ground state occupied orbitals are ``excited'' into higher-energy unoccupied (or ``virtual'') orbitals. 
These occupied and virtual orbitals are often chosen to be the canonical MOs of the reference 
wave function, which in turn are linear combinations of atomic orbital basis functions.
However, as discussed in [PAPER 3], some schemes apply a different approach. Therefore, this 
chapter will focus on \textit{general} expressions, without assuming canonical or orthogonal orbital
spaces except where mentioned.   
Einstein summation notation will be used throughout this chapter, except in cases where an explicit sum
is instructive.

%\subsection{Orbital invariant MP2} \label{ss:mp2}
\subsection{Many-Body Perturbation Theory} \label{ss:mp2}
Correlation energy corrections to the HF energy may be obtained through many-body perturbation theory (MBPT).
%By partitioning the Hamiltonian and expanding the energy and wave function in orders of a perturbation, contributions to the correlation energy are obtained order-by-order. Specifying the Hylleraas partitioning of the normal-ordered electronic Hamiltonian,
By partitioning the electronic Hamiltonian $\hat{H}$ into a zeroth-order (HF) term $\hat{H}^{(0)}$ and a perturbation $\hat{H}^{(1)}$,
\begin{equation} \label{eq:H}
%    \hat{H}_N = E^{(0)} + \hat{F}_N + \hat{W}_N = \hat{H}^{(0)} + \hat{H}^{(1)} 
    \hat{H} = \hat{H}^{(0)} + \hat{H}^{(1)} 
\end{equation}
and expanding the electronic energy $E$ and wave function $\ket{\Psi}$ in orders of the perturbation (denoted by superscript), 
\begin{subequations} \label{eq:pert}
    \begin{equation} \label{eq:pert_psi}
        \ket{\Psi} = \sum_{i=0}^{\infty}\lambda^{(i)}\ket{\Psi^{(i)}}
    \end{equation}
    \begin{equation} \label{eq:pert_e}
        E = \sum_{i=0}^{\infty}\lambda^{(i)}E^{(i)},
    \end{equation}
\end{subequations}
contributions to the correlation energy are obtained order-by-order. $\ket{\Psi^{(0)}}$ is taken to be $\ket{0}$, 
and $\lambda^{(0)}$ to be one.
Implicit here is the assumption that the perturbation is ``small'' relative to the reference -- \textit{i.e.}, 
the correlation energy is several orders of magnitude smaller than the HF energy.
%Specifying the Hylleraas partitioning of the normal-ordered electronic Hamiltonian,
%where $E^{(0)}$ is the Hartree-Fock energy and $\hat{F}_N$ and $\hat{W}_N$ are the normal-ordered Fock 
%operator and fluctuation potential, respectively, the first non-zero contribution is the Hylleraas 
%functional form of the second-order energy,

For a perturbation containing correlation operators, 
inserting ~Eqs(\ref{eq:H}) and (\ref{eq:pert}) into the time-independent Schr\"odinger equation and truncating at the 
second order in $\lambda$ yields the first non-zero contribution to the energy:
\begin{equation} \label{eq:hyll}
    E^{(2)} = \bra{0}\hat{H}^{(0)}\ket{\Psi^{(1)}}.
\end{equation}
Second-order MBPT is equivalent to the minimization of ~Eq(\ref{eq:hyll}). 
Using the Hylleraas partitioning of the Hamiltonian,
\begin{subequations}
    \begin{equation}
        \hat{H}^{(0)} = E^{(0)} + \hat{F}
    \end{equation}
    \begin{equation}
        \hat{H}^{(1)} = \hat{W}
    \end{equation}
\end{subequations}
where $\hat{F}$ and $\hat{W}$ are the one- and two-particle components of $\hat{H}^{(0)}$, respectively, 
this is achieved through minimization of a residual expression defined by the first-order wave 
function equation,
\begin{equation} \label{eq:mp2_res}
    \hat{W}\ket{0} + \hat{F}\ket{\Psi^{(1)}} = 0.
\end{equation}
Since the Hamiltonian does not connect $\ket{0}$ to singly-substituted determinants due to the Brillouin condition, it follows that only double substitutions are present in $\ket{\Psi^{(1)}}$. 
Projection of ~Eq(\ref{eq:mp2_res}) onto a basis of doubly-substituted determinants yields programmable expressions for the coefficients of the first-order wave function $\lambda^{(1)}$, 
commonly referred to as double-excitation amplitudes $t^{ab}_{ij}$. 

In the limit of canonical HF orbitals, ~Eq(\ref{eq:hyll}) reduces to the more commonly known second-order M{\o}ller-Plesset perturbation theory (MP2) expressions,
\begin{subequations}
    \begin{equation} \label{eq:E2}
        %E^{(2)} = \frac{1}{4}t^{ab}_{ij}g^{ab}_{ij}
        E^{(2)} = \frac{1}{4}t^{ab}_{ij}\bra{ij}\ket{ab}
    \end{equation}
    \begin{equation} \label{eq:T2_mp2}
        %t^{ab}_{ij} = - \frac{g^{ab}_{ij}}{\epsilon_a + \epsilon_b - \epsilon_i - \epsilon_j} 
        t^{ab}_{ij} = - \frac{\bra{ij}\ket{ab}}{\epsilon_a + \epsilon_b - \epsilon_i - \epsilon_j} 
    \end{equation}
\end{subequations}
%where $g^{ab}_{ij}$ is the antisymmetryized two-electron integral $\bra{ij}\ket{ab}$ 
where $\bra{ij}\ket{ab}$ is the antisymmetrized two-electron integral between occupied orbitals $\{i,j\}$ and 
virtual orbitals $\{a,b\}$,
and $\epsilon_p$ is the $p$th diagonal element of the Fock matrix. 
This requires only one $\mathcal{O}(N^5)$ step (where $N$ is a measure of system size) in the rotation of the two-electron atomic orbital integrals into the MO basis followed by a ``one-shot" energy calculation. 
However, commonplace localizations of the occupied space require instead the iterative evaluation of 
~Eq(\ref{eq:mp2_res}). Furthermore, oftentimes reduced nonorthogonal virtual spaces are employed in the 
projection basis (as will be explored in [PAPER 3]), which necessitates the inclusion of additional overlap 
matrices between nonorthogonal orbital spaces in these equations. While this increases the cost by adding 
iterations and additional expressions to evaluate, these additional costs are heavily outweighed by the 
computational savings due to the reduction of $N$ in large systems.

\subsection{Coupled Cluster} \label{ss:cc}
The coupled-cluster family of electronic structure methods employ an exponentiated cluster operator $\hat{T}$ to include the effects of substituted determinants,
\begin{equation} \label{eq:cc_TISE}
    \hat{H}e^{\hat{T}}\ket{0} = Ee^{\hat{T}}\ket{0}
\end{equation}
with
\begin{subequations}
    \begin{equation} \label{eq:T}
        \hat{T} = \sum_{n=1}^M\hat{T}_n
    \end{equation}
    \begin{equation} \label{eq:T_n}
        \hat{T}_n = \left(\frac{1}{n!}\right)^2 t_{ij\ldots}^{ab\ldots}a_a^{\dagger}a_b^{\dagger}\ldots a_j a_i
    \end{equation}
\end{subequations}
where $M$ is the number of electrons in the system, and $a_p$ are the standard second-quantized 
excitation operators.
Rather than perturbative expansion, the cluster operator is expanded in orders of substitutions (singles, doubles, \textit{et cetera} in ~Eq.(\ref{eq:T})), then truncated to a given substitution level to produce tractable equations. Additional simplification is achieved by using the similarity-transformed Hamiltonian
\begin{equation} \label{eq:Hbar}
    \bar{H} = e^{-\hat{T}}\hat{H}e^{\hat{T}}
\end{equation}
which has the same eigenspectrum as $\hat{H}$, but naturally truncates at four nested commutators in a Campbell-Baker-Hausdorff expansion
\begin{equation} \label{eq:cbh}
    \bar{H} = \hat{H} + [\hat{H}, \hat{T}] + \frac{1}{2!}[[\hat{H}, \hat{T}], \hat{T}] + \frac{1}{3!}[[[\hat{H}, \hat{T}], \hat{T}], \hat{T}] + \frac{1}{4!}[[[[\hat{H}, \hat{T}], \hat{T}], \hat{T}], \hat{T}].
\end{equation}
Projecting ~Eq.(\ref{eq:cc_TISE}) on the left by $\bra{0}e^{-\hat{T}}$ yields the coupled cluster energy expression. As in Section \ref{ss:mp2}, programmable expressions for the amplitudes $t_{ij\ldots}^{ab\ldots}$ are obtained by projection onto substituted determinants $\ket{\mu}$:
\begin{subequations}
    \begin{equation} \label{eq:cc_E}
        \bra{0}\bar{H}\ket{0} = E_{CC}
    \end{equation}
    \begin{equation} \label{eq:cc_res}
        \bra{\mu}\bar{H}\ket{0} = 0.
    \end{equation}
\end{subequations}
Here also, the choice of projection basis may require the inclusion of overlap matrices in the iterative 
evaluation of the amplitude expressions. 

Untruncated, ~Eqs.(\ref{eq:cc_E}) and (\ref{eq:cc_res}) produce exact solutions to the time-independent 
Schr\"odinger equation for the electronic Hamiltonian in a given basis set. Perhaps the most common 
truncation of ~Eq.(\ref{eq:T}) is to single and double substitutions, defining the CCSD method. 
The computational cost of this method scales as $\mathcal{O}(N^6)$. In terms of MBPT, CCSD may be viewed 
as correct through infinite order, but only within the space of single and double substitutions.
A perturbative correction for triples produces the ``gold standard'' in quantum chemistry, 
the CCSD(T) method, which scales as $\mathcal{O}(N^7)$. Aside from these, additional truncation schemes exist 
for a variety of purposes. For example, the CCD method includes only double substitutions, but the 
complexity of the resulting equations is far simpler. Another method, CC2, is tuned for the calculation
of molecular response properties, such as dipoles and polarizabilities. This method retains the singles
expressions from CCSD, but truncates the doubles expression at first order by considering the $\hat{T}_2$ 
operator as first order (the one- and two-electron terms of the Hamiltonian are still considered to be
zeroth and first order, as in Section \ref{ss:mp2}). Solving the resulting iterative amplitude expressions 
for CC2
scale as $\mathcal{O}(N^5)$ but, unlike the similarly-scaling MP2 method, includes the effects of singles, 
which have been shown to be crucial for computing accurate molecular properties. 
It is important to note that the while the studies in [PAPER 2] and [PAPER 3] utilize CCSD, 
many other flavors of CC may be used as drop-in replacements, both in theory and in the implementation 
of their codes. 

\subsection{Reduced Density Matrices} \label{ss:rdm}
A general expression for the ground-state energy of an arbitrary wave function $\ket{\Psi}$ can be written in terms of one- and two-particle reduced density matrices (1-RDM $D$ and 2-RDM $\Gamma$)
\begin{equation} \label{eq:pdm_e}
    \begin{aligned}
    E & = \bra{\Psi}\hat{H}\ket{\Psi} \\
      & = D_{pq}h_{p}^{q} + \Gamma_{pqrs}\bra{pq}\ket{rs}
    \end{aligned}
\end{equation}
where $h_{p}^{q}$ are the one-electron integrals of the electronic Hamiltonian, and $D$ and $\Gamma$ are defined according to the form of $\ket{\Psi}$ and the excitation operators of the second quantized Hamiltonian:
\begin{subequations}
\begin{equation} \label{eq:opdm}
    D_{pq} = \bra{\Psi}a_p^{\dagger}a_q\ket{\Psi}
\end{equation}
\begin{equation} \label{eq:tpdm}
    \Gamma_{pqrs} = \bra{\Psi}a_p^{\dagger}a_q^{\dagger}a_sa_r\ket{\Psi}.
\end{equation}
\end{subequations}
Clearly, this requires the left-hand wave function $\bra{\Psi}$. For the electronic Hamiltonian, which is Hermitian, the left- and right-hand wave functions are identical, as is the case for MP2. However, this Hermiticity is destroyed for standard CC methods due to the similarity transformation of the Hamiltonian 
in ~Eq(\ref{eq:Hbar}). It is therefore necessary to solve for the CC left-hand wave function,
\begin{equation} \label{eq:cc_lwfn}
    \bra{\mathcal{L}} = \bra{\Psi}\hat{\mathcal{L}}
\end{equation}
where $\bra{\mathcal{L}}$ is a left-hand cluster operator analogous to ~Eq(\ref{eq:T_n}), with analogous amplitudes $\lambda_{ij\ldots}^{ab\ldots}$. This additional set of coupled equations can be solved in the same manner as the right-hand amplitude expressions. We may then rewrite the coupled cluster energy expression using ~Eq(\ref{eq:cc_TISE}): 
\begin{equation} \label{eq:cc_energy_total}
    \bra{\Psi}\hat{\mathcal{L}}\bar{H}\ket{\Psi} = E.
\end{equation}

As will be explored in Section \ref{se:res}, molecular properties may be expressed as derivatives of the electronic energy. Given the left- and right-hand wave functions and assuming the Hellmann-Feynman theorem holds (and thus $\ket{\Psi}$ carries no dependence on $\hat{\Omega}$), the derivative of the energy with respect to an arbitrary operator $\hat\Omega$ can be conveniently expressed
\begin{equation} \label{eq:dE}
    \begin{aligned}
    \frac{\partial E}{\partial \hat\Omega} &= \bra{\Psi}\frac{\partial \hat{H}}{\partial \hat\Omega}\ket{\Psi}\\
                                  &= D_{pq}\frac{\partial h_{pq}}{\partial \hat\Omega} + \Gamma_{pqrs}\frac{\partial g_{pqrs}}{\partial \hat\Omega}
    \end{aligned}
\end{equation}
%% NOTE: a lot of notation to clean up here
without further specification of the wave function or Hamiltonian. Thus, time-independent first-order properties may be expressed without differentiation of the RDMs. This was a driving force behind the study in 
[PAPER 1], and is important for the machine-learning methods discussed in [PAPER 2]. The cases of 
higher-order and time-dependent properties, which are the
focus of [PAPER 1] and [PAPER 3], will be handled in the following section. 
