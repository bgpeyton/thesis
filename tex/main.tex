\documentclass[doublespace,nopageskip]{VTthesis} % nopageskip - Removes arbitrary blank pages.
%\documentclass[doublespace,draft,nopageskip]{VTthesis} % nopageskip - Removes arbitrary blank pages.
% Using the following header instead will create a draft copy of your thesis
%\documentclass[doublespace,draft]{VTthesis}

%%%%%%%%%%%%%%%%%%%%%%%%%%%%%%%%%%%%%%%%%%%%%%%%%%%%%%%%%%%%%%%%%%%%%
% Customized Definitions 
%%%%%%%%%%%%%%%%%%%%%%%%%%%%%%%%%%%%%%%%%%%%%%%%%%%%%%%%%%%%%%%%%%%%%
% Math 
\def\ket#1{| #1 \rangle}        %| >
\def\bra#1{\langle#1 |}         %< |
%\def\bm#1{\mbox{\boldmath$#1$}}
\def\bm#1{\mathbf{#1}}
\def\linresp#1#2{\langle\langle#1;#2\rangle\rangle}
\def\quadresp#1#2#3{\langle\langle#1;#2,#3\rangle\rangle}
\def\degrees{deg dm$^{-1}$ (g/mL)$^{-1}$}
\def\optrot{$[\alpha]$}
\def\crt#1{a_{#1}^{\dagger}}    % Creation operator
\def\ann#1{a_{#1}^{\ }}         % annihilation operator 
\def\cgs{($10^{-40}$ cgs)}      % units 
\def\selfolap#1{\langle#1|#1\rangle} % <i|i>
\def\olap#1#2{\langle#1|#2\rangle} %<i|j>
\def\partt#1{\frac{\partial#1}{\partial t}} % partial i / partial t
\def\piw{\bar\psi(t)} % phase-isolated wfn
%%%%%%%%%%%%%%%%%%%%%%%%%%%%%%%%%%%%%%%%%%%%%%%%%%%%%%%%%%%%%%%%%%%%%

%% ADDED PER RUHEE'S ADVICE
\usepackage[sort&compress,numbers,super]{natbib}

%% ADDED FOR PAPER 1
%\usepackage{caption}
%\usepackage{subcaption}
%\usepackage{graphicx}
%\usepackage{wrapfig}


\title{The Efficient Computation of Field-Dependent Molecular Properties in the Frequency and Time Domains}
\keywords{Electronic structure theory, machine learning, coupled cluster, local correlation}
\author{Benjamin G. Peyton}
\program{Chemistry} 
\degree{Doctor of Philosophy} 
\submitdate{April 13, 2022} 

\principaladvisor{T. Daniel Crawford}
\firstreader{Nicholas Mayhall}
\secondreader{Diego Troya}
\thirdreader{John Morris}

%\dedication{This is where you put your dedications.}
%\acknowledge{This is where you put your acknowledgments.}
%\abstract{Give a brief description of your thesis here.}
%\abstractgenaud{You are also required as of Spring 2016 to include a general audience abstract. This should be geared towards individuals outside of your field that may be reading seeking information about your work. You should avoid language that is particular to your field and clearly define any terms that may have special meaning in your discipline.}

\begin{document}
% The following lines set up the front matter of your thesis or dissertation and are required to ensure proper formatting per the VT ETD standards. 
  \frontmatter
  \maketitle
  \tableofcontents

% The list of figures and tables are now optional per the official ETD standards.  Unless you have a very good reason for removing them, you should leave these lists in the document. Comment them out to remove them.
	\listoffigures
	\listoftables

% The following sets up the document for the main part of the thesis or dissertation. Do not comment out or remove this line.
	\mainmatter
    \chapter{Introduction} \label{ch:int}

Modern synthetic organic chemistry employs a vast array of sophisticated instrumentation.
Principle among these are probes of light-matter interaction, which reveal rich structural and 
electro-magnetic characteristics. These instruments work by measuring the absorption, 
reflection, or refraction of an electromagnetic field (EMF) interacting with a target system. 
Experimental techniques which measure these interactions with respect to the energy or frequency
of the incident EMF are known as spectroscopy. Many staple experimental apparatus probe these
relationships, including (but certainly not limited to)
ultraviolet-visible absorption (UV-Vis), 
optical rotatory dispersion (ORD), 
flame atomic absorption (flame AA),
and electronic circular dichroism (ECD).
These data can be used for characterization or establishing structure-property relationships 
which aid in the development of novel systems for applications in materials, bio-organics, 
and more. 
Interpreting or predicting the results of such experiments requires a knowledge of fundamental 
light-matter interactions on a quantum level.  

A quantum description of any molecular interaction may be viewed as the effects of a perturbation
on a quantum-mechanical system. In the case of EMF, this perturbation may be ``static'' (fixed)
or ``dynamic'' (varying in frequency or time). A fully quantum-mechanical description of the 
light-matter system would require quantum electrodynamics (QED); however, it is often sufficient
to treat the EMF from a classical perspective, treating only the molecular response using
quantum mechanics. This allows us to utilize many well-established methods in the field of 
theoretical chemistry. These generally provide a representation of the system (a wave function or 
density) with which to take the expectation value of a given operator or, in some cases, predict 
the expectation value directly. Methods which utilize only mathematical techniques (that is, no
experimental or phenomenological parameters) are said to be \textit{ab initio} methods. These 
methods often (but not always) simplify or ignore the quantum effects of the nuclei and their 
motion, an approximation known as the Born-Oppenheimer approximation. 
Another subclass of \textit{ab initio} methods, which 
center around solving the time-dependent or time-independent Schr\"odinger wave equation to 
obtain an explicit form of the wave function, are 
known as wave function-based methods. Finally, methods that go beyond the mean-field or Hartree-Fock 
approximation, in which electrons only interact through an average self-consistent field, are 
classified as ``correlated'' methods. 
It is these correlated wave function-based methods under the Born-Oppenheimer
approximation upon which the bulk of the current work is built. As such, following this in Chapter 
\ref{ch:theory} is a primer in the theoretical underpinnings necessary to understand this work which are not described 
in detail in the publications that follow in sections [PAPER 1-3]. Chapter \ref{ch:theory} is roughly 
divided into details of electronic structure theory (Section \ref{se:est}) and molecular response 
properties (Section \ref{se:res}). 

First and foremost is many-body perturbation theory (MBPT). This framework allows us to 
separate the quantum mechanical properties of the isolated system versus some perturbing 
force, which is expanded in orders and truncated under the assumption that higher-order terms 
become negligible.
This force is often taken to be electron correlation when describing the electronic ground 
state wave function; however, it may also be a static or dynamic EMF. In section (\ref{ss:mp2})
we present some basic characteristics of general MBPT, in preparation for its application in the 
chapters that follow.

Secondly, in section (\ref{ss:cc}) we present the wave function-based method upon which the bulk 
of this work is based or seeks to approximate: coupled cluster (CC) theory. 
This method, like perturbation theory, is most commonly used for computing electron correlation 
(though its roots are in nuclear physics). 
Unlike perturbation theory, we do not perform an order-by-order expansion of the perturbation; 
instead, a ``cluster'' operator folds in contributions based on a physical intuition, that is the 
instantaneous occupation of many quantum states which give rise to electron correlation. Truncation
is then based on the number of quantum states (substituted or `excited` determinants). This method,
while accurate and systematically improvable, is notoriously expensive, suffering from high-order
polynomial scaling. Extensions to excited states and molecular response properties only compound this 
issue; as such, [PAPER 1] and [PAPER 2] are focused on ways to circumvent using CC on any but a small 
subset of systems. (It should be noted that, while [Paper 1] utilizes a different correlated method 
based on a density-functional theory approach, the primary goal was to ascertain the effects of 
fragmentation of a system through the many-body expansion on the computation of molecular properties. 
This is a benchmark study, whose goal is understood to be conclusions that would be applied to more 
expensive methods such as CC, where such benchmark studies could not be performed.)
Additional considerations of reduced density matrices, which are used throughout the work to refer
to correlated wave functions, follows in Section \ref{ss:rdm}.

Finally, we describe the method by which we couple the classical perturbing field to the 
quantum-mechanical correlated wave function. 
By applying perturbation theory we derive general expressions for tensors which describe 
the molecular response to an EMF. In the exact theory, these tensors turn out to be functions of the 
excited electronic states of the system; however, an approach to generalize 
these expressions to approximate theories such as truncated coupled cluster is discussed, which also 
avoids the costly evaluation of excited-state wave functions. For dynamic EMF, these properties may be 
described as a function of frequency, so the most common approach is to derive the working equations
using a Fourier transform of expressions obtained using the time-dependent Schr\"odinger equation 
(more generally known as response theory); however, these expressions may also be evaluated explicitly
in the time-domain. In Section \ref{se:res} we focus on the frequency-domain formulation, while
[PAPER 3] explores the recently-revived prospect of explicit time-propagation, as well as a 
technique to make this process cheaper through a well-established concept known as 
``local correlation''. 

    \chapter{Theoretical background} \label{ch:theory}
\section{Ground state electronic structure theory} \label{se:est}
Electron correlation is defined in the present work to mean any additional
effect beyond what is included in the ground-state Hartree-Fock (HF)
wave function $\ket{0}$. In ground-state electronic structure theory,
this is typically taken to be an anti-symmetrized, single-determinant wave
function, generally composed of linear combinations of atomic Gaussian
basis functions. In solving the HF equations, the electronic energy
is variationally minimized, under the constraint that the orbital basis
functions of $\ket{0}$ remain orthonormal.\cite{Szabo1996} The resulting
orbitals will be referred to as the canonical molecular orbital (MO) basis.

This section will briefly describe a selection of methods for recovering
the correlation energy which are pertinent to this work. Correlation is
usually included by considering the effects of substituted determinants in
which electrons in ground-state occupied orbitals are substituted (often
referred to as ``excited'') into higher-energy unoccupied (or virtual)
orbitals. These occupied and virtual orbitals are often chosen to be the
canonical MOs of the reference wave function, which in turn are linear
combinations of atomic orbital basis functions. However, as discussed in
[PAPER 3], some schemes apply a different approach -- namely, localization
(and potentially truncation) of the canonical MOs prior to computing the
correlation energy. Therefore, this chapter will focus on \textit{general}
expressions, without assuming canonical or orthogonal orbital spaces
except where mentioned. Einstein summation notation (implied summation
over repeated indices) will be used throughout this chapter, except in
cases where an explicit sum is instructive.

%\subsection{Orbital invariant MP2} \label{ss:mp2}
\subsection{Many-Body Perturbation Theory} \label{ss:mp2}
Correlation energy corrections to the HF energy may be obtained through many-body perturbation theory (MBPT).
%By partitioning the Hamiltonian and expanding the energy and wave function in orders of a perturbation, contributions to the correlation energy are obtained order-by-order. Specifying the Hylleraas partitioning of the normal-ordered electronic Hamiltonian,
We may begin with either the time-dependent or time-independent Schr\"odinger equation --
notation for time-dependence will be suppressed here, 
but the same rules apply as mentioned in Section $\ref{se:res}$.
By partitioning the electronic Hamiltonian $\hat{H}$ into a zeroth-order (HF) term $\hat{H}^{(0)}$ and a perturbation $\hat{H}^{(1)}$,
which is first-order in an undetermined coefficient $\lambda$,
\begin{equation} \label{eq:H}
%    \hat{H}_N = E^{(0)} + \hat{F}_N + \hat{W}_N = \hat{H}^{(0)} + \hat{H}^{(1)} 
    \hat{H} = \hat{H}^{(0)} + \lambda\hat{H}^{(1)} 
\end{equation}
and expanding the electronic energy $E$ and wave function $\ket{\Psi}$ similarly in orders of the perturbation (denoted by superscript), 
\begin{subequations} \label{eq:pert}
    \begin{equation} \label{eq:pert_psi}
        \ket{\Psi} = \sum_{i=0}^{\infty}\lambda^{i}\ket{\Psi^{(i)}}
    \end{equation}
    \begin{equation} \label{eq:pert_e}
        E = \sum_{i=0}^{\infty}\lambda^{i}E^{(i)},
    \end{equation}
\end{subequations}
contributions to the correlation energy are obtained order-by-order. $\ket{\Psi^{(0)}}$ is taken to be $\ket{0}$, 
and $\lambda$ to be one.
Implicit here is the assumption that the perturbation is small relative to the reference -- \textit{i.e.}, 
the correlation energy is several orders of magnitude smaller than the HF energy.
%Specifying the Hylleraas partitioning of the normal-ordered electronic Hamiltonian,
%where $E^{(0)}$ is the Hartree-Fock energy and $\hat{F}_N$ and $\hat{W}_N$ are the normal-ordered Fock 
%operator and fluctuation potential, respectively, the first non-zero contribution is the Hylleraas 
%functional form of the second-order energy,

For a perturbation containing correlation operators, inserting
Eqs.~(\ref{eq:H}) and (\ref{eq:pert}) into the time-independent Schr\"odinger
equation and truncating at the second order in $\lambda$ yields the first
non-zero contribution to the energy:\cite{Szabo1996} \begin{equation}
\label{eq:hyll}
    E^{(2)} = \bra{0}\hat{H}^{(1)}\ket{\Psi^{(1)}}.
\end{equation} Second-order MBPT is equivalent to the minimization
of Eq.~(\ref{eq:hyll}). Using the Hylleraas partitioning of the
Hamiltonian,\cite{Pulay1986b} \begin{subequations}
    \begin{equation}
        \hat{H}^{(0)} = E^{(0)} + \hat{F}
    \end{equation} \begin{equation}
        \hat{H}^{(1)} = \hat{W}
    \end{equation}
\end{subequations} where $\hat{F}$ and $\hat{W}$ are the one- and
two-particle components of $\hat{H}^{(0)}$, respectively, we may solve Eq.~(\ref{eq:hyll}).
This is achieved
through minimization of a residual expression defined by the first-order
wave function equation, 
\begin{equation} \label{eq:mp2_res}
    \hat{W}\ket{0} + \hat{F}\ket{\Psi^{(1)}} = 0.
\end{equation} 

To generate programmable expressions, 
we may choose to represent our first-order perturbed wave function as a linear combination 
of substituted HF determinants with coefficients $t^{ab}_{ij}$ 
\begin{equation}
    \ket{\Psi^{(1)}} = \sum_{ijab}\frac{1}{4}t^{ab}_{ij}\ket{0}.
\end{equation}
Since the Hamiltonian does not connect
$\ket{0}$ to singly-substituted determinants due to the Brillouin
condition,\cite{Szabo1996} it follows that only double substitutions are
present in $\ket{\Psi^{(1)}}$. Projection of Eq.~(\ref{eq:mp2_res}) onto a
basis of doubly-substituted determinants yields programmable expressions
for $t^{ab}_{ij}$.
In the limit of canonical HF orbitals, Eq.~(\ref{eq:hyll}) reduces to the more commonly known second-order M{\o}ller-Plesset perturbation theory (MP2)\cite{Moller1934,Bartlett1974a} expressions,
\begin{subequations}
    \begin{equation} \label{eq:E2}
        %E^{(2)} = \frac{1}{4}t^{ab}_{ij}g^{ab}_{ij}
        E^{(2)} = \frac{1}{4}t^{ab}_{ij}\bra{ij}\ket{ab}
    \end{equation}
    \begin{equation} \label{eq:T2_mp2}
        %t^{ab}_{ij} = - \frac{g^{ab}_{ij}}{\epsilon_a + \epsilon_b - \epsilon_i - \epsilon_j} 
        t^{ab}_{ij} = - \frac{\bra{ij}\ket{ab}}{\epsilon_a + \epsilon_b - \epsilon_i - \epsilon_j} 
    \end{equation}
\end{subequations}
%where $g^{ab}_{ij}$ is the antisymmetryized two-electron integral $\bra{ij}\ket{ab}$ 
where $\bra{ij}\ket{ab}$ is the antisymmetrized two-electron integral between occupied orbitals $\{i,j\}$ and 
virtual orbitals $\{a,b\}$,
and $\epsilon_p$ is the $p$th diagonal element of the Fock matrix. 
This requires only one $\mathcal{O}(N^5)$ step (where $N$ is a measure of system size) in the rotation of the two-electron atomic orbital integrals into the MO basis followed by a one-step energy calculation. 
However, commonplace localizations of the occupied space
\cite{Pulay1986a,Surjan1989,Boughton1993}
require instead the iterative evaluation of 
Eq.~(\ref{eq:mp2_res}). Furthermore, reduced nonorthogonal virtual spaces may be employed in the 
projection basis (as will be explored in [PAPER 3]),\cite{Werner2006} which necessitates the inclusion of additional overlap 
matrices between nonorthogonal orbital spaces in these equations. While this increases the cost by adding 
iterations and additional expressions to evaluate, these additional costs are heavily outweighed by the 
computational savings due to the reduction of $N$ in large systems.

\subsection{Coupled Cluster} \label{ss:cc}
The coupled-cluster family of electronic structure methods
\cite{Sinanoglu1964,Cizek1966,Cizek1969,Crawford2000}
employs an exponentiated cluster operator $\hat{T}$ to include the effects of substituted determinants,
\begin{equation} \label{eq:cc_TISE}
    \hat{H}e^{\hat{T}}\ket{0} = Ee^{\hat{T}}\ket{0}
\end{equation}
with
\begin{subequations}
    \begin{equation} \label{eq:T}
        \hat{T} = \sum_{n=1}^M\hat{T}_n
    \end{equation}
    \begin{equation} \label{eq:T_n}
        \hat{T}_n = \left(\frac{1}{n!}\right)^2 t_{ij\ldots}^{ab\ldots}a_a^{\dagger}a_b^{\dagger}\ldots a_j a_i
    \end{equation}
\end{subequations}
where $M$ is the number of electrons in the system, and $a_p$ and $a_p^\dagger$ are the standard second-quantized 
excitation and de-excitation operators, respectively, for a general orbital $p$.
Rather than perturbative expansion, the cluster operator is expanded in orders of substitutions 
(singles, doubles, \textit{etc}. in Eq.~(\ref{eq:T})), then truncated to a given substitution 
level to produce tractable equations. Additional simplification is achieved by using the 
similarity-transformed Hamiltonian
\begin{equation} \label{eq:Hbar}
    \bar{H} = e^{-\hat{T}}\hat{H}e^{\hat{T}}
\end{equation}
which has the same eigenspectrum as $\hat{H}$, but naturally truncates at four nested commutators in a Campbell-Baker-Hausdorff expansion
\begin{equation} \label{eq:cbh}
    \bar{H} = \hat{H} + [\hat{H}, \hat{T}] + \frac{1}{2!}[[\hat{H}, \hat{T}], \hat{T}] + \frac{1}{3!}[[[\hat{H}, \hat{T}], \hat{T}], \hat{T}] + \frac{1}{4!}[[[[\hat{H}, \hat{T}], \hat{T}], \hat{T}], \hat{T}].
\end{equation}
Projecting Eq.~(\ref{eq:cc_TISE}) on the left by $\bra{0}e^{-\hat{T}}$ yields the coupled cluster 
energy expression and, as in Section \ref{ss:mp2}, expressions for the amplitudes 
$t_{ij\ldots}^{ab\ldots}$ are obtained by projection onto substituted determinants $\bra{\mu}$:
\begin{subequations}
    \begin{equation} \label{eq:cc_E}
        \bra{0}\bar{H}\ket{0} = E_{CC}
    \end{equation}
    \begin{equation} \label{eq:cc_res}
        \bra{\mu}\bar{H}\ket{0} = 0.
    \end{equation}
\end{subequations}
Here Eq.~(\ref{eq:cc_E}) follows from the normalization condition of $\ket{0}$, 
and Eq.~(\ref{eq:cc_res}) follows from the orthogonality condition. 
In the case of localized orbitals, such as those used in [PAPER 3], additional terms 
involving the overlap of nonorthogonal orbitals will appear in the final expressions. 

Untruncated, Eqs.~(\ref{eq:cc_E}) and (\ref{eq:cc_res}) produce exact
solutions to the time-independent Schr\"odinger equation for the
electronic Hamiltonian in a given basis set. Perhaps the most common
truncation of Eq.~(\ref{eq:T}) is to single and double substitutions,
defining the CCSD method. The computational cost of this method scales as
$\mathcal{O}(N^6)$. In terms of MBPT, CCSD may be viewed as correct through
infinite order, but only within the space of single and double substitutions.
A perturbative correction for triples produces the ``gold standard'' in
quantum chemistry, the CCSD(T) method, \cite{Raghavachari1989,Bartlett2005}
which scales as $\mathcal{O}(N^7)$. Aside from these, additional truncation
schemes exist for a variety of purposes. For example, the CCD method
includes only double substitutions, but the complexity of the resulting
equations is far simpler. Another method, CC2,\cite{Christiansen1995}
is tuned for the calculation of molecular response properties, such as
dipoles and polarizabilities. This method truncates the doubles expression
at first order by considering the $\hat{T}_2$ operator as first order. The
full singles expressions from CCSD are retained by considering $\hat{T}_1$
to be first order, and the one- and two-electron terms of the Hamiltonian are
still considered to be zeroth and first order, respectively, as in Section
\ref{ss:mp2}. Solving the resulting iterative amplitude expressions for CC2
scales as $\mathcal{O}(N^5)$ but, unlike the similarly-scaling MP2 method,
includes the effects of singles, which have been shown to be crucial for
computing accurate molecular properties.
\cite{Christiansen1995,Koch1997}
It is important to note that while the studies in [PAPER 2] and [PAPER 3]
utilize CCSD, many other flavors of CC may be used as drop-in replacements,
both in theory and in the implementation of their codes.

\subsection{Reduced Density Matrices} \label{ss:rdm} A general expression
for the ground-state energy of an arbitrary wave function $\ket{\Psi}$
prepared in a basis $\{p,q,r,s\}$ can be written in terms of one- and
two-particle reduced density matrices (1-RDM $D$ and 2-RDM $\Gamma$):
\cite{Harris1992,Trucks1988}
\begin{equation} \label{eq:pdm_e}
    \begin{aligned} E & = \bra{\Psi}\hat{H}\ket{\Psi} \\
      & = D_{pq}h_{p}^{q} + \Gamma_{pqrs}\bra{pq}\ket{rs}
    \end{aligned}
\end{equation} where $h_{p}^{q}$ are the one-electron integrals of the
electronic Hamiltonian, and $D$ and $\Gamma$ are defined according to the
form of $\ket{\Psi}$ and the excitation operators of the second quantized
Hamiltonian: \begin{subequations} \begin{equation} \label{eq:opdm}
    D_{pq} = \bra{\Psi}a_p^{\dagger}a_q\ket{\Psi}
\end{equation} \begin{equation} \label{eq:tpdm}
    \Gamma_{pqrs} = \bra{\Psi}a_p^{\dagger}a_q^{\dagger}a_sa_r\ket{\Psi}.
\end{equation} \end{subequations} 
Clearly, this requires the
left-hand wave function $\bra{\Psi}$. For the electronic Hamiltonian,
which is Hermitian, the left- and right-hand wave functions are identical,
as is the case for MP2. However, this Hermiticity is destroyed for standard
CC methods due to the similarity transformation of the Hamiltonian in
Eq.~(\ref{eq:Hbar}).\cite{Crawford2000} It is therefore necessary to solve for the CC left-hand
wave function, \begin{equation} \label{eq:cc_lwfn}
    \bra{\mathcal{L}} = \bra{\Psi}\hat{\mathcal{L}}
\end{equation} where $\hat{L}$ is a left-hand cluster
operator analogous to Eq.~(\ref{eq:T_n}), with analogous amplitudes
$\lambda_{ij\ldots}^{ab\ldots}$. This additional set of coupled
equations can be solved in the same manner as the right-hand amplitude
expressions. We may then rewrite the coupled cluster energy expression
using Eq.~(\ref{eq:cc_TISE}): \begin{equation} \label{eq:cc_energy_total}
    \bra{\Psi}\hat{\mathcal{L}}\bar{H}\ket{\Psi} = E.
\end{equation}

As will be explored in Section \ref{se:res}, molecular properties may be expressed as 
derivatives of the electronic energy.\cite{Helgaker2012} Given the left- and right-hand wave functions 
and assuming the Hellmann-Feynman theorem\cite{Hellmann1937,Feynman1939} holds 
(and thus $\ket{\Psi}$ carries no dependence 
on external perturbations), the derivative of the energy with respect to an arbitrary 
perturbation operator 
$\hat\Omega$ can be conveniently expressed
\begin{equation} \label{eq:dE}
    \begin{aligned}
    \frac{\partial E}{\partial \hat\Omega} &= \bra{\Psi}\frac{\partial \hat{H}}{\partial \hat\Omega}\ket{\Psi}\\
                                  &= D_{pq}\frac{\partial h_{pq}}{\partial \hat\Omega} + \Gamma_{pqrs}\frac{\partial g_{pqrs}}{\partial \hat\Omega}
    \end{aligned}
\end{equation}
%% NOTE: a lot of notation to clean up here
without further specification of the wave function or Hamiltonian. 
Thus, time-independent first-order properties may be expressed without 
differentiation of the RDMs. This was a driving force behind the study in 
[PAPER 1], and is important for the machine-learning methods discussed in [PAPER 2]. The cases of 
higher-order and time-dependent properties, which are the
focus of [PAPER 1] and [PAPER 3], will be handled in the following section. 

%\input{theory/loc.tex}
\section{Response theory} \label{se:res}
Response theory combines adiabatic perturbation theory with a Fourier transform of the 
time-dependent equations into the frequency domain\cite{Pederson2020}. The prediction of the response 
of molecular systems to an electromagnetic field (EMF) has important applications in organic synthesis and
characterization. To explore these effects, we may relate properties to a perturbative expansion of a 
total electric or magnetic moment, from which response tensors related to a host of properties may be defined. 
These response tensors generally take the form of products of ground- and excited-state transition moments of 
the electric or magnetic dipole operator. In this chapter we will summarize the derivation of these 
response tensors and their relationships to field-induced molecular phenomena. We then present a method 
for obtaining these tensors through Fourier transform of a time-dependent ``Quasi-energy'' which is 
rigorously defined in the context of electronic structure theory, providing working equations for 
predicting molecular responses to EMF using approximate wave function-based theory.

\subsection{Exact Response} \label{ss:exact}
Under classical electrodynamics, the potential energy $V$ of a system of charges in a static electric field 
may be written in a multipole expansion:

\begin{equation} \label{eq:pot_V}
V = q(\phi)_0 - \mu_\alpha(E_\alpha)_0 - \frac{1}{3}\Theta_{\alpha\beta}(E_{\alpha\beta})_0 + \ldots
\end{equation}
with the scalar potential $\phi$, the cartesian component $\alpha$ of the electric field $E_\alpha$, and the electric field gradient with respect to two cartesian coordinates $E_{\alpha\beta}$. Here we have defined the multipoles as the electric monopole or charge $q$, the electric dipole $\mu_\alpha$, and the electric quadrupole $\Theta_{\alpha\beta}$. The subscript $0$ indicates that the potential, field, or gradient is taken at the system origin.

We may also expand the potential in a Taylor series about the zero-field potential:
\begin{equation} \label{eq:tay_V}
    V[E_0] = V_0 + (E_\alpha)_0\frac{\partial V}{\partial (E_\alpha)_0} + \frac{1}{2}(E_\alpha)_0(E_\beta)_0\frac{\partial V^2}{\partial (E_\alpha) \partial (E_\beta)} + \ldots
\end{equation}

From Eq.~(\ref{eq:pot_V}), we see that a cartesian component of the electric dipole may be written as the derivative of the potential with respect to the field, 
\begin{equation} \label{eq:mu_part}
    \mu_\alpha = -\frac{\partial V}{\partial (E_\alpha)_0}.
\end{equation}
By comparing Eqs.~(\ref{eq:tay_V}) and (\ref{eq:mu_part}), we see that the total electric dipole may be re-written 
\begin{equation} \label{eq:mu_tot}
\mu_\alpha = \mu_{0\alpha} + \alpha_{\alpha\beta}(E_\beta)_0 + \frac{1}{2}\beta_{\alpha\beta\gamma}(E_\beta)_0(E_\gamma)_0 + \ldots
\end{equation}
where we have defined the response tensors for the permanent (field-independent) dipole moment $\mu_{0\alpha}$, electric polarizability $\alpha_{\alpha\beta}$, first electric hyperpolarizability $\beta_{\alpha\beta\gamma}$, and higher order terms as higher-order derivatives of the potential, \textit{viz.}
\begin{subequations} \label{eq:derivs}
    \begin{equation} \label{eq:mu_0}
    \mu_{0\alpha} = -\frac{\partial V}{\partial (E_\alpha)_0}
    \end{equation}

    \begin{equation} \label{eq:alpha}
    \alpha_{\alpha\beta} = -\frac{\partial^2 V}{\partial (E_\alpha)_0 \partial (E_\beta)_0}
    \end{equation}

    \begin{equation} \label{eq:beta}
    \beta_{\alpha\beta\gamma} = -\frac{\partial^3 V}{\partial (E_\alpha)_0 \partial (E_\beta)_0 \partial (E_\gamma)_0}.
    \end{equation}
\end{subequations}
Property tensors for magnetic and mixed electric-magnetic properties are defined in an analogous manner. 

Quantum mechanical expressions for the property tensors may be obtained through perturbation theory. 
For static-field properties, this process is straightforward, and expressions similar to those found for
correlation energy corrections in ~Eq.({\ref{eq:E2}) arise in the perturbative expansion of the time-independent Schr\"odinger
equation, e.g., for the static polarizability tensor:
\begin{equation} \label{eq:static_pol}
\alpha_{\alpha\beta} = -2\sum_{j\neq n}\frac{\bra{n}\mu_\alpha\ket{j}\bra{j}\mu_\beta\ket{n}}{V_n - V_j} 
\end{equation}
where the sum runs over excited states $j$ and $n$.

Of more practical application is the prediction of dynamic-field properties, such as dynamic polarizabilities, 
optical rotation, and circular dichroism (which are the primary targets of [PAPER1] and [PAPER3]).
For these and other time-dependent response properties, some description of the time evolution of the wave function is
required. 
These properties may be solved directly from the time-dependent analogue of ~Eq.(\ref{eq:mu_tot}). 
To compute the dynamic dipole, one must compute the dipole moment expectation value of a system in the presence of an explicitly
time-dependent electric field.
Fourier transforming the dynamic dipole into the frequency domain would produce broadband spectra.
However, explicit time propagation of the wave function is exceedingly expensive. This is the topic of chapters [TDCC] and [PAPER 3]. 
For now, we will consider an alternative approach which computes properties at individual frequencies, 
but does not require explicit propagation of the wave function. 

Beginning from the time-dependent Schr\"odinger equation,
\begin{equation} \label{eq:tdse}
    \hat{H}(t)\ket{\Psi(t)} = i\hbar\partt{}\ket{\Psi(t)}
\end{equation}
we will, without loss of generality, assume a time-dependent wave function expansion of the form
\begin{subequations} \label{eq:td_wfn}
    \begin{equation}
        \ket{\Psi(t)} = d_n(t)\ket{\psi(t)}
    \end{equation}
    \begin{equation}
        \ket{\psi(t)} = e^{-iE_nt/\hbar}\ket{n}
    \end{equation}
\end{subequations}
where $\ket{\psi(t)}$ is an approximate wave function consisting of a set of known, time-independent functions $\ket{n}$
and an exponentiated phase. Expressing the Hamiltonian as in ~Eq{(\ref{eq:H}) as a sum of the time-independent molecular
Hamiltonian $\hat{H}^{(0)}$ with a time-dependent perturbation $\hat{V}(t)$
(generally taken to be an EMF) and inserting this and the wave function expansion into ~Eq.({\ref{eq:tdse}) yields
\begin{equation} \label{eq:sub_tdse}
    (\hat{H}^{(0)} + \hat{V}(t))d_n(t)e^{-iE_nt/\hbar}\ket{n} = i\hbar \partt{}[d_n(t)e^{-iE_nt/\hbar}]\ket{n}.
\end{equation}
It should be noted that, in the absence of a perturbation (time $t = 0$), this reduces to the time-independent Schr\"odinger equation ($d_n(0) = 1$). Using the chain rule, ~Eq.({\ref{eq:sub_tdse}}) becomes  
\begin{equation} \label{eq:chain_tdse}
    (\hat{H}^{(0)} + \hat{V}(t))d_n(t)e^{-iE_nt/\hbar}\ket{n} = [E_n d_n(t) + i\hbar \partt{d_n(t)}]e^{-iE_nt/\hbar}\ket{n}.
\end{equation}
If we assume we have \textit{exact} eigenstates $\ket{n}$ of the molecular Hamiltonian, \textit{i.e.} exact ground and excited state wave functions, then the second terms of both sides of ~Eq.(\ref{eq:chain_tdse}) cancel. Projecting on the left by 
excited states $\bra{m}$ (which contains only one state nonorthogonal to $\ket{n}$ and thus collapses the sum on the 
right-hand side) and rearranging, we arrive at the differential equation
\begin{equation} \label{eq:diffeq}
    \partt{d_m(t)} = \frac{1}{i\hbar}\bra{m}\hat{V}(t)\ket{n}d_n(t)e^{i\omega_{mn} t}
\end{equation}
which describes the time evolution of the wave function coefficient $d_m(t)$. We have also introduced the angular 
frequency $\omega_{mn} = (E_m - E_n)/\hbar$. 

Integration of ~Eq.(\ref{eq:diffeq}) from time $t' = 0$ to $t' = t$ yields a recursive equation for the 
wave function parameter $d_m(t)$, due to the presence of $d_n(t)$ on the right-hand side. 
However, as a consequence of our assumption that the system begins in its unperturbed ground state, all
coefficients at $t' = 0$ reduce to $\delta_{m,0}$. To linear order in $\hat{V}$, ~Eq.(\ref{eq:diffeq}) may be written
\begin{equation} \label{eq:irt}
    d_m(t') = \frac{1}{i\hbar}\int_0^t\bra{m}\hat{V}(t^{'})\ket{0}e^{i\omega_{m0}t^{'}}dt^{'}.
\end{equation}
%\begin{equation} \label{eq:irt}
%    \begin{aligned}
%    d_m(t') = 1 &+ \frac{1}{i\hbar}\int_0^t\bra{m}\hat{V}(t^{'})\ket{n}d_n(t^{'})e^{-i\omega_{mn}t^{'}}dt^{'} \\
%            = 1 &+ \frac{1}{i\hbar}\int_0^t\bra{m}\hat{V}(t^{'})\ket{n}e^{-i\omega_{mn}t^{'}}dt^{'} \\
%                &- \frac{1}{\hbar}\int_0^t\int_0^{t^{'}}\bra{m}\hat{V}(t^{'})\ket{l}\bra{l}\hat{V}(t^{''})\ket{n}e^{-i\omega_{ml}t^{'}}e^{-i\omega_{ln}t^{''}}dt^{''} \\
%                &+ \ldots
%    \end{aligned}
%\end{equation}

To continue, we choose our perturbation to be a simple, time-dependent field at a given frequency $\omega$ 
\begin{equation} \label{eq:field}
\hat{V}(t) = \hat{v}^\omega_\alpha F^\omega_\alpha e^{-i\omega t}
\end{equation}
with field strength $F^\omega_\alpha$ and arbitrary field operator $\hat{v}^\omega_\alpha$, such as the electric or magnetic dipole operator. 
Note that the implicit summation runs over both $\omega$ and $\alpha$, and the sum in ~Eq.(\ref{eq:irt}) runs over excited states $\ket{m}$.
Finally, inserting ~Eq.(\ref{eq:field}) into ~Eq.(\ref{eq:irt}) and integrating, we arrive at the final expression for the
wave function parameters
\begin{equation} \label{eq:sot}
    \begin{aligned}
        d_m(t) = \frac{1}{\hbar}\frac{\bra{m}\hat{v}^\omega_\alpha\ket{0}e^{i(\omega_{m0}-\omega)t}}{\hbar(\omega_{m0}-\omega)}F^{\omega}_{\alpha}.
%        d_m(t') = 1 &+ \frac{-1}{\hbar}\frac{\bra{m}\hat{v}^\omega_\alpha\ket{n}}{\omega_{mn} - \omega_1}e^{-i\omega_{mn}t}e^{-i\omega_1t} \\
%                &- \frac{1}{\hbar}\frac{\bra{m}\hat{v}^{\omega1}_\alpha\ket{l}\bra{l}\hat{v}^{\omega2}_\beta\ket{n}}{(\omega_{ml}-\omega_1)(\omega_{ln}-\omega_2)}e^{-i(\omega_{ml}-\omega_{ln})t}e^{-i(\omega_1-\omega_2)t} \\
%                &+ \ldots
    \end{aligned}
\end{equation}
 
To arrive at a dipole expression in the form of ~Eq.(\ref{eq:mu_tot}), we may compute the expectation 
value of the electric dipole operator $\hat{\mu}$. After rearrangement, we arrive at
\begin{equation} \label{eq:sos}
    \begin{aligned}
    \bra{\Psi(t)}\hat\mu\ket{\Psi(t)} &= \bra{0}\hat\mu\ket{0} \\ 
    &- \sum_{m\neq0}[\frac{\bra{m}\hat\mu\ket{0}\bra{0}\hat{v}^\omega_\alpha\ket{m}}{\hbar(\omega_{mo}+\omega)}
        + \frac{\bra{0}\hat\mu\ket{m}\bra{m}\hat{v}^\omega_\alpha\ket{0}}{\hbar(\omega_{mo}-\omega)}]
    e^{-i\omega t}F^\omega_\alpha
    \end{aligned}
\end{equation}
where we have truncated at the linear response function of $\hat{\mu}$ perturbed by $\hat{V}(t)$. This is known as the \textit{exact} linear response function. In general, 
response functions can be identified by the expansion of an operator $\hat\Omega$:
\begin{equation}
    \begin{aligned}
    \bra{\Psi(t)}\hat\Omega\ket{\Psi(t)} &= \bra{0}\hat\Omega\ket{0} \\ 
    &+ \linresp{\hat\Omega}{\hat{v}^{\omega_1}_\alpha}e^{-i\omega_1 t}F^{\omega_1}_\alpha \\ 
    &+ \quadresp{\hat\Omega}{\hat{v}^{\omega_1}_\alpha}{\hat{v}^{\omega_2}_\beta}e^{-i(\omega_1+\omega_2)t}F^{\omega_1}_\alpha F^{\omega_2}_\beta \\
    &+ \ldots
    \end{aligned}
\end{equation}
where we have introduced notation for the linear $\linresp{\hat\Omega}{\hat{v}^{\omega_1}_\alpha}$, 
quadratic $\quadresp{\hat\Omega}{\hat{v}^{\omega_1}_\alpha}{\hat{v}^{\omega_2}_\beta}$, and higher-order response functions.

\subsection{Approximate Response} \label{ss:apprx}
Sum-over-states expressions like those in Eq.~(\ref{eq:sos}) can be solved by computing the excited state wave functions and 
summing their individual contributions to the property. This poses a unique challenge in that the response tensor is not a product 
of only one targeted excited state, but of all possible excited states. 
While it is expected many high-energy states will have negligible contributions, the number of states 
required to accurately model these properties makes this approach prohibitively expensive for excited-
state extensions to ground-state  methods, such as equation-of-motion (EOM) CC. We can avoid this 
problem by building response equations based on approximate ground-state wave functions, such as ~Eq.(\ref{eq:cc_TISE}).
This eliminates the ability to compute excited-state wave functions, but allows us to replace the sum-over-states 
expression with a set of coupled linear equations which are far more computationally tractable.

Beginning with a time-dependent wave function $\ket{\Psi(t)}$, we may separate this into two time-dependent pieces - the exponential of a phase factor $\phi(t)$, and a phase-isolated wave function $\ket{\piw}$:
\begin{equation} \label{total_psi}
    \ket{\Psi(t)} = e^{-i\phi(t)}\ket{\piw}.
\end{equation}
We may require that the phase of the projection of $\ket{\piw}$ onto the ground-state wave function be 
zero - in other words, in the limit of zero time-dependent perturbation, $\ket{\piw}$ reduces to the
ground state wave function. This is analogous to our approach in ~Eq.(\ref{eq:td_wfn}), where our focus has
now shifted from the wave function parameters to the phase factor.
Inserting these definitions into the time-dependent Schr\"odinger equation,
we arrive at Eq.~(\ref{eq:quasi_schro}):
\begin{equation} \label{eq:quasi_schro}
    (\hat{H} - i\hbar \partt{})\ket{\piw} = Q(t)\ket{\piw}
\end{equation}
where we have defined the time-dependent quasi-energy as the reduced Planck constant times the derivative
of the phase factor,
\begin{equation} \label{eq:quasi_e}
    Q(t) = \hbar\partt{\phi(t)}.
\end{equation}
It is important to note that, just as the phase-isolated wave function reduces to the ground state wave function in the absence of a perturbation, the quasi-energy also reduces to the ground state energy in this case. 

It can be shown that the quasi-energy is manifestly real, and both a time-dependent variational principle and Hellmann-Feynman theorem apply. To obtain a time-independent quantity to perturbatively expand, the quasi-energy is integrated over time to form the time-averaged quasi-energy $Q_T$. The variational and Hellmann-Feynman theorems can then be written
\begin{equation} \label{eq:var}
    \partial Q_T = 0
\end{equation}
and
\begin{equation} \label{eq:hf-theorem}
    \frac{dQ_T}{d\hat\Omega} = \frac{1}{T}\int_{t^{'}}^{t^{'}+T}\bra{\piw}\frac{\partial \hat{H}}{\partial\hat\Omega}\ket{\piw}dt^{'},
\end{equation}
respectively, for an arbitrary perturbation $\hat\Omega$ (usually taken to be an electromagnetic field).

In the preceding derivation, at no point was it assumed that we have \textit{exact} eigenfunctions of the time-independent 
Hamiltonian. Thus, the above equations are valid for approximate ground-state theories, such as CC. We may now define
our approximate response functions as we did in ~Eq.(\ref{eq:derivs}) as derivatives of this time-averaged Quasi-energy,
\begin{subequations}
    \begin{equation}
        \linresp{\hat\Omega}{\hat v^{\omega_1}_{\alpha}} = \frac{d^2Q_T}{d\hat\Omega dF^{\omega_1}_\alpha}
    \end{equation}
    \begin{equation}
        \quadresp{\hat\Omega}{\hat{v}^{\omega_1}_\alpha}{\hat{v}^{\omega_2}_\beta} = \frac{d^3Q_T}{d\hat\Omega dF^{\omega_1}_\alpha dF^{\omega_2}_\beta}
    \end{equation}
\end{subequations}
and working equations may be derived by inserting the specific approximate wave function ansatz.

The response functions of interest to the present work are the linear response functions between 
the electric dipole and an electric field
$\linresp{\hat\mu}{\hat{\mu}}$,
and the magnetic dipole in an electric field
$\linresp{\hat m}{\hat{\mu}}$. 
These are responsible for the dynamic electric polarizability and chiroptical response (optical
rotation and circular dichroism) respectively. The latter constitutes a long-standing challenge
for response theory, requiring mixed electric- and magnetic-field derivatives and minimal room
for error, but is also a 
prime candidate for comparisons to experiment. Some limitations of and alternatives to response
theory will be explored in chapters [PAPER 1] and [PAPER 3].


    \chapter{Basis Set Superposition Errors in the Many-Body Expansion of Molecular Properties} \label{ch:p1}
\chapter{Introduction} \label{ch:int}

Modern synthetic organic chemistry employs a vast array of sophisticated instrumentation.
Principle among these are probes of light-matter interaction, which reveal rich structural and 
electro-magnetic characteristics. These instruments work by measuring the absorption, 
reflection, or refraction of an electromagnetic field (EMF) interacting with a target system. 
Experimental techniques which measure these interactions with respect to the energy or frequency
of the incident EMF are known as spectroscopy. Many staple experimental apparatus probe these
relationships, including (but certainly not limited to)
ultraviolet-visible absorption (UV-Vis), 
optical rotatory dispersion (ORD), 
flame atomic absorption (flame AA),
and electronic circular dichroism (ECD).
These data can be used for characterization or establishing structure-property relationships 
which aid in the development of novel systems for applications in materials, bio-organics, 
and more. 
Interpreting or predicting the results of such experiments requires a knowledge of fundamental 
light-matter interactions on a quantum level.  

A quantum description of any molecular interaction may be viewed as the effects of a perturbation
on a quantum-mechanical system. In the case of EMF, this perturbation may be ``static'' (fixed)
or ``dynamic'' (varying in frequency or time). A fully quantum-mechanical description of the 
light-matter system would require quantum electrodynamics (QED); however, it is often sufficient
to treat the EMF from a classical perspective, treating only the molecular response using
quantum mechanics. This allows us to utilize many well-established methods in the field of 
theoretical chemistry. These generally provide a representation of the system (a wave function or 
density) with which to take the expectation value of a given operator or, in some cases, predict 
the expectation value directly. Methods which utilize only mathematical techniques (that is, no
experimental or phenomenological parameters) are said to be \textit{ab initio} methods. These 
methods often (but not always) simplify or ignore the quantum effects of the nuclei and their 
motion, an approximation known as the Born-Oppenheimer approximation. 
Another subclass of \textit{ab initio} methods, which 
center around solving the time-dependent or time-independent Schr\"odinger wave equation to 
obtain an explicit form of the wave function, are 
known as wave function-based methods. Finally, methods that go beyond the mean-field or Hartree-Fock 
approximation, in which electrons only interact through an average self-consistent field, are 
classified as ``correlated'' methods. 
It is these correlated wave function-based methods under the Born-Oppenheimer
approximation upon which the bulk of the current work is built. As such, following this in Chapter 
\ref{ch:theory} is a primer in the theoretical underpinnings necessary to understand this work which are not described 
in detail in the publications that follow in sections [PAPER 1-3]. Chapter \ref{ch:theory} is roughly 
divided into details of electronic structure theory (Section \ref{se:est}) and molecular response 
properties (Section \ref{se:res}). 

First and foremost is many-body perturbation theory (MBPT). This framework allows us to 
separate the quantum mechanical properties of the isolated system versus some perturbing 
force, which is expanded in orders and truncated under the assumption that higher-order terms 
become negligible.
This force is often taken to be electron correlation when describing the electronic ground 
state wave function; however, it may also be a static or dynamic EMF. In section (\ref{ss:mp2})
we present some basic characteristics of general MBPT, in preparation for its application in the 
chapters that follow.

Secondly, in section (\ref{ss:cc}) we present the wave function-based method upon which the bulk 
of this work is based or seeks to approximate: coupled cluster (CC) theory. 
This method, like perturbation theory, is most commonly used for computing electron correlation 
(though its roots are in nuclear physics). 
Unlike perturbation theory, we do not perform an order-by-order expansion of the perturbation; 
instead, a ``cluster'' operator folds in contributions based on a physical intuition, that is the 
instantaneous occupation of many quantum states which give rise to electron correlation. Truncation
is then based on the number of quantum states (substituted or `excited` determinants). This method,
while accurate and systematically improvable, is notoriously expensive, suffering from high-order
polynomial scaling. Extensions to excited states and molecular response properties only compound this 
issue; as such, [PAPER 1] and [PAPER 2] are focused on ways to circumvent using CC on any but a small 
subset of systems. (It should be noted that, while [Paper 1] utilizes a different correlated method 
based on a density-functional theory approach, the primary goal was to ascertain the effects of 
fragmentation of a system through the many-body expansion on the computation of molecular properties. 
This is a benchmark study, whose goal is understood to be conclusions that would be applied to more 
expensive methods such as CC, where such benchmark studies could not be performed.)
Additional considerations of reduced density matrices, which are used throughout the work to refer
to correlated wave functions, follows in Section \ref{ss:rdm}.

Finally, we describe the method by which we couple the classical perturbing field to the 
quantum-mechanical correlated wave function. 
By applying perturbation theory we derive general expressions for tensors which describe 
the molecular response to an EMF. In the exact theory, these tensors turn out to be functions of the 
excited electronic states of the system; however, an approach to generalize 
these expressions to approximate theories such as truncated coupled cluster is discussed, which also 
avoids the costly evaluation of excited-state wave functions. For dynamic EMF, these properties may be 
described as a function of frequency, so the most common approach is to derive the working equations
using a Fourier transform of expressions obtained using the time-dependent Schr\"odinger equation 
(more generally known as response theory); however, these expressions may also be evaluated explicitly
in the time-domain. In Section \ref{se:res} we focus on the frequency-domain formulation, while
[PAPER 3] explores the recently-revived prospect of explicit time-propagation, as well as a 
technique to make this process cheaper through a well-established concept known as 
``local correlation''. 

%comp.tex%

\section{Computational Details} \label{comp}
Second-order MBPT using a restricted closed-shell Hartree Fock reference wave function (i.e. M{\o}ller-Plesset perturbation theory, MP2)\cite{Møller1934,Bartlett1974a}, was used to generate both TATRs and DTRs for all systems. The TATR is generated from Eq.~(\ref{eq:mbtr}) using the highest 150 (by magnitude) amplitudes of double ``excitations'' 
$t_{ij}^{ab}$. 
For DTRs, the elements of the MP2 1-RDM are used. This choice is advantageous not just for comparison to Ref.~\citenum{Margraf2018}, but also due to the special form of the MP2 reduced density matrices\cite{Trucks1988}:

\begin{subequations} \label{eq:mp2_d}
    \begin{equation} \label{eq:mp2_opdm}
        D_{pq} = \bra{0}\hat T_2^{\dagger} a_p^{\dagger}a_q \hat T_2\ket{0}
    \end{equation}
    \begin{equation} \label{eq:mp2_tpdm}
        \Gamma_{pqrs} = 2\bra{0}a_p^{\dagger}a_q^{\dagger}a_sa_r \hat T_2\ket{0}.
    \end{equation}
\end{subequations}
where the ``excitation'' operators are defined as $\hat T_n = \left(\frac{1}{n!}\right)^2 t_{ij\ldots}^{ab\ldots}a_a^{\dagger}a_b^{\dagger}\ldots a_j a_i$, and we have again implied Einstein summation notation this time over occupied ${i,j,\ldots}$ and virtual ${a,b,\ldots}$ orbital spaces.
The form of Eqs.~(\ref{eq:mp2_opdm}) and~(\ref{eq:mp2_tpdm}) indicates that the reduced density matrices are fully defined by the doubles amplitudes and products thereof. In this light, it is reasonable to suggest that all of the information necessary to reproduce the wave function is coded into either of these and, to reduce the size of the representation, the 1-RDM should suffice for describing the MP2-level correlated wave function. Thus, any correlated one-electron property at the MP2 level can be described by this matrix.

Coupled cluster with single and double excitations (CCSD) was used to generate the high-level reference data, i.e. correlated energies and electronic dipole moments, for all systems. This method represents a marked improvement over MP2 in systems where correlation is strong, such as stretched diatomics, without incurring the additional cost of the gold-standard CCSD with perturbative triples method, CCSD(T). However, the CCSD(T) method could be used to provide higher quality targets (e.g. energies and dipoles) with no additional modifications --- as noted in section~\ref{algorithm}, the algorithm as presented is completely general. 

All electronic structure calculations were performed with the Psi4\cite{Parrish2017} electronic structure package. Molecular symmetry was used to maximize TATR performance with as few amplitudes as possible (except in section \ref{cutoffs}). All data (energies, dipoles, amplitudes, and densities) were harvested in full machine precision through either the Psi4-JSON interface or the PsiAPI infrastructure, with the exception of wave function amplitudes at the MP2 level, which were printed in the output file at ten decimal places. The def2-TZVP\cite{Weigend2005} basis was employed for diatomic calculations for comparison with Ref.~\citenum{Margraf2018}, and the aug-cc-pVDZ (aDZ) basis\cite{Dunning1989,Woon1994} was used for all remaining calculations. Density-fitted integrals were used in the construction of MP2 densities using the default auxiliary basis (the so-called ``RI'' basis sets for def2-TZVP\cite{Hattig2005} and aDZ\cite{Weigend2002}, obtained from the New Basis Set Exchange\cite{Pritchard2019}). 

As outlined above, the general procedure from Margraf and Reuter\cite{Margraf2018} was followed for machine-learning. 
The hyperparameters $\sigma_m$ and $\lambda$ were optimized on a uniform grid from $10^{-8}:10^{8}$ for all models unless otherwise specified, using the ``negative-mean-squared'' loss function. 
All regressions were performed using 20 training points, except in section \ref{cutoffs} where 12 training points were used to emphasize the difference between the two strategies considered. 
k-means clustering was repeated 30 times to account for stochastic deviations in the algorithm. 
Test sets consisting of $N/4$ points are held back before the clustering step.
The radial basis function kernel (sometimes called the Gaussian kernel), Eq.~(\ref{eq:rbf}), was used for all representations. 
Machine-learning algorithms were performed using Scikit-Learn (skl) \cite{Pedregosa2011} and the Machine-Learning Quantum Mechanics (MLQM) python package\cite{mlqm}, which generates a number of molecular representations, provides options for generating Psi4 input files and harvesting their results, and wraps some skl functions. 
Both of these codes are open-source and freely available on GitHub.

Geometries for diatomics were constructed by taking $N = 200$ uniform increments from 0.5-2.0 \AA . All other geometries were randomly selected from MD simulations ($N = 150$) carried out in the Gromacs\cite{Pronk2013} software package.
Simulations of (\textit{S})-methyloxirane and (\textit{R})-methylthiirane surrounded by a 5\AA\ box of water molecules were executed using an all-atom OPLS/AA forcefield\cite{Jorgensen1996} for the solute and the TIP3P model for water\cite{Jorgensen1983}. Each 5 ns trajectory was carried out in the NVT
ensemble, with the solute and solvent coupled separately to a temperature bath
at 300 K using a modified Berendsen thermostat and a coupling time of 0.1 ps. Geometries were selected from a set of 250 evenly-spaced snapshots along the trajectory.

Geometries, Psi4 input and output files, and their accompanying optimized (hyper)parameter values in JSON format can be found at the Virginia Tech Data Repository\cite{vtdata2020}   
to permit reproducibility of our results. Scripts for harvesting and manipulating data using MLQM are also included in the form of Jupyter Notebooks.

%results.tex

\section{Results and Discussion} \label{se:results} 
Here we present results from the first applications of local correlation to RTCC.
Results are examined
by the convergence of absorption and ECD spectra to the reference results
in Section \ref{ss:spectra}, followed by an analysis of the amplitude
dynamics in Section \ref{ss:amps}. 
%Section \ref{ss:ext} considers the
%effects of localization on orbital spatial extent between PAO and PNO
%virtual spaces, and contrasts these results with the amplitude data in
%an attempt to explain their performance. 
Section \ref{ss:alt} explores some potential solutions for building
appropriate virtual spaces for truncation, such as considerations
of orbital extent,
perturbation-aware virtual spaces\cite{Crawford2019,DCunha2021}, and 
including or focusing on the effects of the singles amplitudes. 
%Finally, we present an alternative solution dubbed ``Semi-static RTCC'' 
%(SS-RTCC) in Section \ref{ss:ssrtcc} in which the doubles amplitudes 
%($t_2$ and $\lambda_2$ in Eqs.~(\ref{eq:diff_t}) and (\ref{eq:diff_l}), 
%respectively) are frozen after the determination of the ground-state wave 
%function, and only the singles amplitudes are allowed to respond to the 
%incident perturbation. 

\subsection{Absorption and ECD Spectra} \label{ss:spectra}
\subsubsection{Absorption} \label{sss:abs}
Absorption spectra are obtained from the Fourier transform of Eq.~(\ref{eq:abs}).
Figure~\ref{fig:pno_abs} shows the normalized absorption spectrum obtained from
a reference propagation along with five PNO cutoffs. The average truncated virtual 
orbital spaces are from roughly 20\% to 90\% of the MO virtual space (see caption). 
\begin{figure} 
    \centering
    \includegraphics[scale=.6]{p3/figures/pno_abs.png}
    \caption{Reference and PNO absorption spectra for five cutoffs: 
    [$1\times 10^{-10}$, $1\times 10^{-9}$, $1\times 10^{-8}$, $1\times 10^{-7}$, 
    $2\times 10^{-6}$] corresponding to [$93\%$, $82\%$, $63\%$, $44\%$, $24\%$]
    of the MO virtual space, respectively.}
    \label{fig:pno_abs}
\end{figure}
%For all truncated PNO virtual spaces considered, the base peak appears to within 1.5 eV of the 
%reference. 
Overall, truncated PNO virtual spaces approximate the position of the base peak well,
with the smallest space predicting a base peak within 1.5 eV of the reference,
and the two largest spaces predict this peak to within 0.2 eV of the reference.
Convergence to the reference base peak occurs from the right, indicating
a lowering of excited state energies as the size of the virtual space increases. 
This trend can also be seen for the smaller peak near 10 eV. However, convergence of the shoulder 
peaks on either side of the base peak, indicated by the inset of 
Figure~\ref{fig:pno_abs}, is less predictable. Even the largest spaces considered
do not correctly predict the excitation energy, with no clear advantage to having
93\% of the virtual space as compared to just 83\% for predicting these peaks.
This trend continues into the higher-energy range of the spectrum, with the 
performance of each cutoff being nearly indistinguishable. 

Performance of the PAO space is shown in Figure~\ref{fig:pao_abs}. 
\begin{figure} 
    \centering
    \includegraphics[scale=.6]{p3/figures/pao_abs.png}
    \caption{Reference and PAO absorption spectra for five cutoffs: 
    [$1\times 10^{-4}$, $1\times 10^{-3}$, $1\times 10^{-2}$, $5\times 10^{-2}$, 
    $1\times 10^{-6}$] corresponding to [$95\%$,$86\%$,$63\%$,$46\%$,$23\%$]
    of the MO virtual space, respectively.}
    \label{fig:pao_abs}
\end{figure}
The largest truncated PAO virtual space, on average 95\% of the MO space, accurately 
predicts the excitation energies for each major peak below 17 eV. Particularly around 10 eV, 
this is noticeably improved performance relative to the largest PNO space tested, 
with only a 2\% difference in the average size of the virtual space. 
However, accuracy rapidly declines even at 86\% of the virtual space, where the base peak
position is already worse than what was predicted with a PNO space of just 63\% of the 
MO space. Performance continues to degrade as energy increases and the average size of 
the PAO space decreases. For the final two cutoffs, at averages of 46\% and 23\% of the
MO space, the base peaks are 3 eV or more away from the reference, and no peak is 
exhibited near 25 eV. These spaces also fail to predict the second largest peak, the 
excitation just below 10 eV. 

\subsubsection{ECD} \label{sss:ecd}
Overall, neither scheme produced adequate results upon truncation of the virtual space. 
This result is not entirely surprising -- in studies of local correlation applied to 
response theory by Crawford \textit{et al}., 
\cite{McAlexander2016,Kumar2017,Crawford2019,DCunha2021} 
traditional schemes proved inaccurate for another 
electric dipole--electric dipole property, the electric polarizability.
In terms of response theory,
the polarizability (and the refractive index) is related to the \textit{real} part 
of the electric dipole--electric dipole linear response tensor 
($\boldsymbol{\alpha}_{ij}$ in Eq.~(\ref{eq:mu_exp})), 
while absorption
is related to the \textit{imaginary} part. Indeed, all linear absorptive properties
such as absorption and CD 
are related to the imaginary component of a linear response tensor, while dispersive 
properties such as refractive index and 
circular birefringence (also known as optical rotation)
are related to the real component.
\cite{Barron2004,Norman2011}
To continue, we will look at another absorptive property which is related to the mixed
electric dipole--magnetic dipole linear response tensor -- ECD.

The ECD spectrum is obtained from the Fourier transform of Eq.~(\ref{eq:ecd}).
Being a bisignate, mixed-response property, ECD is a considerable computational
challenge, similar to its dispersive counterpart circular birefringence. 
Figure~\ref{fig:pno_ecd} shows the results for an ECD spectrum in the same PNO 
orbital spaces used in the previous section.
\begin{figure} 
    \centering
    \includegraphics[scale=.6]{p3/figures/pno_ecd.png}
    \caption{Reference and PNO ECD spectra for five cutoffs: 
    [$1\times 10^{-10}$, $1\times 10^{-9}$, $1\times 10^{-8}$, $1\times 10^{-7}$, 
    $2\times 10^{-6}$] corresponding to [$93\%$,$82\%$,$63\%$,$44\%$,$24\%$]
    of the MO virtual space, respectively.}
    \label{fig:pno_ecd}
\end{figure}
The dynamic response of the magnetic dipole to the electric field in this frequency 
range is considerably more complicated than that of the electric dipole. Below 60\% of
the MO space, virtually all distinguishing characteristics of the reference 
spectrum are unidentifiable. Further, at 82\%, the base peak appears to be a pair of 
peaks, more resembling the pair of peaks appearing just above 15 eV in the 
reference spectrum, with the major peak just below 15 eV being the second strongest.
At an average of 93\%, the overall \textit{shape} of the spectrum in the 10 eV to 
20 eV range more closely resembles that of the reference; however, the excitation
energies are, in some cases, even less accurate than those of smaller PNO spaces.
The trend of lowering excited state energies with increased virtual space seen in 
Section~\ref{sss:ecd} is no longer discernible. 

As in the case of absorption, the PAO basis is not noticeably more efficient at
approximating the full MO space than the PNO space. Figure~\ref{fig:pao_ecd} 
shows the results using the same truncated PAO spaces as in Section~$\ref{sss:abs}$.
\begin{figure} 
    \centering
    \includegraphics[scale=.6]{p3/figures/pao_ecd.png}
    \caption{Reference and PAO ECD spectra for five cutoffs: 
    [$1\times 10^{-4}$, $1\times 10^{-3}$, $1\times 10^{-2}$, $5\times 10^{-2}$, 
    $1\times 10^{-6}$] corresponding to [$95\%$,$86\%$,$63\%$,$46\%$,$23\%$]
    of the MO virtual space, respectively.}
    \label{fig:pao_ecd}
\end{figure}
The performance of the PAO basis near the base peak varies wildly with truncation,
as in the PNO case. In the low-frequency region, the PAO results are considerably 
worse -- see the two negative peaks between 10 eV and 15 eV. Curiously, the 
largest PAO spaces considered predict significant peaks above 25 eV that are not 
present in the reference, any of the PNO spaces tested, or the smaller PAO
spaces. This suggests a strong sensitivity of the response of the wave function
to the completeness threshold used for determining the occupied domains.
%fundamentally different electronic structure in the
%presence of the perturbing field, which is a direct result of the charge-based 
%completeness threshold used for determining the occupied domains.
%albeit at very large frequencies which are of little practical use.

\subsection{Amplitude Dynamics} \label{ss:amps}
As evidenced by the preceding data, the truncated PNO and PAO virtual spaces do not
efficiently model the wave function in the presence of a perturbing EMF. As noted in
Section~\ref{sss:abs}, these shortcomings are well-documented in the case of response
theory. However, a real-time formalism offers the opportunity to analyze the wave 
function in great detail over time, perhaps shedding light on \textit{where} and 
\textit{how} the locally correlated wave functions are deficient. The following
section will scrutinize the $t_\mu$ and $\lambda_\mu$ amplitudes of 
Eqs.~(\ref{eq:t_mu}) and (\ref{eq:l_mu}), respectively, in hopes of determining the 
important fluctuations in the wave function and whether these spaces sufficiently 
capture these changes.

Response to external perturbations by the CC amplitudes give rise to
dynamic energetics and properties. In the past, distributions of perturbed amplitudes
(relative to their ground-state counterparts) have been used to justify the 
difficulty in computing response functions with local correlation methods in the 
frequency domain. 
\cite{McAlexander2016,Crawford2019,DCunha2021} 
However, initial findings show that in RTCC, the relative distribution of amplitudes 
by magnitude is not significantly impacted.\cite{Crawford2019} 
Despite this, typical means of exploiting amplitude
sparsity have been shown to be inefficient by the preceding sections. 
First, to understand the response of 
the amplitudes to the external perturbation, we plot the 
change in the norm of the amplitude tensors relative to the ground-state amplitudes 
as a function of time in Figure~\ref{fig:norm}.
\begin{figure} 
    \centering
    \includegraphics[scale=.6]{p3/figures/amp_norm.png}
    \caption{Time-dependent change in the norm of the amplitude 
    tensors relative to the ground-state amplitudes.
    (Field and step parameters remain unchanged, and 
    the amplitude norm is taken at every 1 a.u.)}
    \label{fig:norm}
\end{figure}
Results for the untruncated PNO and PAO spaces are identical to those
for the MO space, as the unitary transformations resulting from untruncated localized virtual spaces
in Eqs.~(\ref{eq:rotate}) preserves the tensor norm.
Amplitude norms from propagations carried out in truncated PNO and PAO spaces
are nearly indistinguishable (see SI).

Figure~\ref{fig:norm} shows that the magnitude of the response by the wave function
is predominantly within the singles amplitudes $t_1$ and $\lambda_1$. This is 
consistent with the notion that singles are paramount for the computation of 
response properties.\cite{Christiansen1995,Koch1997} However, the form of Eq.~(\ref{eq:pair_D})
does not include any contributions by singles, due to being built from MP2-level
amplitudes where singles do not contribute until at least the second order 
in the wave function and fourth order in the energy. This suggests that even in schemes
which seek to include the EMF perturbation in the construction of the reduced virtual space,
such as PNO++,\cite{DCunha2021} response of the singles should be considered.

Aside from the matrix norm, we can also inspect the individual amplitudes to track
their evolution in time. The heat maps in Figure~\ref{fig:amps} show the difference 
in $t_1$ amplitudes, relative to the ground state, for three time steps selected 
from the first 100 a.u. of the simulation.
\begin{figure}
    \begin{subfigure}{.5\textwidth}
        \centering
        \includegraphics[scale=0.5]{p3/figures/MO_delta_t1_1.png}
        \caption{}
        \label{fig:MO_t1_1}
    \end{subfigure}%
    \begin{subfigure}{.5\textwidth}
        \centering
        \includegraphics[scale=0.5]{p3/figures/MO_delta_t1_50.png}
        \caption{}
        \label{fig:MO_t1_50}
    \end{subfigure}
    \begin{subfigure}{.5\textwidth}
        \centering
        \includegraphics[scale=0.5]{p3/figures/MO_delta_t1_100.png}
        \caption{}
        \label{fig:MO_t1_100}
    \end{subfigure}
    \caption{MO-basis $t_1$ amplitude deviations from $t = 0$ after (a) 1 a.u., (b) 50 a.u., and 
    (c) 100 a.u. of time propagation. Each row contains the same four occupied orbital indices
    and a subset of virtual indices as indicated by the x-axis labels.}
    \label{fig:amps}
\end{figure}
The amplitudes are ordered by the orbital energies of the associated MOs. 
The amplitudes which experience significant oscillations vary throughout the simulation,
though there are several discernible trends. First, most large amplitude deviations
are associated with all occupied orbitals simultaneously. This is due to the relatively small size of 
the system, with only four occupied orbitals, all of which are likely important in the
description of the ground- and excited-state wave functions. Secondly, 
at any given time during the propagation,
a large number of amplitudes have not significantly deviated from their ground state values.
This supports the notion
that relative sparsity is maintained within the amplitudes throughout the simulation,
but this sparsity is distributed differently throughout the amplitude tensors as
the wave function is propagated.  

A third trend is that amplitudes which respond strongly tend to be associated with 
low-energy virtual orbitals. Chemical intuition would suggest that energetically 
low-lying molecular orbitals will be the most involved in electronic excitations.
However, while amplitude responses are indeed larger for lower-energy virtual orbitals, 
smaller amplitude 
deviations in Figure~\ref{fig:amps} extend far into the virtual space. This explains
the difficulty of simply truncating with respect to orbital energy: the 
high-energy MOs are still important to the time-evolution of the wave function 
in the presence of an EMF. 

Figure~\ref{fig:pno_amps} shows the $t_1$ amplitudes for the same simulation,
rotated into the untruncated PNO basis using $Q_{ii}$ as defined in
Eq.~(\ref{eq:Q_pno}).
\begin{figure}
    \begin{subfigure}{.5\textwidth}
        \centering
        \includegraphics[scale=0.5]{p3/figures/PNO_delta_t1_1.png}
        \caption{}
        \label{fig:PNO_t1_1}
    \end{subfigure}%
    \begin{subfigure}{.5\textwidth}
        \centering
        \includegraphics[scale=0.5]{p3/figures/PNO_delta_t1_50.png}
        \caption{}
        \label{fig:PNO_t1_50}
    \end{subfigure}
    \begin{subfigure}{.5\textwidth}
        \centering
        \includegraphics[scale=0.5]{p3/figures/PNO_delta_t1_100.png}
        \caption{}
        \label{fig:PNO_t1_100}
    \end{subfigure}
    \caption{PNO-basis $t_1$ amplitude deviations from $t = 0$ after (a) 1 a.u., (b) 50 a.u., and 
    (c) 100 a.u. of time propagation. Each row contains the same four occupied orbital indices
    and a subset of virtual indices as indicated by the x-axis labels.}
    \label{fig:pno_amps}
\end{figure}
(It should be noted that, due to redundancy in the AO-based virtual 
spaces for each pair, PAO-basis amplitudes cannot be compared directly in 
this manner.)
It can be immediately seen that the amplitude deviations
are less sparse in the PNO basis after the application of the EMF. 
Many more amplitudes exhibit
perceivable differences, and strong deviations (magnitudes approaching 
$1\times 10^{-5}$) are no longer present. This is a clear demonstration
of the issue with truncating orbital spaces based on the present criterion ---
rather than exploiting sparsity, the amplitude tensors have become less sparse.
It may also suggest a recipe for building a more appropriate 
virtual space for truncation. 
In the following section, we propose some alternative schemes based on the 
literature and the results of this study.
%In the following section, we compare a selection
%of orbitals which correspond to strong amplitude deviations in the MO basis 
%(specifically virtuals 3, 4, 7, and 15)
%based on orbital spatial extent to determine if this may be a possible criterion 
%for truncation of the virtual space.

\subsection{Possible Alternatives} \label{ss:alt}
\begin{figure}
    \centering
    \includegraphics[scale=0.75]{p3/figures/extent.png}
    \caption{Virtual MO energy $\epsilon_a$ and the
    occupation number $n_a$ (plotted on a log scale)
    for unique PNO spaces $i_1$ and $i_2$ 
    versus orbital extent in arbitrary units.
    Virtual MOs 7 and 15 are denoted by a solid $\boldsymbol{+}$ 
    and $\boldsymbol{\times}$, respectively.
    The horizontal line denotes a PNO cutoff 
    of $1\times 10^{-7}$.}
    \label{fig:extent}
\end{figure}
Figure~\ref{fig:extent} shows the virtual MO energy $\epsilon_a$ and the 
PNO occupation number $n_a$ plotted against the orbital extent
$\langle r^2 \rangle$ in arbitrary units. 
In the PNO basis, a unique virtual space is prepared
for every occupied pair, resulting in 16 unique spaces for the four occupied
spatial orbitals $i$. However, for transforming
a single orbital index, we only require the diagonal rotation matrices,
\textit{i.e.}, $Q_{ii}$. There are four such spaces; however, by symmetry,
only two are unique. Both are included in Figure~\ref{fig:extent}. 

Truncation of the PNO space begins from the bottom of Figure~\ref{fig:extent}. 
At an occupation number cutoff of $1\times 10^{-7}$ (indicated by a horizontal
line), all orbitals below this line are neglected in the PNO space. Roughly
66\% of the virtual space lies in this region. From these data, it is clear
that even modest truncation of the virtual space neglects the diffuse 
regions of the wave function, which are important for excited-state
properties in systems with significantly delocalized characteristics,
such as systems containing Rydberg-type excitations.

%Table~\ref{ta:ext} reports the orbital extent $\langle r\rangle$ for virtual orbitals $a$
%corresponding to four of the strongest deviations in Figure~\ref{fig:amps}. 
%These orbitals are labeled 4, 7, 8, and 15, in
%both the MO and PNO basis. In the PNO basis, a unique virtual space is prepared
%for every occupied pair, resulting in 16 unique spaces for the four occupied
%spatial orbitals $i$. However, for transforming
%a single orbital index, we only require the diagonal rotation matrices,
%\textit{i.e.}, $Q_{ii}$. There are four such spaces; however, by symmetry, the domains of occupied
%orbitals 1 and 4 are the same, as are 2 and 3. Thus, orbital extents for only those
%two unique orbital spaces $i_1$ and $i_2$  are shown.
%\begin{table}
%    \centering
%    \begin{tabular}{|c|c|c|c|}
%        \hline
%%        $a$ &   MO  &   PNO   &        \\  
%        $a$ &  MO  &   \multicolumn{2}{c|}{PNO} \\  
%        \hline
%%        \multicolumn{2}{|c|}{}  & $i_1$   & $i_2$  \\ 
%            &       & $i_1$   & $i_2$  \\
%        \hline
%%         3  & 59.23 & 7.9     & -24.67 \\ 
%%        \hline
%         4  & 88.76 & -13.08  & -3.13  \\ 
%        \hline
%         7  & 66.00 & 0.73    & -0.07  \\ 
%        \hline
%         8  & 65.56 & -3.91   & -10.35 \\ 
%        \hline
%         15 & 37.39 & -15.11  &  1.32  \\
%        \hline
%    \end{tabular}
%    \caption{Orbital spatial extent of four selected virtual orbitals in the MO
%    and PNO spaces.}
%    \label{ta:ext}
%\end{table}
%A full table of virtual orbital extents can be found in the SI. 
%In the MO basis, the orbitals in Table~\ref{ta:ext} 
%correspond to the strongest amplitude deviations in Figure~\ref{fig:amps}.
%These are also some of the most diffuse orbitals in this basis,
%far larger than the average orbital extent of 26.62.
%As expected, the spatial extent of the PNOs built upon the MOs
%are much more localized. This shows that, in the MO basis,
%strong deviations are predominantly exhibited by amplitudes 
%corresponding to virtual orbitals with a larger spatial extent,
%creating sparsity. In the PNO basis, however, the deviations 
%are evenly spread across a large number of amplitudes corresponding
%to orbitals with relatively small spatial extent, resulting in 
%less sparsity in the amplitude tensor. 
%These findings suggest that virtual domains built for excited state
%properties should seek to include orbitals with large spatial extent,
%and any truncation criterion should preserve these orbitals.

Spatial extent alone may not be a suitable criterion for truncation -- 
this would have obvious a negative impact on the accuracy of the 
correlation energy, which is inherently local in nature. 
Additionally, Figure~\ref{fig:extent} highlights virtual MOs 7 
($\boldsymbol{+}$) and 15 ($\boldsymbol{\times}$),
which correspond to the strongest deviations in 
Figures~\ref{fig:MO_t1_50} and \ref{fig:MO_t1_100}, respectively.
That these orbitals are of varying extent
demonstrates that both diffuse and contracted orbitals play a role in 
the wave function dynamics.
In order to attain a balanced description of wave function components 
important for both energy and property calculations, the combination 
of appropriately determined spaces such as the combined PNO++ approach
has been fruitful. Still neglected in this approach
are the singles amplitudes, which are absent in the MP2 wave functions
used to approximate the occupied pair domains. Schemes to include 
these effects, such as approximate CC2-level $t_1$ guess amplitudes,
may further improve the space and allow greater flexibility for 
truncation. The prospect of utilizing these 
methodologies within the current framework is promising, and work is 
underway to explore their efficiency.

%conc.tex%
\section{Conclusions} \label{conc}
Here we introduce the density tensor representation (DTR) for machine-learning quantum mechanics applications. 
The representation is based on the previous t-amplitude tensor representation (TATR), with improvements made through strictly theoretical considerations of three categories: systematic improvement, storage, and simplified representation-target mapping. Investigating the limits of these categories on small test sets show a number of favorable properties. 
The DTR can be easily defined for any electronic structure method in which a density can be defined. When compared to the TATR for MP2, it achieves superior accuracy across most test cases when the MP2 wave function is expected to produce reasonable results. This accuracy is in the sub-mE$_h$ range for correlation energies. 
Furthermore, applications to molecular properties are both theoretically and operationally justified for representations utilizing electronic densities as raw wave function features. 
Roughly milliDebye error was achieved for correlated electronic dipole moments of several small molecules near equilibrium.
Extensions to include additional properties and molecular transferability are also considered, with the data-efficient DTR model providing a vital stepping stone to these generalizing improvements. 


    \chapter{Machine-Learning Coupled Cluster Properties through a Density Tensor Representation} \label{ch:p2}
\chapter{Introduction} \label{ch:int}

Modern synthetic organic chemistry employs a vast array of sophisticated instrumentation.
Principle among these are probes of light-matter interaction, which reveal rich structural and 
electro-magnetic characteristics. These instruments work by measuring the absorption, 
reflection, or refraction of an electromagnetic field (EMF) interacting with a target system. 
Experimental techniques which measure these interactions with respect to the energy or frequency
of the incident EMF are known as spectroscopy. Many staple experimental apparatus probe these
relationships, including (but certainly not limited to)
ultraviolet-visible absorption (UV-Vis), 
optical rotatory dispersion (ORD), 
flame atomic absorption (flame AA),
and electronic circular dichroism (ECD).
These data can be used for characterization or establishing structure-property relationships 
which aid in the development of novel systems for applications in materials, bio-organics, 
and more. 
Interpreting or predicting the results of such experiments requires a knowledge of fundamental 
light-matter interactions on a quantum level.  

A quantum description of any molecular interaction may be viewed as the effects of a perturbation
on a quantum-mechanical system. In the case of EMF, this perturbation may be ``static'' (fixed)
or ``dynamic'' (varying in frequency or time). A fully quantum-mechanical description of the 
light-matter system would require quantum electrodynamics (QED); however, it is often sufficient
to treat the EMF from a classical perspective, treating only the molecular response using
quantum mechanics. This allows us to utilize many well-established methods in the field of 
theoretical chemistry. These generally provide a representation of the system (a wave function or 
density) with which to take the expectation value of a given operator or, in some cases, predict 
the expectation value directly. Methods which utilize only mathematical techniques (that is, no
experimental or phenomenological parameters) are said to be \textit{ab initio} methods. These 
methods often (but not always) simplify or ignore the quantum effects of the nuclei and their 
motion, an approximation known as the Born-Oppenheimer approximation. 
Another subclass of \textit{ab initio} methods, which 
center around solving the time-dependent or time-independent Schr\"odinger wave equation to 
obtain an explicit form of the wave function, are 
known as wave function-based methods. Finally, methods that go beyond the mean-field or Hartree-Fock 
approximation, in which electrons only interact through an average self-consistent field, are 
classified as ``correlated'' methods. 
It is these correlated wave function-based methods under the Born-Oppenheimer
approximation upon which the bulk of the current work is built. As such, following this in Chapter 
\ref{ch:theory} is a primer in the theoretical underpinnings necessary to understand this work which are not described 
in detail in the publications that follow in sections [PAPER 1-3]. Chapter \ref{ch:theory} is roughly 
divided into details of electronic structure theory (Section \ref{se:est}) and molecular response 
properties (Section \ref{se:res}). 

First and foremost is many-body perturbation theory (MBPT). This framework allows us to 
separate the quantum mechanical properties of the isolated system versus some perturbing 
force, which is expanded in orders and truncated under the assumption that higher-order terms 
become negligible.
This force is often taken to be electron correlation when describing the electronic ground 
state wave function; however, it may also be a static or dynamic EMF. In section (\ref{ss:mp2})
we present some basic characteristics of general MBPT, in preparation for its application in the 
chapters that follow.

Secondly, in section (\ref{ss:cc}) we present the wave function-based method upon which the bulk 
of this work is based or seeks to approximate: coupled cluster (CC) theory. 
This method, like perturbation theory, is most commonly used for computing electron correlation 
(though its roots are in nuclear physics). 
Unlike perturbation theory, we do not perform an order-by-order expansion of the perturbation; 
instead, a ``cluster'' operator folds in contributions based on a physical intuition, that is the 
instantaneous occupation of many quantum states which give rise to electron correlation. Truncation
is then based on the number of quantum states (substituted or `excited` determinants). This method,
while accurate and systematically improvable, is notoriously expensive, suffering from high-order
polynomial scaling. Extensions to excited states and molecular response properties only compound this 
issue; as such, [PAPER 1] and [PAPER 2] are focused on ways to circumvent using CC on any but a small 
subset of systems. (It should be noted that, while [Paper 1] utilizes a different correlated method 
based on a density-functional theory approach, the primary goal was to ascertain the effects of 
fragmentation of a system through the many-body expansion on the computation of molecular properties. 
This is a benchmark study, whose goal is understood to be conclusions that would be applied to more 
expensive methods such as CC, where such benchmark studies could not be performed.)
Additional considerations of reduced density matrices, which are used throughout the work to refer
to correlated wave functions, follows in Section \ref{ss:rdm}.

Finally, we describe the method by which we couple the classical perturbing field to the 
quantum-mechanical correlated wave function. 
By applying perturbation theory we derive general expressions for tensors which describe 
the molecular response to an EMF. In the exact theory, these tensors turn out to be functions of the 
excited electronic states of the system; however, an approach to generalize 
these expressions to approximate theories such as truncated coupled cluster is discussed, which also 
avoids the costly evaluation of excited-state wave functions. For dynamic EMF, these properties may be 
described as a function of frequency, so the most common approach is to derive the working equations
using a Fourier transform of expressions obtained using the time-dependent Schr\"odinger equation 
(more generally known as response theory); however, these expressions may also be evaluated explicitly
in the time-domain. In Section \ref{se:res} we focus on the frequency-domain formulation, while
[PAPER 3] explores the recently-revived prospect of explicit time-propagation, as well as a 
technique to make this process cheaper through a well-established concept known as 
``local correlation''. 

\chapter{Theoretical background} \label{ch:theory}
\section{Ground state electronic structure theory} \label{se:est}
Electron correlation is defined in the present work to mean any additional
effect beyond what is included in the ground-state Hartree-Fock (HF)
wave function $\ket{0}$. In ground-state electronic structure theory,
this is typically taken to be an anti-symmetrized, single-determinant wave
function, generally composed of linear combinations of atomic Gaussian
basis functions. In solving the HF equations, the electronic energy
is variationally minimized, under the constraint that the orbital basis
functions of $\ket{0}$ remain orthonormal.\cite{Szabo1996} The resulting
orbitals will be referred to as the canonical molecular orbital (MO) basis.

This section will briefly describe a selection of methods for recovering
the correlation energy which are pertinent to this work. Correlation is
usually included by considering the effects of substituted determinants in
which electrons in ground-state occupied orbitals are substituted (often
referred to as ``excited'') into higher-energy unoccupied (or virtual)
orbitals. These occupied and virtual orbitals are often chosen to be the
canonical MOs of the reference wave function, which in turn are linear
combinations of atomic orbital basis functions. However, as discussed in
[PAPER 3], some schemes apply a different approach -- namely, localization
(and potentially truncation) of the canonical MOs prior to computing the
correlation energy. Therefore, this chapter will focus on \textit{general}
expressions, without assuming canonical or orthogonal orbital spaces
except where mentioned. Einstein summation notation (implied summation
over repeated indices) will be used throughout this chapter, except in
cases where an explicit sum is instructive.

%\subsection{Orbital invariant MP2} \label{ss:mp2}
\subsection{Many-Body Perturbation Theory} \label{ss:mp2}
Correlation energy corrections to the HF energy may be obtained through many-body perturbation theory (MBPT).
%By partitioning the Hamiltonian and expanding the energy and wave function in orders of a perturbation, contributions to the correlation energy are obtained order-by-order. Specifying the Hylleraas partitioning of the normal-ordered electronic Hamiltonian,
We may begin with either the time-dependent or time-independent Schr\"odinger equation --
notation for time-dependence will be suppressed here, 
but the same rules apply as mentioned in Section $\ref{se:res}$.
By partitioning the electronic Hamiltonian $\hat{H}$ into a zeroth-order (HF) term $\hat{H}^{(0)}$ and a perturbation $\hat{H}^{(1)}$,
which is first-order in an undetermined coefficient $\lambda$,
\begin{equation} \label{eq:H}
%    \hat{H}_N = E^{(0)} + \hat{F}_N + \hat{W}_N = \hat{H}^{(0)} + \hat{H}^{(1)} 
    \hat{H} = \hat{H}^{(0)} + \lambda\hat{H}^{(1)} 
\end{equation}
and expanding the electronic energy $E$ and wave function $\ket{\Psi}$ similarly in orders of the perturbation (denoted by superscript), 
\begin{subequations} \label{eq:pert}
    \begin{equation} \label{eq:pert_psi}
        \ket{\Psi} = \sum_{i=0}^{\infty}\lambda^{i}\ket{\Psi^{(i)}}
    \end{equation}
    \begin{equation} \label{eq:pert_e}
        E = \sum_{i=0}^{\infty}\lambda^{i}E^{(i)},
    \end{equation}
\end{subequations}
contributions to the correlation energy are obtained order-by-order. $\ket{\Psi^{(0)}}$ is taken to be $\ket{0}$, 
and $\lambda$ to be one.
Implicit here is the assumption that the perturbation is small relative to the reference -- \textit{i.e.}, 
the correlation energy is several orders of magnitude smaller than the HF energy.
%Specifying the Hylleraas partitioning of the normal-ordered electronic Hamiltonian,
%where $E^{(0)}$ is the Hartree-Fock energy and $\hat{F}_N$ and $\hat{W}_N$ are the normal-ordered Fock 
%operator and fluctuation potential, respectively, the first non-zero contribution is the Hylleraas 
%functional form of the second-order energy,

For a perturbation containing correlation operators, inserting
Eqs.~(\ref{eq:H}) and (\ref{eq:pert}) into the time-independent Schr\"odinger
equation and truncating at the second order in $\lambda$ yields the first
non-zero contribution to the energy:\cite{Szabo1996} \begin{equation}
\label{eq:hyll}
    E^{(2)} = \bra{0}\hat{H}^{(1)}\ket{\Psi^{(1)}}.
\end{equation} Second-order MBPT is equivalent to the minimization
of Eq.~(\ref{eq:hyll}). Using the Hylleraas partitioning of the
Hamiltonian,\cite{Pulay1986b} \begin{subequations}
    \begin{equation}
        \hat{H}^{(0)} = E^{(0)} + \hat{F}
    \end{equation} \begin{equation}
        \hat{H}^{(1)} = \hat{W}
    \end{equation}
\end{subequations} where $\hat{F}$ and $\hat{W}$ are the one- and
two-particle components of $\hat{H}^{(0)}$, respectively, we may solve Eq.~(\ref{eq:hyll}).
This is achieved
through minimization of a residual expression defined by the first-order
wave function equation, 
\begin{equation} \label{eq:mp2_res}
    \hat{W}\ket{0} + \hat{F}\ket{\Psi^{(1)}} = 0.
\end{equation} 

To generate programmable expressions, 
we may choose to represent our first-order perturbed wave function as a linear combination 
of substituted HF determinants with coefficients $t^{ab}_{ij}$ 
\begin{equation}
    \ket{\Psi^{(1)}} = \sum_{ijab}\frac{1}{4}t^{ab}_{ij}\ket{0}.
\end{equation}
Since the Hamiltonian does not connect
$\ket{0}$ to singly-substituted determinants due to the Brillouin
condition,\cite{Szabo1996} it follows that only double substitutions are
present in $\ket{\Psi^{(1)}}$. Projection of Eq.~(\ref{eq:mp2_res}) onto a
basis of doubly-substituted determinants yields programmable expressions
for $t^{ab}_{ij}$.
In the limit of canonical HF orbitals, Eq.~(\ref{eq:hyll}) reduces to the more commonly known second-order M{\o}ller-Plesset perturbation theory (MP2)\cite{Moller1934,Bartlett1974a} expressions,
\begin{subequations}
    \begin{equation} \label{eq:E2}
        %E^{(2)} = \frac{1}{4}t^{ab}_{ij}g^{ab}_{ij}
        E^{(2)} = \frac{1}{4}t^{ab}_{ij}\bra{ij}\ket{ab}
    \end{equation}
    \begin{equation} \label{eq:T2_mp2}
        %t^{ab}_{ij} = - \frac{g^{ab}_{ij}}{\epsilon_a + \epsilon_b - \epsilon_i - \epsilon_j} 
        t^{ab}_{ij} = - \frac{\bra{ij}\ket{ab}}{\epsilon_a + \epsilon_b - \epsilon_i - \epsilon_j} 
    \end{equation}
\end{subequations}
%where $g^{ab}_{ij}$ is the antisymmetryized two-electron integral $\bra{ij}\ket{ab}$ 
where $\bra{ij}\ket{ab}$ is the antisymmetrized two-electron integral between occupied orbitals $\{i,j\}$ and 
virtual orbitals $\{a,b\}$,
and $\epsilon_p$ is the $p$th diagonal element of the Fock matrix. 
This requires only one $\mathcal{O}(N^5)$ step (where $N$ is a measure of system size) in the rotation of the two-electron atomic orbital integrals into the MO basis followed by a one-step energy calculation. 
However, commonplace localizations of the occupied space
\cite{Pulay1986a,Surjan1989,Boughton1993}
require instead the iterative evaluation of 
Eq.~(\ref{eq:mp2_res}). Furthermore, reduced nonorthogonal virtual spaces may be employed in the 
projection basis (as will be explored in [PAPER 3]),\cite{Werner2006} which necessitates the inclusion of additional overlap 
matrices between nonorthogonal orbital spaces in these equations. While this increases the cost by adding 
iterations and additional expressions to evaluate, these additional costs are heavily outweighed by the 
computational savings due to the reduction of $N$ in large systems.

\subsection{Coupled Cluster} \label{ss:cc}
The coupled-cluster family of electronic structure methods
\cite{Sinanoglu1964,Cizek1966,Cizek1969,Crawford2000}
employs an exponentiated cluster operator $\hat{T}$ to include the effects of substituted determinants,
\begin{equation} \label{eq:cc_TISE}
    \hat{H}e^{\hat{T}}\ket{0} = Ee^{\hat{T}}\ket{0}
\end{equation}
with
\begin{subequations}
    \begin{equation} \label{eq:T}
        \hat{T} = \sum_{n=1}^M\hat{T}_n
    \end{equation}
    \begin{equation} \label{eq:T_n}
        \hat{T}_n = \left(\frac{1}{n!}\right)^2 t_{ij\ldots}^{ab\ldots}a_a^{\dagger}a_b^{\dagger}\ldots a_j a_i
    \end{equation}
\end{subequations}
where $M$ is the number of electrons in the system, and $a_p$ and $a_p^\dagger$ are the standard second-quantized 
excitation and de-excitation operators, respectively, for a general orbital $p$.
Rather than perturbative expansion, the cluster operator is expanded in orders of substitutions 
(singles, doubles, \textit{etc}. in Eq.~(\ref{eq:T})), then truncated to a given substitution 
level to produce tractable equations. Additional simplification is achieved by using the 
similarity-transformed Hamiltonian
\begin{equation} \label{eq:Hbar}
    \bar{H} = e^{-\hat{T}}\hat{H}e^{\hat{T}}
\end{equation}
which has the same eigenspectrum as $\hat{H}$, but naturally truncates at four nested commutators in a Campbell-Baker-Hausdorff expansion
\begin{equation} \label{eq:cbh}
    \bar{H} = \hat{H} + [\hat{H}, \hat{T}] + \frac{1}{2!}[[\hat{H}, \hat{T}], \hat{T}] + \frac{1}{3!}[[[\hat{H}, \hat{T}], \hat{T}], \hat{T}] + \frac{1}{4!}[[[[\hat{H}, \hat{T}], \hat{T}], \hat{T}], \hat{T}].
\end{equation}
Projecting Eq.~(\ref{eq:cc_TISE}) on the left by $\bra{0}e^{-\hat{T}}$ yields the coupled cluster 
energy expression and, as in Section \ref{ss:mp2}, expressions for the amplitudes 
$t_{ij\ldots}^{ab\ldots}$ are obtained by projection onto substituted determinants $\bra{\mu}$:
\begin{subequations}
    \begin{equation} \label{eq:cc_E}
        \bra{0}\bar{H}\ket{0} = E_{CC}
    \end{equation}
    \begin{equation} \label{eq:cc_res}
        \bra{\mu}\bar{H}\ket{0} = 0.
    \end{equation}
\end{subequations}
Here Eq.~(\ref{eq:cc_E}) follows from the normalization condition of $\ket{0}$, 
and Eq.~(\ref{eq:cc_res}) follows from the orthogonality condition. 
In the case of localized orbitals, such as those used in [PAPER 3], additional terms 
involving the overlap of nonorthogonal orbitals will appear in the final expressions. 

Untruncated, Eqs.~(\ref{eq:cc_E}) and (\ref{eq:cc_res}) produce exact
solutions to the time-independent Schr\"odinger equation for the
electronic Hamiltonian in a given basis set. Perhaps the most common
truncation of Eq.~(\ref{eq:T}) is to single and double substitutions,
defining the CCSD method. The computational cost of this method scales as
$\mathcal{O}(N^6)$. In terms of MBPT, CCSD may be viewed as correct through
infinite order, but only within the space of single and double substitutions.
A perturbative correction for triples produces the ``gold standard'' in
quantum chemistry, the CCSD(T) method, \cite{Raghavachari1989,Bartlett2005}
which scales as $\mathcal{O}(N^7)$. Aside from these, additional truncation
schemes exist for a variety of purposes. For example, the CCD method
includes only double substitutions, but the complexity of the resulting
equations is far simpler. Another method, CC2,\cite{Christiansen1995}
is tuned for the calculation of molecular response properties, such as
dipoles and polarizabilities. This method truncates the doubles expression
at first order by considering the $\hat{T}_2$ operator as first order. The
full singles expressions from CCSD are retained by considering $\hat{T}_1$
to be first order, and the one- and two-electron terms of the Hamiltonian are
still considered to be zeroth and first order, respectively, as in Section
\ref{ss:mp2}. Solving the resulting iterative amplitude expressions for CC2
scales as $\mathcal{O}(N^5)$ but, unlike the similarly-scaling MP2 method,
includes the effects of singles, which have been shown to be crucial for
computing accurate molecular properties.
\cite{Christiansen1995,Koch1997}
It is important to note that while the studies in [PAPER 2] and [PAPER 3]
utilize CCSD, many other flavors of CC may be used as drop-in replacements,
both in theory and in the implementation of their codes.

\subsection{Reduced Density Matrices} \label{ss:rdm} A general expression
for the ground-state energy of an arbitrary wave function $\ket{\Psi}$
prepared in a basis $\{p,q,r,s\}$ can be written in terms of one- and
two-particle reduced density matrices (1-RDM $D$ and 2-RDM $\Gamma$):
\cite{Harris1992,Trucks1988}
\begin{equation} \label{eq:pdm_e}
    \begin{aligned} E & = \bra{\Psi}\hat{H}\ket{\Psi} \\
      & = D_{pq}h_{p}^{q} + \Gamma_{pqrs}\bra{pq}\ket{rs}
    \end{aligned}
\end{equation} where $h_{p}^{q}$ are the one-electron integrals of the
electronic Hamiltonian, and $D$ and $\Gamma$ are defined according to the
form of $\ket{\Psi}$ and the excitation operators of the second quantized
Hamiltonian: \begin{subequations} \begin{equation} \label{eq:opdm}
    D_{pq} = \bra{\Psi}a_p^{\dagger}a_q\ket{\Psi}
\end{equation} \begin{equation} \label{eq:tpdm}
    \Gamma_{pqrs} = \bra{\Psi}a_p^{\dagger}a_q^{\dagger}a_sa_r\ket{\Psi}.
\end{equation} \end{subequations} 
Clearly, this requires the
left-hand wave function $\bra{\Psi}$. For the electronic Hamiltonian,
which is Hermitian, the left- and right-hand wave functions are identical,
as is the case for MP2. However, this Hermiticity is destroyed for standard
CC methods due to the similarity transformation of the Hamiltonian in
Eq.~(\ref{eq:Hbar}).\cite{Crawford2000} It is therefore necessary to solve for the CC left-hand
wave function, \begin{equation} \label{eq:cc_lwfn}
    \bra{\mathcal{L}} = \bra{\Psi}\hat{\mathcal{L}}
\end{equation} where $\hat{L}$ is a left-hand cluster
operator analogous to Eq.~(\ref{eq:T_n}), with analogous amplitudes
$\lambda_{ij\ldots}^{ab\ldots}$. This additional set of coupled
equations can be solved in the same manner as the right-hand amplitude
expressions. We may then rewrite the coupled cluster energy expression
using Eq.~(\ref{eq:cc_TISE}): \begin{equation} \label{eq:cc_energy_total}
    \bra{\Psi}\hat{\mathcal{L}}\bar{H}\ket{\Psi} = E.
\end{equation}

As will be explored in Section \ref{se:res}, molecular properties may be expressed as 
derivatives of the electronic energy.\cite{Helgaker2012} Given the left- and right-hand wave functions 
and assuming the Hellmann-Feynman theorem\cite{Hellmann1937,Feynman1939} holds 
(and thus $\ket{\Psi}$ carries no dependence 
on external perturbations), the derivative of the energy with respect to an arbitrary 
perturbation operator 
$\hat\Omega$ can be conveniently expressed
\begin{equation} \label{eq:dE}
    \begin{aligned}
    \frac{\partial E}{\partial \hat\Omega} &= \bra{\Psi}\frac{\partial \hat{H}}{\partial \hat\Omega}\ket{\Psi}\\
                                  &= D_{pq}\frac{\partial h_{pq}}{\partial \hat\Omega} + \Gamma_{pqrs}\frac{\partial g_{pqrs}}{\partial \hat\Omega}
    \end{aligned}
\end{equation}
%% NOTE: a lot of notation to clean up here
without further specification of the wave function or Hamiltonian. 
Thus, time-independent first-order properties may be expressed without 
differentiation of the RDMs. This was a driving force behind the study in 
[PAPER 1], and is important for the machine-learning methods discussed in [PAPER 2]. The cases of 
higher-order and time-dependent properties, which are the
focus of [PAPER 1] and [PAPER 3], will be handled in the following section. 

%\input{theory/loc.tex}
\section{Response theory} \label{se:res}
Response theory combines adiabatic perturbation theory with a Fourier transform of the 
time-dependent equations into the frequency domain\cite{Pederson2020}. The prediction of the response 
of molecular systems to an electromagnetic field (EMF) has important applications in organic synthesis and
characterization. To explore these effects, we may relate properties to a perturbative expansion of a 
total electric or magnetic moment, from which response tensors related to a host of properties may be defined. 
These response tensors generally take the form of products of ground- and excited-state transition moments of 
the electric or magnetic dipole operator. In this chapter we will summarize the derivation of these 
response tensors and their relationships to field-induced molecular phenomena. We then present a method 
for obtaining these tensors through Fourier transform of a time-dependent ``Quasi-energy'' which is 
rigorously defined in the context of electronic structure theory, providing working equations for 
predicting molecular responses to EMF using approximate wave function-based theory.

\subsection{Exact Response} \label{ss:exact}
Under classical electrodynamics, the potential energy $V$ of a system of charges in a static electric field 
may be written in a multipole expansion:

\begin{equation} \label{eq:pot_V}
V = q(\phi)_0 - \mu_\alpha(E_\alpha)_0 - \frac{1}{3}\Theta_{\alpha\beta}(E_{\alpha\beta})_0 + \ldots
\end{equation}
with the scalar potential $\phi$, the cartesian component $\alpha$ of the electric field $E_\alpha$, and the electric field gradient with respect to two cartesian coordinates $E_{\alpha\beta}$. Here we have defined the multipoles as the electric monopole or charge $q$, the electric dipole $\mu_\alpha$, and the electric quadrupole $\Theta_{\alpha\beta}$. The subscript $0$ indicates that the potential, field, or gradient is taken at the system origin.

We may also expand the potential in a Taylor series about the zero-field potential:
\begin{equation} \label{eq:tay_V}
    V[E_0] = V_0 + (E_\alpha)_0\frac{\partial V}{\partial (E_\alpha)_0} + \frac{1}{2}(E_\alpha)_0(E_\beta)_0\frac{\partial V^2}{\partial (E_\alpha) \partial (E_\beta)} + \ldots
\end{equation}

From Eq.~(\ref{eq:pot_V}), we see that a cartesian component of the electric dipole may be written as the derivative of the potential with respect to the field, 
\begin{equation} \label{eq:mu_part}
    \mu_\alpha = -\frac{\partial V}{\partial (E_\alpha)_0}.
\end{equation}
By comparing Eqs.~(\ref{eq:tay_V}) and (\ref{eq:mu_part}), we see that the total electric dipole may be re-written 
\begin{equation} \label{eq:mu_tot}
\mu_\alpha = \mu_{0\alpha} + \alpha_{\alpha\beta}(E_\beta)_0 + \frac{1}{2}\beta_{\alpha\beta\gamma}(E_\beta)_0(E_\gamma)_0 + \ldots
\end{equation}
where we have defined the response tensors for the permanent (field-independent) dipole moment $\mu_{0\alpha}$, electric polarizability $\alpha_{\alpha\beta}$, first electric hyperpolarizability $\beta_{\alpha\beta\gamma}$, and higher order terms as higher-order derivatives of the potential, \textit{viz.}
\begin{subequations} \label{eq:derivs}
    \begin{equation} \label{eq:mu_0}
    \mu_{0\alpha} = -\frac{\partial V}{\partial (E_\alpha)_0}
    \end{equation}

    \begin{equation} \label{eq:alpha}
    \alpha_{\alpha\beta} = -\frac{\partial^2 V}{\partial (E_\alpha)_0 \partial (E_\beta)_0}
    \end{equation}

    \begin{equation} \label{eq:beta}
    \beta_{\alpha\beta\gamma} = -\frac{\partial^3 V}{\partial (E_\alpha)_0 \partial (E_\beta)_0 \partial (E_\gamma)_0}.
    \end{equation}
\end{subequations}
Property tensors for magnetic and mixed electric-magnetic properties are defined in an analogous manner. 

Quantum mechanical expressions for the property tensors may be obtained through perturbation theory. 
For static-field properties, this process is straightforward, and expressions similar to those found for
correlation energy corrections in ~Eq.({\ref{eq:E2}) arise in the perturbative expansion of the time-independent Schr\"odinger
equation, e.g., for the static polarizability tensor:
\begin{equation} \label{eq:static_pol}
\alpha_{\alpha\beta} = -2\sum_{j\neq n}\frac{\bra{n}\mu_\alpha\ket{j}\bra{j}\mu_\beta\ket{n}}{V_n - V_j} 
\end{equation}
where the sum runs over excited states $j$ and $n$.

Of more practical application is the prediction of dynamic-field properties, such as dynamic polarizabilities, 
optical rotation, and circular dichroism (which are the primary targets of [PAPER1] and [PAPER3]).
For these and other time-dependent response properties, some description of the time evolution of the wave function is
required. 
These properties may be solved directly from the time-dependent analogue of ~Eq.(\ref{eq:mu_tot}). 
To compute the dynamic dipole, one must compute the dipole moment expectation value of a system in the presence of an explicitly
time-dependent electric field.
Fourier transforming the dynamic dipole into the frequency domain would produce broadband spectra.
However, explicit time propagation of the wave function is exceedingly expensive. This is the topic of chapters [TDCC] and [PAPER 3]. 
For now, we will consider an alternative approach which computes properties at individual frequencies, 
but does not require explicit propagation of the wave function. 

Beginning from the time-dependent Schr\"odinger equation,
\begin{equation} \label{eq:tdse}
    \hat{H}(t)\ket{\Psi(t)} = i\hbar\partt{}\ket{\Psi(t)}
\end{equation}
we will, without loss of generality, assume a time-dependent wave function expansion of the form
\begin{subequations} \label{eq:td_wfn}
    \begin{equation}
        \ket{\Psi(t)} = d_n(t)\ket{\psi(t)}
    \end{equation}
    \begin{equation}
        \ket{\psi(t)} = e^{-iE_nt/\hbar}\ket{n}
    \end{equation}
\end{subequations}
where $\ket{\psi(t)}$ is an approximate wave function consisting of a set of known, time-independent functions $\ket{n}$
and an exponentiated phase. Expressing the Hamiltonian as in ~Eq{(\ref{eq:H}) as a sum of the time-independent molecular
Hamiltonian $\hat{H}^{(0)}$ with a time-dependent perturbation $\hat{V}(t)$
(generally taken to be an EMF) and inserting this and the wave function expansion into ~Eq.({\ref{eq:tdse}) yields
\begin{equation} \label{eq:sub_tdse}
    (\hat{H}^{(0)} + \hat{V}(t))d_n(t)e^{-iE_nt/\hbar}\ket{n} = i\hbar \partt{}[d_n(t)e^{-iE_nt/\hbar}]\ket{n}.
\end{equation}
It should be noted that, in the absence of a perturbation (time $t = 0$), this reduces to the time-independent Schr\"odinger equation ($d_n(0) = 1$). Using the chain rule, ~Eq.({\ref{eq:sub_tdse}}) becomes  
\begin{equation} \label{eq:chain_tdse}
    (\hat{H}^{(0)} + \hat{V}(t))d_n(t)e^{-iE_nt/\hbar}\ket{n} = [E_n d_n(t) + i\hbar \partt{d_n(t)}]e^{-iE_nt/\hbar}\ket{n}.
\end{equation}
If we assume we have \textit{exact} eigenstates $\ket{n}$ of the molecular Hamiltonian, \textit{i.e.} exact ground and excited state wave functions, then the second terms of both sides of ~Eq.(\ref{eq:chain_tdse}) cancel. Projecting on the left by 
excited states $\bra{m}$ (which contains only one state nonorthogonal to $\ket{n}$ and thus collapses the sum on the 
right-hand side) and rearranging, we arrive at the differential equation
\begin{equation} \label{eq:diffeq}
    \partt{d_m(t)} = \frac{1}{i\hbar}\bra{m}\hat{V}(t)\ket{n}d_n(t)e^{i\omega_{mn} t}
\end{equation}
which describes the time evolution of the wave function coefficient $d_m(t)$. We have also introduced the angular 
frequency $\omega_{mn} = (E_m - E_n)/\hbar$. 

Integration of ~Eq.(\ref{eq:diffeq}) from time $t' = 0$ to $t' = t$ yields a recursive equation for the 
wave function parameter $d_m(t)$, due to the presence of $d_n(t)$ on the right-hand side. 
However, as a consequence of our assumption that the system begins in its unperturbed ground state, all
coefficients at $t' = 0$ reduce to $\delta_{m,0}$. To linear order in $\hat{V}$, ~Eq.(\ref{eq:diffeq}) may be written
\begin{equation} \label{eq:irt}
    d_m(t') = \frac{1}{i\hbar}\int_0^t\bra{m}\hat{V}(t^{'})\ket{0}e^{i\omega_{m0}t^{'}}dt^{'}.
\end{equation}
%\begin{equation} \label{eq:irt}
%    \begin{aligned}
%    d_m(t') = 1 &+ \frac{1}{i\hbar}\int_0^t\bra{m}\hat{V}(t^{'})\ket{n}d_n(t^{'})e^{-i\omega_{mn}t^{'}}dt^{'} \\
%            = 1 &+ \frac{1}{i\hbar}\int_0^t\bra{m}\hat{V}(t^{'})\ket{n}e^{-i\omega_{mn}t^{'}}dt^{'} \\
%                &- \frac{1}{\hbar}\int_0^t\int_0^{t^{'}}\bra{m}\hat{V}(t^{'})\ket{l}\bra{l}\hat{V}(t^{''})\ket{n}e^{-i\omega_{ml}t^{'}}e^{-i\omega_{ln}t^{''}}dt^{''} \\
%                &+ \ldots
%    \end{aligned}
%\end{equation}

To continue, we choose our perturbation to be a simple, time-dependent field at a given frequency $\omega$ 
\begin{equation} \label{eq:field}
\hat{V}(t) = \hat{v}^\omega_\alpha F^\omega_\alpha e^{-i\omega t}
\end{equation}
with field strength $F^\omega_\alpha$ and arbitrary field operator $\hat{v}^\omega_\alpha$, such as the electric or magnetic dipole operator. 
Note that the implicit summation runs over both $\omega$ and $\alpha$, and the sum in ~Eq.(\ref{eq:irt}) runs over excited states $\ket{m}$.
Finally, inserting ~Eq.(\ref{eq:field}) into ~Eq.(\ref{eq:irt}) and integrating, we arrive at the final expression for the
wave function parameters
\begin{equation} \label{eq:sot}
    \begin{aligned}
        d_m(t) = \frac{1}{\hbar}\frac{\bra{m}\hat{v}^\omega_\alpha\ket{0}e^{i(\omega_{m0}-\omega)t}}{\hbar(\omega_{m0}-\omega)}F^{\omega}_{\alpha}.
%        d_m(t') = 1 &+ \frac{-1}{\hbar}\frac{\bra{m}\hat{v}^\omega_\alpha\ket{n}}{\omega_{mn} - \omega_1}e^{-i\omega_{mn}t}e^{-i\omega_1t} \\
%                &- \frac{1}{\hbar}\frac{\bra{m}\hat{v}^{\omega1}_\alpha\ket{l}\bra{l}\hat{v}^{\omega2}_\beta\ket{n}}{(\omega_{ml}-\omega_1)(\omega_{ln}-\omega_2)}e^{-i(\omega_{ml}-\omega_{ln})t}e^{-i(\omega_1-\omega_2)t} \\
%                &+ \ldots
    \end{aligned}
\end{equation}
 
To arrive at a dipole expression in the form of ~Eq.(\ref{eq:mu_tot}), we may compute the expectation 
value of the electric dipole operator $\hat{\mu}$. After rearrangement, we arrive at
\begin{equation} \label{eq:sos}
    \begin{aligned}
    \bra{\Psi(t)}\hat\mu\ket{\Psi(t)} &= \bra{0}\hat\mu\ket{0} \\ 
    &- \sum_{m\neq0}[\frac{\bra{m}\hat\mu\ket{0}\bra{0}\hat{v}^\omega_\alpha\ket{m}}{\hbar(\omega_{mo}+\omega)}
        + \frac{\bra{0}\hat\mu\ket{m}\bra{m}\hat{v}^\omega_\alpha\ket{0}}{\hbar(\omega_{mo}-\omega)}]
    e^{-i\omega t}F^\omega_\alpha
    \end{aligned}
\end{equation}
where we have truncated at the linear response function of $\hat{\mu}$ perturbed by $\hat{V}(t)$. This is known as the \textit{exact} linear response function. In general, 
response functions can be identified by the expansion of an operator $\hat\Omega$:
\begin{equation}
    \begin{aligned}
    \bra{\Psi(t)}\hat\Omega\ket{\Psi(t)} &= \bra{0}\hat\Omega\ket{0} \\ 
    &+ \linresp{\hat\Omega}{\hat{v}^{\omega_1}_\alpha}e^{-i\omega_1 t}F^{\omega_1}_\alpha \\ 
    &+ \quadresp{\hat\Omega}{\hat{v}^{\omega_1}_\alpha}{\hat{v}^{\omega_2}_\beta}e^{-i(\omega_1+\omega_2)t}F^{\omega_1}_\alpha F^{\omega_2}_\beta \\
    &+ \ldots
    \end{aligned}
\end{equation}
where we have introduced notation for the linear $\linresp{\hat\Omega}{\hat{v}^{\omega_1}_\alpha}$, 
quadratic $\quadresp{\hat\Omega}{\hat{v}^{\omega_1}_\alpha}{\hat{v}^{\omega_2}_\beta}$, and higher-order response functions.

\subsection{Approximate Response} \label{ss:apprx}
Sum-over-states expressions like those in Eq.~(\ref{eq:sos}) can be solved by computing the excited state wave functions and 
summing their individual contributions to the property. This poses a unique challenge in that the response tensor is not a product 
of only one targeted excited state, but of all possible excited states. 
While it is expected many high-energy states will have negligible contributions, the number of states 
required to accurately model these properties makes this approach prohibitively expensive for excited-
state extensions to ground-state  methods, such as equation-of-motion (EOM) CC. We can avoid this 
problem by building response equations based on approximate ground-state wave functions, such as ~Eq.(\ref{eq:cc_TISE}).
This eliminates the ability to compute excited-state wave functions, but allows us to replace the sum-over-states 
expression with a set of coupled linear equations which are far more computationally tractable.

Beginning with a time-dependent wave function $\ket{\Psi(t)}$, we may separate this into two time-dependent pieces - the exponential of a phase factor $\phi(t)$, and a phase-isolated wave function $\ket{\piw}$:
\begin{equation} \label{total_psi}
    \ket{\Psi(t)} = e^{-i\phi(t)}\ket{\piw}.
\end{equation}
We may require that the phase of the projection of $\ket{\piw}$ onto the ground-state wave function be 
zero - in other words, in the limit of zero time-dependent perturbation, $\ket{\piw}$ reduces to the
ground state wave function. This is analogous to our approach in ~Eq.(\ref{eq:td_wfn}), where our focus has
now shifted from the wave function parameters to the phase factor.
Inserting these definitions into the time-dependent Schr\"odinger equation,
we arrive at Eq.~(\ref{eq:quasi_schro}):
\begin{equation} \label{eq:quasi_schro}
    (\hat{H} - i\hbar \partt{})\ket{\piw} = Q(t)\ket{\piw}
\end{equation}
where we have defined the time-dependent quasi-energy as the reduced Planck constant times the derivative
of the phase factor,
\begin{equation} \label{eq:quasi_e}
    Q(t) = \hbar\partt{\phi(t)}.
\end{equation}
It is important to note that, just as the phase-isolated wave function reduces to the ground state wave function in the absence of a perturbation, the quasi-energy also reduces to the ground state energy in this case. 

It can be shown that the quasi-energy is manifestly real, and both a time-dependent variational principle and Hellmann-Feynman theorem apply. To obtain a time-independent quantity to perturbatively expand, the quasi-energy is integrated over time to form the time-averaged quasi-energy $Q_T$. The variational and Hellmann-Feynman theorems can then be written
\begin{equation} \label{eq:var}
    \partial Q_T = 0
\end{equation}
and
\begin{equation} \label{eq:hf-theorem}
    \frac{dQ_T}{d\hat\Omega} = \frac{1}{T}\int_{t^{'}}^{t^{'}+T}\bra{\piw}\frac{\partial \hat{H}}{\partial\hat\Omega}\ket{\piw}dt^{'},
\end{equation}
respectively, for an arbitrary perturbation $\hat\Omega$ (usually taken to be an electromagnetic field).

In the preceding derivation, at no point was it assumed that we have \textit{exact} eigenfunctions of the time-independent 
Hamiltonian. Thus, the above equations are valid for approximate ground-state theories, such as CC. We may now define
our approximate response functions as we did in ~Eq.(\ref{eq:derivs}) as derivatives of this time-averaged Quasi-energy,
\begin{subequations}
    \begin{equation}
        \linresp{\hat\Omega}{\hat v^{\omega_1}_{\alpha}} = \frac{d^2Q_T}{d\hat\Omega dF^{\omega_1}_\alpha}
    \end{equation}
    \begin{equation}
        \quadresp{\hat\Omega}{\hat{v}^{\omega_1}_\alpha}{\hat{v}^{\omega_2}_\beta} = \frac{d^3Q_T}{d\hat\Omega dF^{\omega_1}_\alpha dF^{\omega_2}_\beta}
    \end{equation}
\end{subequations}
and working equations may be derived by inserting the specific approximate wave function ansatz.

The response functions of interest to the present work are the linear response functions between 
the electric dipole and an electric field
$\linresp{\hat\mu}{\hat{\mu}}$,
and the magnetic dipole in an electric field
$\linresp{\hat m}{\hat{\mu}}$. 
These are responsible for the dynamic electric polarizability and chiroptical response (optical
rotation and circular dichroism) respectively. The latter constitutes a long-standing challenge
for response theory, requiring mixed electric- and magnetic-field derivatives and minimal room
for error, but is also a 
prime candidate for comparisons to experiment. Some limitations of and alternatives to response
theory will be explored in chapters [PAPER 1] and [PAPER 3].


%comp.tex%

\section{Computational Details} \label{comp}
Second-order MBPT using a restricted closed-shell Hartree Fock reference wave function (i.e. M{\o}ller-Plesset perturbation theory, MP2)\cite{Møller1934,Bartlett1974a}, was used to generate both TATRs and DTRs for all systems. The TATR is generated from Eq.~(\ref{eq:mbtr}) using the highest 150 (by magnitude) amplitudes of double ``excitations'' 
$t_{ij}^{ab}$. 
For DTRs, the elements of the MP2 1-RDM are used. This choice is advantageous not just for comparison to Ref.~\citenum{Margraf2018}, but also due to the special form of the MP2 reduced density matrices\cite{Trucks1988}:

\begin{subequations} \label{eq:mp2_d}
    \begin{equation} \label{eq:mp2_opdm}
        D_{pq} = \bra{0}\hat T_2^{\dagger} a_p^{\dagger}a_q \hat T_2\ket{0}
    \end{equation}
    \begin{equation} \label{eq:mp2_tpdm}
        \Gamma_{pqrs} = 2\bra{0}a_p^{\dagger}a_q^{\dagger}a_sa_r \hat T_2\ket{0}.
    \end{equation}
\end{subequations}
where the ``excitation'' operators are defined as $\hat T_n = \left(\frac{1}{n!}\right)^2 t_{ij\ldots}^{ab\ldots}a_a^{\dagger}a_b^{\dagger}\ldots a_j a_i$, and we have again implied Einstein summation notation this time over occupied ${i,j,\ldots}$ and virtual ${a,b,\ldots}$ orbital spaces.
The form of Eqs.~(\ref{eq:mp2_opdm}) and~(\ref{eq:mp2_tpdm}) indicates that the reduced density matrices are fully defined by the doubles amplitudes and products thereof. In this light, it is reasonable to suggest that all of the information necessary to reproduce the wave function is coded into either of these and, to reduce the size of the representation, the 1-RDM should suffice for describing the MP2-level correlated wave function. Thus, any correlated one-electron property at the MP2 level can be described by this matrix.

Coupled cluster with single and double excitations (CCSD) was used to generate the high-level reference data, i.e. correlated energies and electronic dipole moments, for all systems. This method represents a marked improvement over MP2 in systems where correlation is strong, such as stretched diatomics, without incurring the additional cost of the gold-standard CCSD with perturbative triples method, CCSD(T). However, the CCSD(T) method could be used to provide higher quality targets (e.g. energies and dipoles) with no additional modifications --- as noted in section~\ref{algorithm}, the algorithm as presented is completely general. 

All electronic structure calculations were performed with the Psi4\cite{Parrish2017} electronic structure package. Molecular symmetry was used to maximize TATR performance with as few amplitudes as possible (except in section \ref{cutoffs}). All data (energies, dipoles, amplitudes, and densities) were harvested in full machine precision through either the Psi4-JSON interface or the PsiAPI infrastructure, with the exception of wave function amplitudes at the MP2 level, which were printed in the output file at ten decimal places. The def2-TZVP\cite{Weigend2005} basis was employed for diatomic calculations for comparison with Ref.~\citenum{Margraf2018}, and the aug-cc-pVDZ (aDZ) basis\cite{Dunning1989,Woon1994} was used for all remaining calculations. Density-fitted integrals were used in the construction of MP2 densities using the default auxiliary basis (the so-called ``RI'' basis sets for def2-TZVP\cite{Hattig2005} and aDZ\cite{Weigend2002}, obtained from the New Basis Set Exchange\cite{Pritchard2019}). 

As outlined above, the general procedure from Margraf and Reuter\cite{Margraf2018} was followed for machine-learning. 
The hyperparameters $\sigma_m$ and $\lambda$ were optimized on a uniform grid from $10^{-8}:10^{8}$ for all models unless otherwise specified, using the ``negative-mean-squared'' loss function. 
All regressions were performed using 20 training points, except in section \ref{cutoffs} where 12 training points were used to emphasize the difference between the two strategies considered. 
k-means clustering was repeated 30 times to account for stochastic deviations in the algorithm. 
Test sets consisting of $N/4$ points are held back before the clustering step.
The radial basis function kernel (sometimes called the Gaussian kernel), Eq.~(\ref{eq:rbf}), was used for all representations. 
Machine-learning algorithms were performed using Scikit-Learn (skl) \cite{Pedregosa2011} and the Machine-Learning Quantum Mechanics (MLQM) python package\cite{mlqm}, which generates a number of molecular representations, provides options for generating Psi4 input files and harvesting their results, and wraps some skl functions. 
Both of these codes are open-source and freely available on GitHub.

Geometries for diatomics were constructed by taking $N = 200$ uniform increments from 0.5-2.0 \AA . All other geometries were randomly selected from MD simulations ($N = 150$) carried out in the Gromacs\cite{Pronk2013} software package.
Simulations of (\textit{S})-methyloxirane and (\textit{R})-methylthiirane surrounded by a 5\AA\ box of water molecules were executed using an all-atom OPLS/AA forcefield\cite{Jorgensen1996} for the solute and the TIP3P model for water\cite{Jorgensen1983}. Each 5 ns trajectory was carried out in the NVT
ensemble, with the solute and solvent coupled separately to a temperature bath
at 300 K using a modified Berendsen thermostat and a coupling time of 0.1 ps. Geometries were selected from a set of 250 evenly-spaced snapshots along the trajectory.

Geometries, Psi4 input and output files, and their accompanying optimized (hyper)parameter values in JSON format can be found at the Virginia Tech Data Repository\cite{vtdata2020}   
to permit reproducibility of our results. Scripts for harvesting and manipulating data using MLQM are also included in the form of Jupyter Notebooks.

%results.tex

\section{Results and Discussion} \label{se:results} 
Here we present results from the first applications of local correlation to RTCC.
Results are examined
by the convergence of absorption and ECD spectra to the reference results
in Section \ref{ss:spectra}, followed by an analysis of the amplitude
dynamics in Section \ref{ss:amps}. 
%Section \ref{ss:ext} considers the
%effects of localization on orbital spatial extent between PAO and PNO
%virtual spaces, and contrasts these results with the amplitude data in
%an attempt to explain their performance. 
Section \ref{ss:alt} explores some potential solutions for building
appropriate virtual spaces for truncation, such as considerations
of orbital extent,
perturbation-aware virtual spaces\cite{Crawford2019,DCunha2021}, and 
including or focusing on the effects of the singles amplitudes. 
%Finally, we present an alternative solution dubbed ``Semi-static RTCC'' 
%(SS-RTCC) in Section \ref{ss:ssrtcc} in which the doubles amplitudes 
%($t_2$ and $\lambda_2$ in Eqs.~(\ref{eq:diff_t}) and (\ref{eq:diff_l}), 
%respectively) are frozen after the determination of the ground-state wave 
%function, and only the singles amplitudes are allowed to respond to the 
%incident perturbation. 

\subsection{Absorption and ECD Spectra} \label{ss:spectra}
\subsubsection{Absorption} \label{sss:abs}
Absorption spectra are obtained from the Fourier transform of Eq.~(\ref{eq:abs}).
Figure~\ref{fig:pno_abs} shows the normalized absorption spectrum obtained from
a reference propagation along with five PNO cutoffs. The average truncated virtual 
orbital spaces are from roughly 20\% to 90\% of the MO virtual space (see caption). 
\begin{figure} 
    \centering
    \includegraphics[scale=.6]{p3/figures/pno_abs.png}
    \caption{Reference and PNO absorption spectra for five cutoffs: 
    [$1\times 10^{-10}$, $1\times 10^{-9}$, $1\times 10^{-8}$, $1\times 10^{-7}$, 
    $2\times 10^{-6}$] corresponding to [$93\%$, $82\%$, $63\%$, $44\%$, $24\%$]
    of the MO virtual space, respectively.}
    \label{fig:pno_abs}
\end{figure}
%For all truncated PNO virtual spaces considered, the base peak appears to within 1.5 eV of the 
%reference. 
Overall, truncated PNO virtual spaces approximate the position of the base peak well,
with the smallest space predicting a base peak within 1.5 eV of the reference,
and the two largest spaces predict this peak to within 0.2 eV of the reference.
Convergence to the reference base peak occurs from the right, indicating
a lowering of excited state energies as the size of the virtual space increases. 
This trend can also be seen for the smaller peak near 10 eV. However, convergence of the shoulder 
peaks on either side of the base peak, indicated by the inset of 
Figure~\ref{fig:pno_abs}, is less predictable. Even the largest spaces considered
do not correctly predict the excitation energy, with no clear advantage to having
93\% of the virtual space as compared to just 83\% for predicting these peaks.
This trend continues into the higher-energy range of the spectrum, with the 
performance of each cutoff being nearly indistinguishable. 

Performance of the PAO space is shown in Figure~\ref{fig:pao_abs}. 
\begin{figure} 
    \centering
    \includegraphics[scale=.6]{p3/figures/pao_abs.png}
    \caption{Reference and PAO absorption spectra for five cutoffs: 
    [$1\times 10^{-4}$, $1\times 10^{-3}$, $1\times 10^{-2}$, $5\times 10^{-2}$, 
    $1\times 10^{-6}$] corresponding to [$95\%$,$86\%$,$63\%$,$46\%$,$23\%$]
    of the MO virtual space, respectively.}
    \label{fig:pao_abs}
\end{figure}
The largest truncated PAO virtual space, on average 95\% of the MO space, accurately 
predicts the excitation energies for each major peak below 17 eV. Particularly around 10 eV, 
this is noticeably improved performance relative to the largest PNO space tested, 
with only a 2\% difference in the average size of the virtual space. 
However, accuracy rapidly declines even at 86\% of the virtual space, where the base peak
position is already worse than what was predicted with a PNO space of just 63\% of the 
MO space. Performance continues to degrade as energy increases and the average size of 
the PAO space decreases. For the final two cutoffs, at averages of 46\% and 23\% of the
MO space, the base peaks are 3 eV or more away from the reference, and no peak is 
exhibited near 25 eV. These spaces also fail to predict the second largest peak, the 
excitation just below 10 eV. 

\subsubsection{ECD} \label{sss:ecd}
Overall, neither scheme produced adequate results upon truncation of the virtual space. 
This result is not entirely surprising -- in studies of local correlation applied to 
response theory by Crawford \textit{et al}., 
\cite{McAlexander2016,Kumar2017,Crawford2019,DCunha2021} 
traditional schemes proved inaccurate for another 
electric dipole--electric dipole property, the electric polarizability.
In terms of response theory,
the polarizability (and the refractive index) is related to the \textit{real} part 
of the electric dipole--electric dipole linear response tensor 
($\boldsymbol{\alpha}_{ij}$ in Eq.~(\ref{eq:mu_exp})), 
while absorption
is related to the \textit{imaginary} part. Indeed, all linear absorptive properties
such as absorption and CD 
are related to the imaginary component of a linear response tensor, while dispersive 
properties such as refractive index and 
circular birefringence (also known as optical rotation)
are related to the real component.
\cite{Barron2004,Norman2011}
To continue, we will look at another absorptive property which is related to the mixed
electric dipole--magnetic dipole linear response tensor -- ECD.

The ECD spectrum is obtained from the Fourier transform of Eq.~(\ref{eq:ecd}).
Being a bisignate, mixed-response property, ECD is a considerable computational
challenge, similar to its dispersive counterpart circular birefringence. 
Figure~\ref{fig:pno_ecd} shows the results for an ECD spectrum in the same PNO 
orbital spaces used in the previous section.
\begin{figure} 
    \centering
    \includegraphics[scale=.6]{p3/figures/pno_ecd.png}
    \caption{Reference and PNO ECD spectra for five cutoffs: 
    [$1\times 10^{-10}$, $1\times 10^{-9}$, $1\times 10^{-8}$, $1\times 10^{-7}$, 
    $2\times 10^{-6}$] corresponding to [$93\%$,$82\%$,$63\%$,$44\%$,$24\%$]
    of the MO virtual space, respectively.}
    \label{fig:pno_ecd}
\end{figure}
The dynamic response of the magnetic dipole to the electric field in this frequency 
range is considerably more complicated than that of the electric dipole. Below 60\% of
the MO space, virtually all distinguishing characteristics of the reference 
spectrum are unidentifiable. Further, at 82\%, the base peak appears to be a pair of 
peaks, more resembling the pair of peaks appearing just above 15 eV in the 
reference spectrum, with the major peak just below 15 eV being the second strongest.
At an average of 93\%, the overall \textit{shape} of the spectrum in the 10 eV to 
20 eV range more closely resembles that of the reference; however, the excitation
energies are, in some cases, even less accurate than those of smaller PNO spaces.
The trend of lowering excited state energies with increased virtual space seen in 
Section~\ref{sss:ecd} is no longer discernible. 

As in the case of absorption, the PAO basis is not noticeably more efficient at
approximating the full MO space than the PNO space. Figure~\ref{fig:pao_ecd} 
shows the results using the same truncated PAO spaces as in Section~$\ref{sss:abs}$.
\begin{figure} 
    \centering
    \includegraphics[scale=.6]{p3/figures/pao_ecd.png}
    \caption{Reference and PAO ECD spectra for five cutoffs: 
    [$1\times 10^{-4}$, $1\times 10^{-3}$, $1\times 10^{-2}$, $5\times 10^{-2}$, 
    $1\times 10^{-6}$] corresponding to [$95\%$,$86\%$,$63\%$,$46\%$,$23\%$]
    of the MO virtual space, respectively.}
    \label{fig:pao_ecd}
\end{figure}
The performance of the PAO basis near the base peak varies wildly with truncation,
as in the PNO case. In the low-frequency region, the PAO results are considerably 
worse -- see the two negative peaks between 10 eV and 15 eV. Curiously, the 
largest PAO spaces considered predict significant peaks above 25 eV that are not 
present in the reference, any of the PNO spaces tested, or the smaller PAO
spaces. This suggests a strong sensitivity of the response of the wave function
to the completeness threshold used for determining the occupied domains.
%fundamentally different electronic structure in the
%presence of the perturbing field, which is a direct result of the charge-based 
%completeness threshold used for determining the occupied domains.
%albeit at very large frequencies which are of little practical use.

\subsection{Amplitude Dynamics} \label{ss:amps}
As evidenced by the preceding data, the truncated PNO and PAO virtual spaces do not
efficiently model the wave function in the presence of a perturbing EMF. As noted in
Section~\ref{sss:abs}, these shortcomings are well-documented in the case of response
theory. However, a real-time formalism offers the opportunity to analyze the wave 
function in great detail over time, perhaps shedding light on \textit{where} and 
\textit{how} the locally correlated wave functions are deficient. The following
section will scrutinize the $t_\mu$ and $\lambda_\mu$ amplitudes of 
Eqs.~(\ref{eq:t_mu}) and (\ref{eq:l_mu}), respectively, in hopes of determining the 
important fluctuations in the wave function and whether these spaces sufficiently 
capture these changes.

Response to external perturbations by the CC amplitudes give rise to
dynamic energetics and properties. In the past, distributions of perturbed amplitudes
(relative to their ground-state counterparts) have been used to justify the 
difficulty in computing response functions with local correlation methods in the 
frequency domain. 
\cite{McAlexander2016,Crawford2019,DCunha2021} 
However, initial findings show that in RTCC, the relative distribution of amplitudes 
by magnitude is not significantly impacted.\cite{Crawford2019} 
Despite this, typical means of exploiting amplitude
sparsity have been shown to be inefficient by the preceding sections. 
First, to understand the response of 
the amplitudes to the external perturbation, we plot the 
change in the norm of the amplitude tensors relative to the ground-state amplitudes 
as a function of time in Figure~\ref{fig:norm}.
\begin{figure} 
    \centering
    \includegraphics[scale=.6]{p3/figures/amp_norm.png}
    \caption{Time-dependent change in the norm of the amplitude 
    tensors relative to the ground-state amplitudes.
    (Field and step parameters remain unchanged, and 
    the amplitude norm is taken at every 1 a.u.)}
    \label{fig:norm}
\end{figure}
Results for the untruncated PNO and PAO spaces are identical to those
for the MO space, as the unitary transformations resulting from untruncated localized virtual spaces
in Eqs.~(\ref{eq:rotate}) preserves the tensor norm.
Amplitude norms from propagations carried out in truncated PNO and PAO spaces
are nearly indistinguishable (see SI).

Figure~\ref{fig:norm} shows that the magnitude of the response by the wave function
is predominantly within the singles amplitudes $t_1$ and $\lambda_1$. This is 
consistent with the notion that singles are paramount for the computation of 
response properties.\cite{Christiansen1995,Koch1997} However, the form of Eq.~(\ref{eq:pair_D})
does not include any contributions by singles, due to being built from MP2-level
amplitudes where singles do not contribute until at least the second order 
in the wave function and fourth order in the energy. This suggests that even in schemes
which seek to include the EMF perturbation in the construction of the reduced virtual space,
such as PNO++,\cite{DCunha2021} response of the singles should be considered.

Aside from the matrix norm, we can also inspect the individual amplitudes to track
their evolution in time. The heat maps in Figure~\ref{fig:amps} show the difference 
in $t_1$ amplitudes, relative to the ground state, for three time steps selected 
from the first 100 a.u. of the simulation.
\begin{figure}
    \begin{subfigure}{.5\textwidth}
        \centering
        \includegraphics[scale=0.5]{p3/figures/MO_delta_t1_1.png}
        \caption{}
        \label{fig:MO_t1_1}
    \end{subfigure}%
    \begin{subfigure}{.5\textwidth}
        \centering
        \includegraphics[scale=0.5]{p3/figures/MO_delta_t1_50.png}
        \caption{}
        \label{fig:MO_t1_50}
    \end{subfigure}
    \begin{subfigure}{.5\textwidth}
        \centering
        \includegraphics[scale=0.5]{p3/figures/MO_delta_t1_100.png}
        \caption{}
        \label{fig:MO_t1_100}
    \end{subfigure}
    \caption{MO-basis $t_1$ amplitude deviations from $t = 0$ after (a) 1 a.u., (b) 50 a.u., and 
    (c) 100 a.u. of time propagation. Each row contains the same four occupied orbital indices
    and a subset of virtual indices as indicated by the x-axis labels.}
    \label{fig:amps}
\end{figure}
The amplitudes are ordered by the orbital energies of the associated MOs. 
The amplitudes which experience significant oscillations vary throughout the simulation,
though there are several discernible trends. First, most large amplitude deviations
are associated with all occupied orbitals simultaneously. This is due to the relatively small size of 
the system, with only four occupied orbitals, all of which are likely important in the
description of the ground- and excited-state wave functions. Secondly, 
at any given time during the propagation,
a large number of amplitudes have not significantly deviated from their ground state values.
This supports the notion
that relative sparsity is maintained within the amplitudes throughout the simulation,
but this sparsity is distributed differently throughout the amplitude tensors as
the wave function is propagated.  

A third trend is that amplitudes which respond strongly tend to be associated with 
low-energy virtual orbitals. Chemical intuition would suggest that energetically 
low-lying molecular orbitals will be the most involved in electronic excitations.
However, while amplitude responses are indeed larger for lower-energy virtual orbitals, 
smaller amplitude 
deviations in Figure~\ref{fig:amps} extend far into the virtual space. This explains
the difficulty of simply truncating with respect to orbital energy: the 
high-energy MOs are still important to the time-evolution of the wave function 
in the presence of an EMF. 

Figure~\ref{fig:pno_amps} shows the $t_1$ amplitudes for the same simulation,
rotated into the untruncated PNO basis using $Q_{ii}$ as defined in
Eq.~(\ref{eq:Q_pno}).
\begin{figure}
    \begin{subfigure}{.5\textwidth}
        \centering
        \includegraphics[scale=0.5]{p3/figures/PNO_delta_t1_1.png}
        \caption{}
        \label{fig:PNO_t1_1}
    \end{subfigure}%
    \begin{subfigure}{.5\textwidth}
        \centering
        \includegraphics[scale=0.5]{p3/figures/PNO_delta_t1_50.png}
        \caption{}
        \label{fig:PNO_t1_50}
    \end{subfigure}
    \begin{subfigure}{.5\textwidth}
        \centering
        \includegraphics[scale=0.5]{p3/figures/PNO_delta_t1_100.png}
        \caption{}
        \label{fig:PNO_t1_100}
    \end{subfigure}
    \caption{PNO-basis $t_1$ amplitude deviations from $t = 0$ after (a) 1 a.u., (b) 50 a.u., and 
    (c) 100 a.u. of time propagation. Each row contains the same four occupied orbital indices
    and a subset of virtual indices as indicated by the x-axis labels.}
    \label{fig:pno_amps}
\end{figure}
(It should be noted that, due to redundancy in the AO-based virtual 
spaces for each pair, PAO-basis amplitudes cannot be compared directly in 
this manner.)
It can be immediately seen that the amplitude deviations
are less sparse in the PNO basis after the application of the EMF. 
Many more amplitudes exhibit
perceivable differences, and strong deviations (magnitudes approaching 
$1\times 10^{-5}$) are no longer present. This is a clear demonstration
of the issue with truncating orbital spaces based on the present criterion ---
rather than exploiting sparsity, the amplitude tensors have become less sparse.
It may also suggest a recipe for building a more appropriate 
virtual space for truncation. 
In the following section, we propose some alternative schemes based on the 
literature and the results of this study.
%In the following section, we compare a selection
%of orbitals which correspond to strong amplitude deviations in the MO basis 
%(specifically virtuals 3, 4, 7, and 15)
%based on orbital spatial extent to determine if this may be a possible criterion 
%for truncation of the virtual space.

\subsection{Possible Alternatives} \label{ss:alt}
\begin{figure}
    \centering
    \includegraphics[scale=0.75]{p3/figures/extent.png}
    \caption{Virtual MO energy $\epsilon_a$ and the
    occupation number $n_a$ (plotted on a log scale)
    for unique PNO spaces $i_1$ and $i_2$ 
    versus orbital extent in arbitrary units.
    Virtual MOs 7 and 15 are denoted by a solid $\boldsymbol{+}$ 
    and $\boldsymbol{\times}$, respectively.
    The horizontal line denotes a PNO cutoff 
    of $1\times 10^{-7}$.}
    \label{fig:extent}
\end{figure}
Figure~\ref{fig:extent} shows the virtual MO energy $\epsilon_a$ and the 
PNO occupation number $n_a$ plotted against the orbital extent
$\langle r^2 \rangle$ in arbitrary units. 
In the PNO basis, a unique virtual space is prepared
for every occupied pair, resulting in 16 unique spaces for the four occupied
spatial orbitals $i$. However, for transforming
a single orbital index, we only require the diagonal rotation matrices,
\textit{i.e.}, $Q_{ii}$. There are four such spaces; however, by symmetry,
only two are unique. Both are included in Figure~\ref{fig:extent}. 

Truncation of the PNO space begins from the bottom of Figure~\ref{fig:extent}. 
At an occupation number cutoff of $1\times 10^{-7}$ (indicated by a horizontal
line), all orbitals below this line are neglected in the PNO space. Roughly
66\% of the virtual space lies in this region. From these data, it is clear
that even modest truncation of the virtual space neglects the diffuse 
regions of the wave function, which are important for excited-state
properties in systems with significantly delocalized characteristics,
such as systems containing Rydberg-type excitations.

%Table~\ref{ta:ext} reports the orbital extent $\langle r\rangle$ for virtual orbitals $a$
%corresponding to four of the strongest deviations in Figure~\ref{fig:amps}. 
%These orbitals are labeled 4, 7, 8, and 15, in
%both the MO and PNO basis. In the PNO basis, a unique virtual space is prepared
%for every occupied pair, resulting in 16 unique spaces for the four occupied
%spatial orbitals $i$. However, for transforming
%a single orbital index, we only require the diagonal rotation matrices,
%\textit{i.e.}, $Q_{ii}$. There are four such spaces; however, by symmetry, the domains of occupied
%orbitals 1 and 4 are the same, as are 2 and 3. Thus, orbital extents for only those
%two unique orbital spaces $i_1$ and $i_2$  are shown.
%\begin{table}
%    \centering
%    \begin{tabular}{|c|c|c|c|}
%        \hline
%%        $a$ &   MO  &   PNO   &        \\  
%        $a$ &  MO  &   \multicolumn{2}{c|}{PNO} \\  
%        \hline
%%        \multicolumn{2}{|c|}{}  & $i_1$   & $i_2$  \\ 
%            &       & $i_1$   & $i_2$  \\
%        \hline
%%         3  & 59.23 & 7.9     & -24.67 \\ 
%%        \hline
%         4  & 88.76 & -13.08  & -3.13  \\ 
%        \hline
%         7  & 66.00 & 0.73    & -0.07  \\ 
%        \hline
%         8  & 65.56 & -3.91   & -10.35 \\ 
%        \hline
%         15 & 37.39 & -15.11  &  1.32  \\
%        \hline
%    \end{tabular}
%    \caption{Orbital spatial extent of four selected virtual orbitals in the MO
%    and PNO spaces.}
%    \label{ta:ext}
%\end{table}
%A full table of virtual orbital extents can be found in the SI. 
%In the MO basis, the orbitals in Table~\ref{ta:ext} 
%correspond to the strongest amplitude deviations in Figure~\ref{fig:amps}.
%These are also some of the most diffuse orbitals in this basis,
%far larger than the average orbital extent of 26.62.
%As expected, the spatial extent of the PNOs built upon the MOs
%are much more localized. This shows that, in the MO basis,
%strong deviations are predominantly exhibited by amplitudes 
%corresponding to virtual orbitals with a larger spatial extent,
%creating sparsity. In the PNO basis, however, the deviations 
%are evenly spread across a large number of amplitudes corresponding
%to orbitals with relatively small spatial extent, resulting in 
%less sparsity in the amplitude tensor. 
%These findings suggest that virtual domains built for excited state
%properties should seek to include orbitals with large spatial extent,
%and any truncation criterion should preserve these orbitals.

Spatial extent alone may not be a suitable criterion for truncation -- 
this would have obvious a negative impact on the accuracy of the 
correlation energy, which is inherently local in nature. 
Additionally, Figure~\ref{fig:extent} highlights virtual MOs 7 
($\boldsymbol{+}$) and 15 ($\boldsymbol{\times}$),
which correspond to the strongest deviations in 
Figures~\ref{fig:MO_t1_50} and \ref{fig:MO_t1_100}, respectively.
That these orbitals are of varying extent
demonstrates that both diffuse and contracted orbitals play a role in 
the wave function dynamics.
In order to attain a balanced description of wave function components 
important for both energy and property calculations, the combination 
of appropriately determined spaces such as the combined PNO++ approach
has been fruitful. Still neglected in this approach
are the singles amplitudes, which are absent in the MP2 wave functions
used to approximate the occupied pair domains. Schemes to include 
these effects, such as approximate CC2-level $t_1$ guess amplitudes,
may further improve the space and allow greater flexibility for 
truncation. The prospect of utilizing these 
methodologies within the current framework is promising, and work is 
underway to explore their efficiency.

%conc.tex%
\section{Conclusions} \label{conc}
Here we introduce the density tensor representation (DTR) for machine-learning quantum mechanics applications. 
The representation is based on the previous t-amplitude tensor representation (TATR), with improvements made through strictly theoretical considerations of three categories: systematic improvement, storage, and simplified representation-target mapping. Investigating the limits of these categories on small test sets show a number of favorable properties. 
The DTR can be easily defined for any electronic structure method in which a density can be defined. When compared to the TATR for MP2, it achieves superior accuracy across most test cases when the MP2 wave function is expected to produce reasonable results. This accuracy is in the sub-mE$_h$ range for correlation energies. 
Furthermore, applications to molecular properties are both theoretically and operationally justified for representations utilizing electronic densities as raw wave function features. 
Roughly milliDebye error was achieved for correlated electronic dipole moments of several small molecules near equilibrium.
Extensions to include additional properties and molecular transferability are also considered, with the data-efficient DTR model providing a vital stepping stone to these generalizing improvements. 


    \chapter{PAPER 3} \label{ch:p3}
%\chapter{Introduction} \label{ch:int}

Modern synthetic organic chemistry employs a vast array of sophisticated instrumentation.
Principle among these are probes of light-matter interaction, which reveal rich structural and 
electro-magnetic characteristics. These instruments work by measuring the absorption, 
reflection, or refraction of an electromagnetic field (EMF) interacting with a target system. 
Experimental techniques which measure these interactions with respect to the energy or frequency
of the incident EMF are known as spectroscopy. Many staple experimental apparatus probe these
relationships, including (but certainly not limited to)
ultraviolet-visible absorption (UV-Vis), 
optical rotatory dispersion (ORD), 
flame atomic absorption (flame AA),
and electronic circular dichroism (ECD).
These data can be used for characterization or establishing structure-property relationships 
which aid in the development of novel systems for applications in materials, bio-organics, 
and more. 
Interpreting or predicting the results of such experiments requires a knowledge of fundamental 
light-matter interactions on a quantum level.  

A quantum description of any molecular interaction may be viewed as the effects of a perturbation
on a quantum-mechanical system. In the case of EMF, this perturbation may be ``static'' (fixed)
or ``dynamic'' (varying in frequency or time). A fully quantum-mechanical description of the 
light-matter system would require quantum electrodynamics (QED); however, it is often sufficient
to treat the EMF from a classical perspective, treating only the molecular response using
quantum mechanics. This allows us to utilize many well-established methods in the field of 
theoretical chemistry. These generally provide a representation of the system (a wave function or 
density) with which to take the expectation value of a given operator or, in some cases, predict 
the expectation value directly. Methods which utilize only mathematical techniques (that is, no
experimental or phenomenological parameters) are said to be \textit{ab initio} methods. These 
methods often (but not always) simplify or ignore the quantum effects of the nuclei and their 
motion, an approximation known as the Born-Oppenheimer approximation. 
Another subclass of \textit{ab initio} methods, which 
center around solving the time-dependent or time-independent Schr\"odinger wave equation to 
obtain an explicit form of the wave function, are 
known as wave function-based methods. Finally, methods that go beyond the mean-field or Hartree-Fock 
approximation, in which electrons only interact through an average self-consistent field, are 
classified as ``correlated'' methods. 
It is these correlated wave function-based methods under the Born-Oppenheimer
approximation upon which the bulk of the current work is built. As such, following this in Chapter 
\ref{ch:theory} is a primer in the theoretical underpinnings necessary to understand this work which are not described 
in detail in the publications that follow in sections [PAPER 1-3]. Chapter \ref{ch:theory} is roughly 
divided into details of electronic structure theory (Section \ref{se:est}) and molecular response 
properties (Section \ref{se:res}). 

First and foremost is many-body perturbation theory (MBPT). This framework allows us to 
separate the quantum mechanical properties of the isolated system versus some perturbing 
force, which is expanded in orders and truncated under the assumption that higher-order terms 
become negligible.
This force is often taken to be electron correlation when describing the electronic ground 
state wave function; however, it may also be a static or dynamic EMF. In section (\ref{ss:mp2})
we present some basic characteristics of general MBPT, in preparation for its application in the 
chapters that follow.

Secondly, in section (\ref{ss:cc}) we present the wave function-based method upon which the bulk 
of this work is based or seeks to approximate: coupled cluster (CC) theory. 
This method, like perturbation theory, is most commonly used for computing electron correlation 
(though its roots are in nuclear physics). 
Unlike perturbation theory, we do not perform an order-by-order expansion of the perturbation; 
instead, a ``cluster'' operator folds in contributions based on a physical intuition, that is the 
instantaneous occupation of many quantum states which give rise to electron correlation. Truncation
is then based on the number of quantum states (substituted or `excited` determinants). This method,
while accurate and systematically improvable, is notoriously expensive, suffering from high-order
polynomial scaling. Extensions to excited states and molecular response properties only compound this 
issue; as such, [PAPER 1] and [PAPER 2] are focused on ways to circumvent using CC on any but a small 
subset of systems. (It should be noted that, while [Paper 1] utilizes a different correlated method 
based on a density-functional theory approach, the primary goal was to ascertain the effects of 
fragmentation of a system through the many-body expansion on the computation of molecular properties. 
This is a benchmark study, whose goal is understood to be conclusions that would be applied to more 
expensive methods such as CC, where such benchmark studies could not be performed.)
Additional considerations of reduced density matrices, which are used throughout the work to refer
to correlated wave functions, follows in Section \ref{ss:rdm}.

Finally, we describe the method by which we couple the classical perturbing field to the 
quantum-mechanical correlated wave function. 
By applying perturbation theory we derive general expressions for tensors which describe 
the molecular response to an EMF. In the exact theory, these tensors turn out to be functions of the 
excited electronic states of the system; however, an approach to generalize 
these expressions to approximate theories such as truncated coupled cluster is discussed, which also 
avoids the costly evaluation of excited-state wave functions. For dynamic EMF, these properties may be 
described as a function of frequency, so the most common approach is to derive the working equations
using a Fourier transform of expressions obtained using the time-dependent Schr\"odinger equation 
(more generally known as response theory); however, these expressions may also be evaluated explicitly
in the time-domain. In Section \ref{se:res} we focus on the frequency-domain formulation, while
[PAPER 3] explores the recently-revived prospect of explicit time-propagation, as well as a 
technique to make this process cheaper through a well-established concept known as 
``local correlation''. 

%%comp.tex%

\section{Computational Details} \label{comp}
Second-order MBPT using a restricted closed-shell Hartree Fock reference wave function (i.e. M{\o}ller-Plesset perturbation theory, MP2)\cite{Møller1934,Bartlett1974a}, was used to generate both TATRs and DTRs for all systems. The TATR is generated from Eq.~(\ref{eq:mbtr}) using the highest 150 (by magnitude) amplitudes of double ``excitations'' 
$t_{ij}^{ab}$. 
For DTRs, the elements of the MP2 1-RDM are used. This choice is advantageous not just for comparison to Ref.~\citenum{Margraf2018}, but also due to the special form of the MP2 reduced density matrices\cite{Trucks1988}:

\begin{subequations} \label{eq:mp2_d}
    \begin{equation} \label{eq:mp2_opdm}
        D_{pq} = \bra{0}\hat T_2^{\dagger} a_p^{\dagger}a_q \hat T_2\ket{0}
    \end{equation}
    \begin{equation} \label{eq:mp2_tpdm}
        \Gamma_{pqrs} = 2\bra{0}a_p^{\dagger}a_q^{\dagger}a_sa_r \hat T_2\ket{0}.
    \end{equation}
\end{subequations}
where the ``excitation'' operators are defined as $\hat T_n = \left(\frac{1}{n!}\right)^2 t_{ij\ldots}^{ab\ldots}a_a^{\dagger}a_b^{\dagger}\ldots a_j a_i$, and we have again implied Einstein summation notation this time over occupied ${i,j,\ldots}$ and virtual ${a,b,\ldots}$ orbital spaces.
The form of Eqs.~(\ref{eq:mp2_opdm}) and~(\ref{eq:mp2_tpdm}) indicates that the reduced density matrices are fully defined by the doubles amplitudes and products thereof. In this light, it is reasonable to suggest that all of the information necessary to reproduce the wave function is coded into either of these and, to reduce the size of the representation, the 1-RDM should suffice for describing the MP2-level correlated wave function. Thus, any correlated one-electron property at the MP2 level can be described by this matrix.

Coupled cluster with single and double excitations (CCSD) was used to generate the high-level reference data, i.e. correlated energies and electronic dipole moments, for all systems. This method represents a marked improvement over MP2 in systems where correlation is strong, such as stretched diatomics, without incurring the additional cost of the gold-standard CCSD with perturbative triples method, CCSD(T). However, the CCSD(T) method could be used to provide higher quality targets (e.g. energies and dipoles) with no additional modifications --- as noted in section~\ref{algorithm}, the algorithm as presented is completely general. 

All electronic structure calculations were performed with the Psi4\cite{Parrish2017} electronic structure package. Molecular symmetry was used to maximize TATR performance with as few amplitudes as possible (except in section \ref{cutoffs}). All data (energies, dipoles, amplitudes, and densities) were harvested in full machine precision through either the Psi4-JSON interface or the PsiAPI infrastructure, with the exception of wave function amplitudes at the MP2 level, which were printed in the output file at ten decimal places. The def2-TZVP\cite{Weigend2005} basis was employed for diatomic calculations for comparison with Ref.~\citenum{Margraf2018}, and the aug-cc-pVDZ (aDZ) basis\cite{Dunning1989,Woon1994} was used for all remaining calculations. Density-fitted integrals were used in the construction of MP2 densities using the default auxiliary basis (the so-called ``RI'' basis sets for def2-TZVP\cite{Hattig2005} and aDZ\cite{Weigend2002}, obtained from the New Basis Set Exchange\cite{Pritchard2019}). 

As outlined above, the general procedure from Margraf and Reuter\cite{Margraf2018} was followed for machine-learning. 
The hyperparameters $\sigma_m$ and $\lambda$ were optimized on a uniform grid from $10^{-8}:10^{8}$ for all models unless otherwise specified, using the ``negative-mean-squared'' loss function. 
All regressions were performed using 20 training points, except in section \ref{cutoffs} where 12 training points were used to emphasize the difference between the two strategies considered. 
k-means clustering was repeated 30 times to account for stochastic deviations in the algorithm. 
Test sets consisting of $N/4$ points are held back before the clustering step.
The radial basis function kernel (sometimes called the Gaussian kernel), Eq.~(\ref{eq:rbf}), was used for all representations. 
Machine-learning algorithms were performed using Scikit-Learn (skl) \cite{Pedregosa2011} and the Machine-Learning Quantum Mechanics (MLQM) python package\cite{mlqm}, which generates a number of molecular representations, provides options for generating Psi4 input files and harvesting their results, and wraps some skl functions. 
Both of these codes are open-source and freely available on GitHub.

Geometries for diatomics were constructed by taking $N = 200$ uniform increments from 0.5-2.0 \AA . All other geometries were randomly selected from MD simulations ($N = 150$) carried out in the Gromacs\cite{Pronk2013} software package.
Simulations of (\textit{S})-methyloxirane and (\textit{R})-methylthiirane surrounded by a 5\AA\ box of water molecules were executed using an all-atom OPLS/AA forcefield\cite{Jorgensen1996} for the solute and the TIP3P model for water\cite{Jorgensen1983}. Each 5 ns trajectory was carried out in the NVT
ensemble, with the solute and solvent coupled separately to a temperature bath
at 300 K using a modified Berendsen thermostat and a coupling time of 0.1 ps. Geometries were selected from a set of 250 evenly-spaced snapshots along the trajectory.

Geometries, Psi4 input and output files, and their accompanying optimized (hyper)parameter values in JSON format can be found at the Virginia Tech Data Repository\cite{vtdata2020}   
to permit reproducibility of our results. Scripts for harvesting and manipulating data using MLQM are also included in the form of Jupyter Notebooks.

%%results.tex

\section{Results and Discussion} \label{se:results} 
Here we present results from the first applications of local correlation to RTCC.
Results are examined
by the convergence of absorption and ECD spectra to the reference results
in Section \ref{ss:spectra}, followed by an analysis of the amplitude
dynamics in Section \ref{ss:amps}. 
%Section \ref{ss:ext} considers the
%effects of localization on orbital spatial extent between PAO and PNO
%virtual spaces, and contrasts these results with the amplitude data in
%an attempt to explain their performance. 
Section \ref{ss:alt} explores some potential solutions for building
appropriate virtual spaces for truncation, such as considerations
of orbital extent,
perturbation-aware virtual spaces\cite{Crawford2019,DCunha2021}, and 
including or focusing on the effects of the singles amplitudes. 
%Finally, we present an alternative solution dubbed ``Semi-static RTCC'' 
%(SS-RTCC) in Section \ref{ss:ssrtcc} in which the doubles amplitudes 
%($t_2$ and $\lambda_2$ in Eqs.~(\ref{eq:diff_t}) and (\ref{eq:diff_l}), 
%respectively) are frozen after the determination of the ground-state wave 
%function, and only the singles amplitudes are allowed to respond to the 
%incident perturbation. 

\subsection{Absorption and ECD Spectra} \label{ss:spectra}
\subsubsection{Absorption} \label{sss:abs}
Absorption spectra are obtained from the Fourier transform of Eq.~(\ref{eq:abs}).
Figure~\ref{fig:pno_abs} shows the normalized absorption spectrum obtained from
a reference propagation along with five PNO cutoffs. The average truncated virtual 
orbital spaces are from roughly 20\% to 90\% of the MO virtual space (see caption). 
\begin{figure} 
    \centering
    \includegraphics[scale=.6]{p3/figures/pno_abs.png}
    \caption{Reference and PNO absorption spectra for five cutoffs: 
    [$1\times 10^{-10}$, $1\times 10^{-9}$, $1\times 10^{-8}$, $1\times 10^{-7}$, 
    $2\times 10^{-6}$] corresponding to [$93\%$, $82\%$, $63\%$, $44\%$, $24\%$]
    of the MO virtual space, respectively.}
    \label{fig:pno_abs}
\end{figure}
%For all truncated PNO virtual spaces considered, the base peak appears to within 1.5 eV of the 
%reference. 
Overall, truncated PNO virtual spaces approximate the position of the base peak well,
with the smallest space predicting a base peak within 1.5 eV of the reference,
and the two largest spaces predict this peak to within 0.2 eV of the reference.
Convergence to the reference base peak occurs from the right, indicating
a lowering of excited state energies as the size of the virtual space increases. 
This trend can also be seen for the smaller peak near 10 eV. However, convergence of the shoulder 
peaks on either side of the base peak, indicated by the inset of 
Figure~\ref{fig:pno_abs}, is less predictable. Even the largest spaces considered
do not correctly predict the excitation energy, with no clear advantage to having
93\% of the virtual space as compared to just 83\% for predicting these peaks.
This trend continues into the higher-energy range of the spectrum, with the 
performance of each cutoff being nearly indistinguishable. 

Performance of the PAO space is shown in Figure~\ref{fig:pao_abs}. 
\begin{figure} 
    \centering
    \includegraphics[scale=.6]{p3/figures/pao_abs.png}
    \caption{Reference and PAO absorption spectra for five cutoffs: 
    [$1\times 10^{-4}$, $1\times 10^{-3}$, $1\times 10^{-2}$, $5\times 10^{-2}$, 
    $1\times 10^{-6}$] corresponding to [$95\%$,$86\%$,$63\%$,$46\%$,$23\%$]
    of the MO virtual space, respectively.}
    \label{fig:pao_abs}
\end{figure}
The largest truncated PAO virtual space, on average 95\% of the MO space, accurately 
predicts the excitation energies for each major peak below 17 eV. Particularly around 10 eV, 
this is noticeably improved performance relative to the largest PNO space tested, 
with only a 2\% difference in the average size of the virtual space. 
However, accuracy rapidly declines even at 86\% of the virtual space, where the base peak
position is already worse than what was predicted with a PNO space of just 63\% of the 
MO space. Performance continues to degrade as energy increases and the average size of 
the PAO space decreases. For the final two cutoffs, at averages of 46\% and 23\% of the
MO space, the base peaks are 3 eV or more away from the reference, and no peak is 
exhibited near 25 eV. These spaces also fail to predict the second largest peak, the 
excitation just below 10 eV. 

\subsubsection{ECD} \label{sss:ecd}
Overall, neither scheme produced adequate results upon truncation of the virtual space. 
This result is not entirely surprising -- in studies of local correlation applied to 
response theory by Crawford \textit{et al}., 
\cite{McAlexander2016,Kumar2017,Crawford2019,DCunha2021} 
traditional schemes proved inaccurate for another 
electric dipole--electric dipole property, the electric polarizability.
In terms of response theory,
the polarizability (and the refractive index) is related to the \textit{real} part 
of the electric dipole--electric dipole linear response tensor 
($\boldsymbol{\alpha}_{ij}$ in Eq.~(\ref{eq:mu_exp})), 
while absorption
is related to the \textit{imaginary} part. Indeed, all linear absorptive properties
such as absorption and CD 
are related to the imaginary component of a linear response tensor, while dispersive 
properties such as refractive index and 
circular birefringence (also known as optical rotation)
are related to the real component.
\cite{Barron2004,Norman2011}
To continue, we will look at another absorptive property which is related to the mixed
electric dipole--magnetic dipole linear response tensor -- ECD.

The ECD spectrum is obtained from the Fourier transform of Eq.~(\ref{eq:ecd}).
Being a bisignate, mixed-response property, ECD is a considerable computational
challenge, similar to its dispersive counterpart circular birefringence. 
Figure~\ref{fig:pno_ecd} shows the results for an ECD spectrum in the same PNO 
orbital spaces used in the previous section.
\begin{figure} 
    \centering
    \includegraphics[scale=.6]{p3/figures/pno_ecd.png}
    \caption{Reference and PNO ECD spectra for five cutoffs: 
    [$1\times 10^{-10}$, $1\times 10^{-9}$, $1\times 10^{-8}$, $1\times 10^{-7}$, 
    $2\times 10^{-6}$] corresponding to [$93\%$,$82\%$,$63\%$,$44\%$,$24\%$]
    of the MO virtual space, respectively.}
    \label{fig:pno_ecd}
\end{figure}
The dynamic response of the magnetic dipole to the electric field in this frequency 
range is considerably more complicated than that of the electric dipole. Below 60\% of
the MO space, virtually all distinguishing characteristics of the reference 
spectrum are unidentifiable. Further, at 82\%, the base peak appears to be a pair of 
peaks, more resembling the pair of peaks appearing just above 15 eV in the 
reference spectrum, with the major peak just below 15 eV being the second strongest.
At an average of 93\%, the overall \textit{shape} of the spectrum in the 10 eV to 
20 eV range more closely resembles that of the reference; however, the excitation
energies are, in some cases, even less accurate than those of smaller PNO spaces.
The trend of lowering excited state energies with increased virtual space seen in 
Section~\ref{sss:ecd} is no longer discernible. 

As in the case of absorption, the PAO basis is not noticeably more efficient at
approximating the full MO space than the PNO space. Figure~\ref{fig:pao_ecd} 
shows the results using the same truncated PAO spaces as in Section~$\ref{sss:abs}$.
\begin{figure} 
    \centering
    \includegraphics[scale=.6]{p3/figures/pao_ecd.png}
    \caption{Reference and PAO ECD spectra for five cutoffs: 
    [$1\times 10^{-4}$, $1\times 10^{-3}$, $1\times 10^{-2}$, $5\times 10^{-2}$, 
    $1\times 10^{-6}$] corresponding to [$95\%$,$86\%$,$63\%$,$46\%$,$23\%$]
    of the MO virtual space, respectively.}
    \label{fig:pao_ecd}
\end{figure}
The performance of the PAO basis near the base peak varies wildly with truncation,
as in the PNO case. In the low-frequency region, the PAO results are considerably 
worse -- see the two negative peaks between 10 eV and 15 eV. Curiously, the 
largest PAO spaces considered predict significant peaks above 25 eV that are not 
present in the reference, any of the PNO spaces tested, or the smaller PAO
spaces. This suggests a strong sensitivity of the response of the wave function
to the completeness threshold used for determining the occupied domains.
%fundamentally different electronic structure in the
%presence of the perturbing field, which is a direct result of the charge-based 
%completeness threshold used for determining the occupied domains.
%albeit at very large frequencies which are of little practical use.

\subsection{Amplitude Dynamics} \label{ss:amps}
As evidenced by the preceding data, the truncated PNO and PAO virtual spaces do not
efficiently model the wave function in the presence of a perturbing EMF. As noted in
Section~\ref{sss:abs}, these shortcomings are well-documented in the case of response
theory. However, a real-time formalism offers the opportunity to analyze the wave 
function in great detail over time, perhaps shedding light on \textit{where} and 
\textit{how} the locally correlated wave functions are deficient. The following
section will scrutinize the $t_\mu$ and $\lambda_\mu$ amplitudes of 
Eqs.~(\ref{eq:t_mu}) and (\ref{eq:l_mu}), respectively, in hopes of determining the 
important fluctuations in the wave function and whether these spaces sufficiently 
capture these changes.

Response to external perturbations by the CC amplitudes give rise to
dynamic energetics and properties. In the past, distributions of perturbed amplitudes
(relative to their ground-state counterparts) have been used to justify the 
difficulty in computing response functions with local correlation methods in the 
frequency domain. 
\cite{McAlexander2016,Crawford2019,DCunha2021} 
However, initial findings show that in RTCC, the relative distribution of amplitudes 
by magnitude is not significantly impacted.\cite{Crawford2019} 
Despite this, typical means of exploiting amplitude
sparsity have been shown to be inefficient by the preceding sections. 
First, to understand the response of 
the amplitudes to the external perturbation, we plot the 
change in the norm of the amplitude tensors relative to the ground-state amplitudes 
as a function of time in Figure~\ref{fig:norm}.
\begin{figure} 
    \centering
    \includegraphics[scale=.6]{p3/figures/amp_norm.png}
    \caption{Time-dependent change in the norm of the amplitude 
    tensors relative to the ground-state amplitudes.
    (Field and step parameters remain unchanged, and 
    the amplitude norm is taken at every 1 a.u.)}
    \label{fig:norm}
\end{figure}
Results for the untruncated PNO and PAO spaces are identical to those
for the MO space, as the unitary transformations resulting from untruncated localized virtual spaces
in Eqs.~(\ref{eq:rotate}) preserves the tensor norm.
Amplitude norms from propagations carried out in truncated PNO and PAO spaces
are nearly indistinguishable (see SI).

Figure~\ref{fig:norm} shows that the magnitude of the response by the wave function
is predominantly within the singles amplitudes $t_1$ and $\lambda_1$. This is 
consistent with the notion that singles are paramount for the computation of 
response properties.\cite{Christiansen1995,Koch1997} However, the form of Eq.~(\ref{eq:pair_D})
does not include any contributions by singles, due to being built from MP2-level
amplitudes where singles do not contribute until at least the second order 
in the wave function and fourth order in the energy. This suggests that even in schemes
which seek to include the EMF perturbation in the construction of the reduced virtual space,
such as PNO++,\cite{DCunha2021} response of the singles should be considered.

Aside from the matrix norm, we can also inspect the individual amplitudes to track
their evolution in time. The heat maps in Figure~\ref{fig:amps} show the difference 
in $t_1$ amplitudes, relative to the ground state, for three time steps selected 
from the first 100 a.u. of the simulation.
\begin{figure}
    \begin{subfigure}{.5\textwidth}
        \centering
        \includegraphics[scale=0.5]{p3/figures/MO_delta_t1_1.png}
        \caption{}
        \label{fig:MO_t1_1}
    \end{subfigure}%
    \begin{subfigure}{.5\textwidth}
        \centering
        \includegraphics[scale=0.5]{p3/figures/MO_delta_t1_50.png}
        \caption{}
        \label{fig:MO_t1_50}
    \end{subfigure}
    \begin{subfigure}{.5\textwidth}
        \centering
        \includegraphics[scale=0.5]{p3/figures/MO_delta_t1_100.png}
        \caption{}
        \label{fig:MO_t1_100}
    \end{subfigure}
    \caption{MO-basis $t_1$ amplitude deviations from $t = 0$ after (a) 1 a.u., (b) 50 a.u., and 
    (c) 100 a.u. of time propagation. Each row contains the same four occupied orbital indices
    and a subset of virtual indices as indicated by the x-axis labels.}
    \label{fig:amps}
\end{figure}
The amplitudes are ordered by the orbital energies of the associated MOs. 
The amplitudes which experience significant oscillations vary throughout the simulation,
though there are several discernible trends. First, most large amplitude deviations
are associated with all occupied orbitals simultaneously. This is due to the relatively small size of 
the system, with only four occupied orbitals, all of which are likely important in the
description of the ground- and excited-state wave functions. Secondly, 
at any given time during the propagation,
a large number of amplitudes have not significantly deviated from their ground state values.
This supports the notion
that relative sparsity is maintained within the amplitudes throughout the simulation,
but this sparsity is distributed differently throughout the amplitude tensors as
the wave function is propagated.  

A third trend is that amplitudes which respond strongly tend to be associated with 
low-energy virtual orbitals. Chemical intuition would suggest that energetically 
low-lying molecular orbitals will be the most involved in electronic excitations.
However, while amplitude responses are indeed larger for lower-energy virtual orbitals, 
smaller amplitude 
deviations in Figure~\ref{fig:amps} extend far into the virtual space. This explains
the difficulty of simply truncating with respect to orbital energy: the 
high-energy MOs are still important to the time-evolution of the wave function 
in the presence of an EMF. 

Figure~\ref{fig:pno_amps} shows the $t_1$ amplitudes for the same simulation,
rotated into the untruncated PNO basis using $Q_{ii}$ as defined in
Eq.~(\ref{eq:Q_pno}).
\begin{figure}
    \begin{subfigure}{.5\textwidth}
        \centering
        \includegraphics[scale=0.5]{p3/figures/PNO_delta_t1_1.png}
        \caption{}
        \label{fig:PNO_t1_1}
    \end{subfigure}%
    \begin{subfigure}{.5\textwidth}
        \centering
        \includegraphics[scale=0.5]{p3/figures/PNO_delta_t1_50.png}
        \caption{}
        \label{fig:PNO_t1_50}
    \end{subfigure}
    \begin{subfigure}{.5\textwidth}
        \centering
        \includegraphics[scale=0.5]{p3/figures/PNO_delta_t1_100.png}
        \caption{}
        \label{fig:PNO_t1_100}
    \end{subfigure}
    \caption{PNO-basis $t_1$ amplitude deviations from $t = 0$ after (a) 1 a.u., (b) 50 a.u., and 
    (c) 100 a.u. of time propagation. Each row contains the same four occupied orbital indices
    and a subset of virtual indices as indicated by the x-axis labels.}
    \label{fig:pno_amps}
\end{figure}
(It should be noted that, due to redundancy in the AO-based virtual 
spaces for each pair, PAO-basis amplitudes cannot be compared directly in 
this manner.)
It can be immediately seen that the amplitude deviations
are less sparse in the PNO basis after the application of the EMF. 
Many more amplitudes exhibit
perceivable differences, and strong deviations (magnitudes approaching 
$1\times 10^{-5}$) are no longer present. This is a clear demonstration
of the issue with truncating orbital spaces based on the present criterion ---
rather than exploiting sparsity, the amplitude tensors have become less sparse.
It may also suggest a recipe for building a more appropriate 
virtual space for truncation. 
In the following section, we propose some alternative schemes based on the 
literature and the results of this study.
%In the following section, we compare a selection
%of orbitals which correspond to strong amplitude deviations in the MO basis 
%(specifically virtuals 3, 4, 7, and 15)
%based on orbital spatial extent to determine if this may be a possible criterion 
%for truncation of the virtual space.

\subsection{Possible Alternatives} \label{ss:alt}
\begin{figure}
    \centering
    \includegraphics[scale=0.75]{p3/figures/extent.png}
    \caption{Virtual MO energy $\epsilon_a$ and the
    occupation number $n_a$ (plotted on a log scale)
    for unique PNO spaces $i_1$ and $i_2$ 
    versus orbital extent in arbitrary units.
    Virtual MOs 7 and 15 are denoted by a solid $\boldsymbol{+}$ 
    and $\boldsymbol{\times}$, respectively.
    The horizontal line denotes a PNO cutoff 
    of $1\times 10^{-7}$.}
    \label{fig:extent}
\end{figure}
Figure~\ref{fig:extent} shows the virtual MO energy $\epsilon_a$ and the 
PNO occupation number $n_a$ plotted against the orbital extent
$\langle r^2 \rangle$ in arbitrary units. 
In the PNO basis, a unique virtual space is prepared
for every occupied pair, resulting in 16 unique spaces for the four occupied
spatial orbitals $i$. However, for transforming
a single orbital index, we only require the diagonal rotation matrices,
\textit{i.e.}, $Q_{ii}$. There are four such spaces; however, by symmetry,
only two are unique. Both are included in Figure~\ref{fig:extent}. 

Truncation of the PNO space begins from the bottom of Figure~\ref{fig:extent}. 
At an occupation number cutoff of $1\times 10^{-7}$ (indicated by a horizontal
line), all orbitals below this line are neglected in the PNO space. Roughly
66\% of the virtual space lies in this region. From these data, it is clear
that even modest truncation of the virtual space neglects the diffuse 
regions of the wave function, which are important for excited-state
properties in systems with significantly delocalized characteristics,
such as systems containing Rydberg-type excitations.

%Table~\ref{ta:ext} reports the orbital extent $\langle r\rangle$ for virtual orbitals $a$
%corresponding to four of the strongest deviations in Figure~\ref{fig:amps}. 
%These orbitals are labeled 4, 7, 8, and 15, in
%both the MO and PNO basis. In the PNO basis, a unique virtual space is prepared
%for every occupied pair, resulting in 16 unique spaces for the four occupied
%spatial orbitals $i$. However, for transforming
%a single orbital index, we only require the diagonal rotation matrices,
%\textit{i.e.}, $Q_{ii}$. There are four such spaces; however, by symmetry, the domains of occupied
%orbitals 1 and 4 are the same, as are 2 and 3. Thus, orbital extents for only those
%two unique orbital spaces $i_1$ and $i_2$  are shown.
%\begin{table}
%    \centering
%    \begin{tabular}{|c|c|c|c|}
%        \hline
%%        $a$ &   MO  &   PNO   &        \\  
%        $a$ &  MO  &   \multicolumn{2}{c|}{PNO} \\  
%        \hline
%%        \multicolumn{2}{|c|}{}  & $i_1$   & $i_2$  \\ 
%            &       & $i_1$   & $i_2$  \\
%        \hline
%%         3  & 59.23 & 7.9     & -24.67 \\ 
%%        \hline
%         4  & 88.76 & -13.08  & -3.13  \\ 
%        \hline
%         7  & 66.00 & 0.73    & -0.07  \\ 
%        \hline
%         8  & 65.56 & -3.91   & -10.35 \\ 
%        \hline
%         15 & 37.39 & -15.11  &  1.32  \\
%        \hline
%    \end{tabular}
%    \caption{Orbital spatial extent of four selected virtual orbitals in the MO
%    and PNO spaces.}
%    \label{ta:ext}
%\end{table}
%A full table of virtual orbital extents can be found in the SI. 
%In the MO basis, the orbitals in Table~\ref{ta:ext} 
%correspond to the strongest amplitude deviations in Figure~\ref{fig:amps}.
%These are also some of the most diffuse orbitals in this basis,
%far larger than the average orbital extent of 26.62.
%As expected, the spatial extent of the PNOs built upon the MOs
%are much more localized. This shows that, in the MO basis,
%strong deviations are predominantly exhibited by amplitudes 
%corresponding to virtual orbitals with a larger spatial extent,
%creating sparsity. In the PNO basis, however, the deviations 
%are evenly spread across a large number of amplitudes corresponding
%to orbitals with relatively small spatial extent, resulting in 
%less sparsity in the amplitude tensor. 
%These findings suggest that virtual domains built for excited state
%properties should seek to include orbitals with large spatial extent,
%and any truncation criterion should preserve these orbitals.

Spatial extent alone may not be a suitable criterion for truncation -- 
this would have obvious a negative impact on the accuracy of the 
correlation energy, which is inherently local in nature. 
Additionally, Figure~\ref{fig:extent} highlights virtual MOs 7 
($\boldsymbol{+}$) and 15 ($\boldsymbol{\times}$),
which correspond to the strongest deviations in 
Figures~\ref{fig:MO_t1_50} and \ref{fig:MO_t1_100}, respectively.
That these orbitals are of varying extent
demonstrates that both diffuse and contracted orbitals play a role in 
the wave function dynamics.
In order to attain a balanced description of wave function components 
important for both energy and property calculations, the combination 
of appropriately determined spaces such as the combined PNO++ approach
has been fruitful. Still neglected in this approach
are the singles amplitudes, which are absent in the MP2 wave functions
used to approximate the occupied pair domains. Schemes to include 
these effects, such as approximate CC2-level $t_1$ guess amplitudes,
may further improve the space and allow greater flexibility for 
truncation. The prospect of utilizing these 
methodologies within the current framework is promising, and work is 
underway to explore their efficiency.

%%conc.tex%
\section{Conclusions} \label{conc}
Here we introduce the density tensor representation (DTR) for machine-learning quantum mechanics applications. 
The representation is based on the previous t-amplitude tensor representation (TATR), with improvements made through strictly theoretical considerations of three categories: systematic improvement, storage, and simplified representation-target mapping. Investigating the limits of these categories on small test sets show a number of favorable properties. 
The DTR can be easily defined for any electronic structure method in which a density can be defined. When compared to the TATR for MP2, it achieves superior accuracy across most test cases when the MP2 wave function is expected to produce reasonable results. This accuracy is in the sub-mE$_h$ range for correlation energies. 
Furthermore, applications to molecular properties are both theoretically and operationally justified for representations utilizing electronic densities as raw wave function features. 
Roughly milliDebye error was achieved for correlated electronic dipole moments of several small molecules near equilibrium.
Extensions to include additional properties and molecular transferability are also considered, with the data-efficient DTR model providing a vital stepping stone to these generalizing improvements. 


    %conc.tex%
\section{Conclusions} \label{conc}
Here we introduce the density tensor representation (DTR) for machine-learning quantum mechanics applications. 
The representation is based on the previous t-amplitude tensor representation (TATR), with improvements made through strictly theoretical considerations of three categories: systematic improvement, storage, and simplified representation-target mapping. Investigating the limits of these categories on small test sets show a number of favorable properties. 
The DTR can be easily defined for any electronic structure method in which a density can be defined. When compared to the TATR for MP2, it achieves superior accuracy across most test cases when the MP2 wave function is expected to produce reasonable results. This accuracy is in the sub-mE$_h$ range for correlation energies. 
Furthermore, applications to molecular properties are both theoretically and operationally justified for representations utilizing electronic densities as raw wave function features. 
Roughly milliDebye error was achieved for correlated electronic dipole moments of several small molecules near equilibrium.
Extensions to include additional properties and molecular transferability are also considered, with the data-efficient DTR model providing a vital stepping stone to these generalizing improvements. 


    %%%% ADDING TO TRY TO FIX BIB....
    \bibliographystyle{achemso}

      
	% This is the standard bibtex file. Do not include the .bib extension in <bib_file_name>.
	% Uncomment the following lines to include your bibliography: 
	\bibliography{thesis,p1/p1,p2/p2}
%	\bibliographystyle{plainnat}   

	% This formats the chapter name to appendix to properly define the headers:
%	\appendix
	% Add your appendices here. You must leave the appendices enclosed in the appendices environment in order for the table of contents to be correct.
%	\begin{appendices}
%		\chapter{First Appendix} \label{app:appendix_one}
%			\section{Section one} \label{ase:app_one_sect_1}
%				\lipsum[1-3]
%			\section{Section two} \label{ase:app_one_sect_2}
%				\lipsum[1-3]
%		\chapter{Second Appendix} \label{app:appendix_two}
%			\lipsum[2]
%	\end{appendices}

\end{document}

