\section{Introduction} \label{se:introp1}

The accurate simulation of solvent effects on molecular optical response is exceptionally challenging because of the numerous dynamical factors required for a physically faithful model.\cite{Tomasi94,Gao96,Cramer99,Tomasi02, Mennucci02, Mukhopadhyay2006,Mukhopadhyay2007,Lipparini2013,Egidi18} Whereas many properties may be viewed as intrinsic to the solute subjected to a perturbation by the solvent environment (perhaps as represented by a dielectric continuum), others --- particularly mixed electric-/magnetic-field responses\cite{Barron2004,Ruud05:review,Crawford2006,Crawford2007,Crawford12:review,Crawford2018} --- are, in fact, correctly viewed as inherent to the combined solute/solvent system.  Schemes based on implicit solvent\cite{Mennucci02,Kongsted05,Kongsted06,Howard2018a} and frozen density embedding (FDE)\cite{Govind98,Neugebauer05:FDE,Jacob07:FDE,Gomes08,Crawford2015a} have been used, but with limited success.  Concomitant non-additivity effects, dynamic configurational sampling, and molecule-specific interactions serve to exacerbate the computational and theoretical demands on robust simulations.\cite{Mukhopadhyay2006,Mukhopadhyay2007} Thus, reducing the cost of these calculations for explicitly solvated systems has become a major concern.\cite{Crawford2010,McAlexander2016,Kumar2017,Howard2018a}

The many-body expansion (MBE) formalism has seen widespread success in predicting the energies of large molecular systems at a reduced cost, though recently the limits of these successes have been tested with clusters of increasing size and complexity.\cite{Ouyang2014,Richard2014,Ouyang2016,Richard2018a} These limits are bounded by well-explored challenges like loss of precision\cite{Richard2014,Liu2017a,Richard2018a} and basis set superposition error (BSSE).\cite{Boys1970,Wells1983,Valiron1997,Salvador2003,Kamiya2008,Richard2013,Ouyang2014,Ouyang2015,Liu2017a,Mayer2017,Richard2018a} Electrostatic embedding (EE) has been applied to MBE treatments of water clusters alongside BSSE corrections to improve convergence\cite{Richard2013}, as well as the \textit{N}-body:Many-body QM:QM technique\cite{hopkins2003a} to reduce the cost of including higher-order effects. The latter has also been applied to small water clusters for predicting vibrational frequencies.\cite{Howard2013}
    
    The fundamental concept behind the MBE is the decomposition of the energy (or other properties) into a sum of smaller contributions from sub-components of a complex, interacting system.\cite{Hankins1970,Kaplan1986} In the MBE, the energy (or energy derivative) of $n$ interacting fragments may be expanded in orders of interaction energies:
    \begin{subequations} 
        \begin{equation} \label{eq:mb_energy}
            E_{1,2,\ldots ,n} = \sum_{i\subseteq N_1}E_{i} + \sum_{ij\subseteq N_2}\epsilon_{ij} + \sum_{ijk\subseteq N_3}\epsilon_{ijk} + \ldots,
        \end{equation}
   where $N_1$, $N_2$, $N_3$, etc., denote the sets of unique monomers, dimers, trimers, etc. The interaction energies are defined as, for example, for dimers,
        \begin{equation} \label{eq:mb_energy2} 
            \epsilon_{ij} = E_{ij} - E_{i} - E_{j},
        \end{equation}
    and for trimers,
        \begin{equation} \label{eq:mb_energy3}
            \epsilon_{ijk} = E_{ijk} - (E_{ij} + E_{ik} + E_{jk}) + (E_{i} + E_{j} + E_{k}),
        \end{equation}
    \end{subequations}
    where the subtraction of energies of sub-components is necessary to avoid overcounting of energy contributions. While the untruncated MBE scheme is formally exact, large computational savings result from the truncation of Eq.~(\ref{eq:mb_energy}) to $k$-body terms where $k < n$. Truncation to two- or three-body terms only has proved sufficient for interaction energies in many examples. However, limiting cases exist for increasingly complex systems.\cite{Ouyang2014,Richard2014,Ouyang2016,Richard2018a}.
    
    While interaction energies are the popular target of MBE applications, properties computed using the MBE have received comparatively little attention. Though some work has been carried out concerning induced electronic properties in linear species,\cite{Skwara2007a,Skwara2009,Baranowska2010,Zawada2011,Zawada2011a} solvated systems such as those studied in Ref.~\citenum{Leverentz2012} are of particular interest for chiroptical property prediction. A study by Mach and Crawford\cite{Mach2014} showed that the MBE for a number of solvated chiral systems exhibited oscillatory convergence in specific rotation with respect to $n$-body truncation, while interaction energies, dipole moments, and dynamic polarizabilities still converged well by the three-body approximation as is typical for interaction energies with the MBE for systems with fewer than 12 monomers\cite{Ouyang2014,Richard2018a}.   

    The successive sign-flips of $k$-body terms, such as $E_{i}$ changing from positive to negative in Eqs.~(\ref{eq:mb_energy2}) and (\ref{eq:mb_energy3}), have been suggested to be the cause of oscillations in the MBE for large clusters,\cite{Ouyang2014} and BSSE has been identified as a major contributing factor to these oscillating errors. BSSE is a result of an imbalance of basis functions and thus will be present in any electronic structure calculation using a finite number of functions. The BSSE of a dimer interaction term in Eq.~(\ref{eq:mb_energy2}) is relatively straightforward to correct using the Boys and Bernardi Counterpoise Method (BBCP):\cite{Boys1970}
    \begin{equation} \label{eq:Boys}
        \epsilon_{ij} = E_{ij} - E_{i}^{(ij)} - E_j^{(ij)},
    \end{equation}
    where the superscript denotes the basis set used, and any terms without a superscript are calculated in their own basis. When $E_i$ and $E_j$ are calculated in just their own basis sets, each monomer does not benefit from the nearby basis functions placed on the partner monomer.  However, when calculating $E_{ij}$, the basis functions are shared between the monomers, which increases flexibility in the wavefunction and changes the energy despite having little bearing on the interaction between monomers $i$ and $j$, only their (incomplete) basis sets. By using the same basis (that of the dimer $ij$) for all three calculations, there is no imbalance in the wavefunction, hence the term ``counterpoise (CP) correction''.
    
    The BBCP method was originally intended for correcting dimer interaction energies, but two fundamentally different generalizations of the BBCP correction scheme for an $n$-body interaction term are currently in use: the site-site function counterpoise (SSFC) and Valiron-Mayer Function Counterpoise (VMFC) methods.\cite{Wells1983,Valiron1997} The former, also referred to as the ``full cluster basis,'' simply uses the basis functions of the full $n$-body cluster for each fragment calculation,
    \begin{equation} \label{eq:SSFC}
        E_{1,2,\ldots ,n} = \sum_{i\subseteq N_1}E_{i}^{(n)} + \sum_{ij\subseteq N_2}\epsilon_{ij}^{(n)} + \sum_{ijk\subseteq N_3}\epsilon_{ijk}^{(n)} + \ldots 
    \end{equation}
with similar generalizations for Eq.~(\ref{eq:mb_energy2}) and (\ref{eq:mb_energy3}). While often prohibitively expensive, it has been shown to be effective in eliminating oscillations in the MBE.\cite{Ouyang2014,Richard2018a}

    The VMFC method is based on correcting for BSSE at each $k$-body level such that the $k$-mer basis is used for each $k$-body interaction term,
    \begin{equation} \label{eq:VMFC}
        E_{1,2,\ldots,n} \approx \sum_{i\subseteq N_1}E_{i}^{(i)} + \sum_{ij\subseteq N_2}\epsilon_{ij}^{(ij)} + \sum_{ijk\subseteq N_3}\epsilon_{ijk}^{(ijk)} + \ldots 
    \end{equation}
    By correcting at each $k$-body term, the VMFC method prevents spurious ``ghost dipoles'' from appearing in the $k$-body approximations, which can occur with the SSFC method: asymmetrically distributed basis functions placed around a real fragment can cause electron density to move to those locations, causing net dipoles (and possibly other properties) which are solely dependent on the placement of ghost functions, though such effects should only appear significantly for close-packed fragments. The VMFC method requires that each calculation be performed in multiple basis sets, resulting in many more calculations (an additional binomial coefficient, in fact) than the classic MBE. The exact nature of the MBE at $n$-body is also lost, e.g. the monomer energies in Eq.~(\ref{eq:mb_energy}) no longer cancel exactly with the monomer energies in Eq.~(\ref{eq:mb_energy2}), due to the difference in basis sets. The resulting energy is a ``counterpoise corrected'' energy, which makes benchmarking relative to complete cluster calculations difficult. In practice, results from the VMFC method vary only slightly relative to those from the SSFC method and other correction schemes\cite{Salvador2003,Richard2013,Richard2018a}.

  The goal of the present study is to understand the cause of the oscillations in the MBE for specific rotation observed in the earlier work by Mach and Crawford\cite{Mach2014} and to determine whether or not they can be removed using a BSSE correction scheme.  We will focus on the SSFC method due to its convergence to the correct $n$-body limit.  We have chosen to exclude the VMFC method at present because of its high computational cost and ambiguity in benchmarking calculations.  We further extend the previous work by employing larger solvation clusters and testing additional configurations of explicitly solvated systems.
