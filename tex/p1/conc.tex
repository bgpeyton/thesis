\section{Conclusions} \label{se:concp1}
We find that the SSFC serves to reduce significantly the BSSE in MBE expansions of higher-order properties such as dipole polarizabilities and specific rotations, resulting in somewhat dampened oscillations and thus smoother convergence in most cases.  Computed standard deviations decrease on average with inclusion of SSFC corrections, with the only exceptions being (\textit{S})-methylthiirane in a six-water solvent shell computed at the CAM-B3LYP/aDZ level of theory and (\textit{S})-methyloxirane in a seven-water solvent shell computed at the B3LYP/aTZ level of theory due to the erratic behavior of the MBE at the two-body correction.  Furthermore, these trends held for variations in the choice of MD snapshot, density functional, and basis set.  We thus conclude that, in agreement with previous studies on interaction energies,\cite{Ouyang2014,Liu2017a,Richard2018a} counterpoise corrections are essential for the MBE calculation of response properties, whose magnitude (and sign) often depend heavily on the diffuse regions of electron density most prone to BSSE.  However, the convergence difficulties characteristic of properties such as specific rotations remain a challenge, exhibiting oscillations in the MBE that cannot be remedied by a counterpoise correction alone.  As illustrated clearly by the example of (\textit{S})-methyloxirane in a solvent shell of 13 water molecules, significant oscillations remain even after BSSE corrections, and these are inherent to non-additive, non-perturbative properties such as mixed electric-/magnetic-field responses.  Significant changes in the subsystem calculations as $n$ increases can cause sign flips of successive terms, and small changes in the subsystem can cause large changes in the computed specific rotation contribution.  In short, the rapid convergence of the MBE inherently assumes that the property in question is fundamentally additive, in this case that the property may be viewed as that of the solute with relatively small perturbations arising from the nearby solvent molecules.  This requirement does not hold for specific rotations (and related chiroptical responses), leading to the conclusion that the MBE is of very limited utility for such cases.
