\section{Computational Details} \label{se:compp1}
The dynamic polarizability at an external field frequency, $\omega$, is computed as the isotropic average of the dynamic polarizability property tensor:\cite{Barron2004,Crawford2006}
    \begin{equation} \label{eq:alpha}
        \alpha_{\alpha\beta}(\omega) = \frac{2}{\hbar}\sum_{j\neq n}\frac{\omega}{\omega_{jn}^2-\omega^2}\textrm{Re}(\bra{n}\mu_\alpha\ket{j}\bra{j}\mu_\beta\ket{n}),
    \end{equation}
    where $n$ is the electronic ground state, $j$ is an electronic excited state with excitation frequency $\omega_{jn}$, and $\boldsymbol{\mu}$ is the electric dipole operator.  
    Similarly, the specific rotation (in $\mathrm{deg\ dm^{-1}\ (g/mL)^{-1}}$) is related to the isotropic average of the electric-dipole/magnetic-dipole property tensor (also called the Rosenfeld tensor):\cite{Rosenfeld1929a}
    \begin{equation} \label{eq:G'}
        G'_{\alpha\beta}(\omega)=-\frac{2}{\hbar}\sum_{j\neq n}\frac{\omega}{\omega_{jn}^2-\omega^2}\textrm{Im}(\bra{n}\mu_\alpha\ket{j}\bra{j}m_\beta\ket{n}).
    \end{equation}

%    The $\boldsymbol{\alpha}$ and $\boldsymbol{G'}$ property tensors were computed in this
%work using the linear response formalism\cite{Norman2011} with gauge-including atomic
%orbitals (GIAOs)\cite{Pulay1990}. The linear response function can be expressed as a
%derivative of the time-dependent quasi-energy\cite{Norman2011}, suggesting it can in
%principle be expanded using an MBE, which was the original driving force for the earlier
%work\cite{Mach2014}. The property tensors resulting from the linear response treatment were expanded within the MBE approximation Eq.~(\ref{eq:mb_energy}) or SSFC approximation Eq.~(\ref{eq:SSFC}), and the resulting fragment polarizabilities and specific rotations (along with dipole moments and interaction energies) examined for convergence relative to $n$-body truncation as in the previous study.\cite{Mach2014} 
    
    The $\boldsymbol{\alpha}$ and $\boldsymbol{G'}$ property tensors were computed in this
work using the linear response formalism\cite{Norman2011}, the latter using gauge-including atomic
orbitals (GIAOs)\cite{Pulay1990}. 
Dynamic response properties such as polarizabilities or specific rotations may
be formulated in terms of the time-averaged quasi-energy,
\begin{equation}
{\cal Q} = \frac{1}{T} \int_0^T \bra{\bar{\Psi}} \left( \hat{H} -
i \frac{\partial}{\partial t}\right) \ket{\bar{\Psi}} dt,
\end{equation}
where $\bar{\Psi}$ is the regular part of the phase-isolated wave
function\cite{Langhoff72} and $\hat{H}$ includes the time-dependent
external field.  For periodic potentials, the time-averaged
quasi-energy is uniquely defined, and both variational and
Hellmann-Feynman theorems apply.\cite{Helgaker12}  Indeed, in the limit of a
time-independent Hamiltonian, the quasi-energy reduces to the energy of
the stationary state.  Thus, just as the many-body expansion applies to
the time-independent energy, it also applies to the time-averaged
quasi-energy.   Furthermore, since response functions are computed as
derivatives of ${\cal Q}$ with respect to the perturbation coefficients
(e.g., the polarizability is the negative of the second derivative of the
 time-averaged quasi-energy with respect to the strength parameter of
 the external electric field), response functions should also be
subject to the many-body expansion, just as in the time-independent
case.  This was the driving force behind the 
Mach paper, which also showed that this was correct for one linear response property, the 
polarizability\cite{Mach2014}. In the following, it will be shown that while this expansion is 
valid, its effective truncation depends highly on the additivity of the subjected property. 
CP corrections should remove BSSE errors which do not cancel in the MBE equations, and other 
potential issues will be explored for the non-convergent specific rotation.

    Geometries for (\textit{S})-methyloxirane in a cage of seven water molecules, (\textit{S})-methylthiirane surrounded by six water molecules, and (\textit{M})-dimethylallene with seven water molecules were re-used from the previous study\cite{Mach2014} for consistency. These geometries were generated by Gromacs\cite{Pronk2013} simulations of each solute in water, and snapshots with 5.5 \AA\ solvent shells were extracted from the resulting trajectories. Additional geometries for (\textit{S})-methyloxirane in seven- and 13-water-molecule cages and (\textit{S})-methylthiirane in a six water-molecule cage were generated to test the effects of geometry and solvent shell size on MBE convergence and BSSE.  All geometries are available in the Supporting Information.

    Interaction energies, dipole moments, and specific rotations were calculated using the B3LYP\cite{Lee,Becke1993} functional in the aug-cc-pVDZ (aDZ) basis\cite{Dunning1989,Woon1994} as in the previous study. Additionally, the aug-cc-pVTZ (aTZ) basis set was explored in selected examples to determine the effects of basis set size on the MBE convergence and the BSSE. The CAM-B3LYP\cite{Yanai} functional was also employed to explore the effects of long-range interactions on the $\boldsymbol{G'}$ tensor. Dipole polarizabilities for all systems (and all properties for the 13-water/\textit{S}-methyloxirane system) were computed using only CAM-B3LYP/aug-cc-pVDZ. Specific rotations and dynamic polarizabilities were computed at four common wavelengths: 355, 436, 589, and 633 nm. All calculations were performed using Gaussian 09\cite{g16} using inputs generated by a Psi4\cite{Parrish2017} plugin, which also carried out the subsequent data collection and analysis. This plugin gathers the data from formatted checkpoint files generated by Gaussian 09 to address precision issues brought about by propagation of error as noted in recent studies.\cite{Richard2014,Liu2017a,Richard2018a} Default SCF and CPHF convergence criteria ($10^{-7}$ and $10^{-10}$, respectively) were used with standard pruned integration grids, which include 75 and 35 radial shells, and 302 and 110 angular points per shell for SCF and CPHF, respectively. When compared to tighter convergence ($10^{-12}$) and ``fine'' grids (pruned 75 shell, 302 nodes) for both SCF and CPHF, no appreciable difference was found for the (\textit{S})-methyloxirane system in a seven-water solvent shell (compare Figs.~7 and 13 of the Supporting Information). Additionally, unpruned ``fine'' grids were also tested with the same system; similarly, this had little effect on the convergence of the expansion (compare Figs.~7 and 14 of the Supporting Information). 

    We will use both graphical evaluations and standard deviations of the error relative to the converged results to assess the oscillations present in the MBE. We calculated standard deviations only for the two-body and higher approximations, due to the much larger, non-representative errors associated with the one-body approximation.
We report standard deviations and percent errors for dynamic polarizabilities and specific rotations at 633 nm unless otherwise noted. Plots of data not discussed, such as the interaction energies and electric dipole moments of smaller solvent cages, are available in the Supporting Information. 
